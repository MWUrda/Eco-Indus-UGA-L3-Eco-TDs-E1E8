%\documentclass[ignorenonframetext, compress, 9pt, xcolor=svgnames]{beamer} 
\input{../../../Config_diapos}
\usepackage[svgnames]{xcolor}
\usepackage{tikz}
\usetikzlibrary{shapes.geometric, arrows}
\usepackage{enumerate}   
\usepackage{multirow}
\usepackage{txfonts}
\usepackage{mathrsfs}
\usepackage{pgfplots}
\pgfplotsset{compat = newest}
\usetikzlibrary{positioning, arrows.meta}
\usepgfplotslibrary{fillbetween}
\newcommand{\A}{(0,0) ++(135:2) circle (2)}
\newcommand{\B}{(0,0) ++(45:2) circle (2)}
\DeclareMathOperator{\C}{C}
\DeclareMathOperator{\util}{u}
%\setbeamersize{text margin left=1.5em,text margin right=1.5em} 
%\setbeamersize{text margin left=1.2cm,text margin right=1.2cm} 
\setbeamersize{text margin left=1.5em,text margin right=1.5em} 
%\usepackage{xr}
%\externaldocument{Econometrie1_UGA_P2e}
  \usepackage{eso-pic}
%\newcommand\AtPagemyUpperLeft[1]{\AtPageLowerLeft{%
%\put(\LenToUnit{0.9\paperwidth},\LenToUnit{0.85\paperheight}){#1}}}
%\AddToShipoutPictureFG{
 % \AtPagemyUpperLeft{{\includegraphics[width=1.1cm,keepaspectratio]{logoUGA2020.pdf}}}
%}%

%\setbeamercolor{title}{fg=black}
%\setbeamercolor{frametitle}{fg=black}
%\setbeamercolor{section in head/foot}{fg=black}
%\setbeamercolor{author in head/foot}{bg=Brown}
%\setbeamercolor{date in head/foot}{fg=Brown}
\AtBeginSection[]
  {
    \ifnum \value{framenumber}>1
      \begin{frame}<beamer>
      \frametitle{PLAN}
      \tableofcontents[currentsection]
      \end{frame}
    \else
    \fi
  }
\setbeamertemplate{section page}
{
    \begin{centering}
    \begin{beamercolorbox}[sep=11pt,center]{part title}
    \usebeamerfont{section title}\thesection.~\insertsection\par
    \end{beamercolorbox}
    \end{centering}
}
%\titlegraphic{\includegraphics[width=1cm]{logoUGA2020.pdf}}
\title[]{ \textbf{ÉCONOMIE INDUSTRIELLE}\footnote{Responsable du cours: Sylvain Rossiaud}\\(\textbf{UGA, L3 EGE, S2})}
\subtitle{TRAVAUX DIRIGÉS: TD 1}
\date{\today}
\author{Michal W. Urdanivia\inst{*}}
\institute{\inst{*}UGA, Facult\'e d'\'Economie, GAEL, \\
e-mail:
 \href{
     mailto:michal.wong-urdanivia@univ-grenoble-alpes.fr}{michal.wong-urdanivia@univ-grenoble-alpes.fr}}

%\titlegraphic{\includegraphics[width=1cm]{logoUGA2020.pdf}
%}

\begin{document}

%%% TIKZ STUFF
\usetikzlibrary{positioning}
\usetikzlibrary{snakes}
\usetikzlibrary{calc}
\usetikzlibrary{arrows}
\usetikzlibrary{decorations.markings}
\usetikzlibrary{shapes.misc}
\usetikzlibrary{matrix,shapes,arrows,fit,tikzmark}
\usetikzlibrary{matrix,chains,positioning,decorations.pathreplacing,arrows}
\usetikzlibrary{shapes}
\usetikzlibrary{shapes.geometric, arrows}
\tikzset{   
        every picture/.style={remember picture,baseline},
        every node/.style={anchor=base,align=center,outer sep=1.5pt},
        every path/.style={thick},
        }
\newcommand\marktopleft[1]{
    \tikz[overlay,remember picture] 
        \node (marker-#1-a) at (-.3em,.3em) {};%
}
\newcommand\markbottomright[2]{%
    \tikz[overlay,remember picture] 
        \node (marker-#1-b) at (0em,0em) {};%
}
\tikzstyle{every picture}+=[remember picture] 
\tikzstyle{mybox} =[draw=black, very thick, rectangle, inner sep=10pt, inner ysep=20pt]
\tikzstyle{fancytitle} =[draw=black,fill=red, text=white]
\tikzstyle{observed}=[draw,circle,fill=gray!50]

\begin{frame}
\titlepage
\end{frame}
\begin{frame}
 \tableofcontents
    \end{frame}
%\begin{frame}
%\frametitle{Contenu}
%\tableofcontents[pausesections, pausesubsections]
%\end{frame}

%\section{Qu'est-ce que l’économétrie ? A quoi (à qui) ça sert ?}
%\frame{\sectionpage}
%\begin{frame}
%  \tableofcontents  
%\end{frame}
\section{Introduction}
\frame{\sectionpage}

\begin{frame}
  [allowframebreaks]{Références}
  \begin{itemize}
\item En premier lieu : \href{https://cours.univ-grenoble-alpes.fr/course/view.php?id=5946\#section-1}{Le cours magistral de Sylvain}.
\item Un classique: \cite{Tirole_BookIO_1988}.
\end{itemize}
\end{frame}
\section{Exercice 1}
\frame{\sectionpage}
\begin{frame}[allowframebreaks]{\insertsection}
\framesubtitle{Modèle de base de concurrence à la Bertrand}
\begin{itemize}
    \item 2 firmes identiques produisent un bien homogène.  
    \item Coût de la firme $i=1, 2$:
    \begin{align*}
        c_i(q_i)=cq_i \Rightarrow c^m_i(q_i):= \frac{\partial c_i(q_i)}{\partial q_i} = c, \ \text{avec} \ c > 0.
    \end{align*}
    \item \underline{}\textbf{Remarque:} pour l'exercice, $a=20$, $b=1/5$,   $c=2$.
    \item Demande donnée par: 
    \begin{align*}
        q = \frac{a}{b} -\frac{p}{b}, \ \text{avec} \ a, b > 0.
    \end{align*}
    \item Elles se concurrencent par les prix avec $p_i$ le prix de $i$.
    \item Les biens produits étant homogènes, tous les consommateurs consomment celui produit par la firme 
    qui propose le prix le plus bas. 
    \item Le profit de $i$ est alors(avec $j=1, 2$, $j\neq i$)
    \begin{align*}
        \pi_i &=\left\{
            \begin{array}{ll}
                (p_i-c)\left(  \frac{a}{b} -\frac{p_i}{b} \right) & \ \text{si} \ p_i<p_j.\\
                \frac{1}{2}(p_i-c)\left(  \frac{a}{b} -\frac{p_i}{b} \right) & \ \text{si} \ p_i = p_j.\\
                 0& \ \text{si} \ p_i > p_j.
            \end{array}
        \right.
    \end{align*}
    \item \textbf{\underline{Meilleure réponse de $i$:}}
    \begin{align*}
         p^{mr}_i(p_j) &=
         \left\{
         \begin{array}{ll}
            p_j - \epsilon & \ \text{si} \ p_j  -\epsilon> c\\
            c & \ \text{sinon},
         \end{array}
         \right.
    \end{align*}
    où $\epsilon > 0$.
    \item \textbf{\underline{Résultat(Paradoxe de Bertrand)}}: Le prix d'équilibre $p^*$ est tel que $p^* = p^{mr}_1(p_2) = p^{mr}_2(p_1) =c=c^m_1(q_1) = c^m_2(q_2)$.
    \item En particulier, avec les données de l'exercice $p^* = 2$.
\end{itemize}
\end{frame}

\section{Exercice 2}
\frame{\sectionpage}
\begin{frame}[allowframebreaks]{\insertsection}
\framesubtitle{Modèle de base de concurrence à la Cournot}
\begin{itemize}
\item Deux firmes produisent un bien homogène. 
\item Coûts identiques avec: 
\begin{align}
    c_i(q_i) &= 1 + 2q_i \Rightarrow c_i^m(q_i):=\frac{\partial c_i(q_i)}{\partial q_i}= 2, \ i=1, 2.
    \label{eq1}
\end{align}
\item Demande sur le marché: 
\begin{align}
Q_d(p) = 400 - 100p &\Rightarrow p(Q) \coloneqq Q_d^{-1}(Q)= 4-\frac{Q}{100} \ (\text{Fonction de demande inverse}),
    \label{eq2}
\end{align}
où $Q=q_1+q_2$.
\item Concurrence par les quantités, c.à.d., $q_i$ est la variable de décision de $i$.
\end{itemize}
\end{frame}

\begin{frame}[allowframebreaks]{\insertsection}
\framesubtitle{Modèle de base de concurrence à la Cournot: équilibre}
    \begin{itemize}
        \item Profit $i=1, 2$:
        \begin{align}
            \pi_i(q_i) &= pq_i - c_i(q_i) = \left(\underbrace{4-\frac{Q}{100}}_{ =  p(Q) \ \text{par \eqref{eq2}}}\right)q_i 
            - (\underbrace{1+2q_i}_{=c_i(q_i) \ \text{par \eqref{eq1}}}) \nonumber\\
            &= \left(4-\frac{(q_i + q_j)}{100}\right)q_i - (1+ 2q_i) \nonumber\\
            &= 2q_i - 1 - \frac{(q_j + q_i)}{100}q_i.
            \label{eq3}
        \end{align}
        où $j=1,2$, $j\neq i$.
        \item $i$ maximise \eqref{eq3} par rapport à $q_i$ pour $q_j$ donné. 
        \item Soit $q_i*$ la valeur de $q_i$ où  \eqref{eq3} est maximisée. Elle peut être obtenu à partir de la c.p.o.:
        \begin{align}
          \frac{\partial \pi_i}{\partial q_i}(q_i^*) = 0 \Leftrightarrow 2 -\frac{q_j}{100} - \frac{q_i^*}{50} = 0
             \Rightarrow q_i^* = 100 - \frac{q_j}{2} =: q_i^{mr}(q_j)  \ (\text{Meilleure réponse de $i$}).
             \label{eq4}
        \end{align}
        \item L'équilibre est tel que les deux firmes maximisent leur profit en fixant leurs choix optimaux  $q_1^*$ et $q_2^*$  d'après leurs fonctions de meilleur réponse respectives telles que définies par \eqref{eq4}.
        \item Il est obtenu comme solution du système donné par:
        \begin{align*}
        q_i^* &= q_i^{mr}(q^*_j) , \ i, j = 1, 2; \ i\neq j,
        \end{align*}
        soit,
        \begin{align*}
            \left\{
            \begin{array}{l}
            q_1^* = 100 - \frac{q_2^*}{2}\\
            q_2^* = 100 - \frac{q_1^*}{2}
            \end{array}
            \right.
            &\Rightarrow q_1^* = q_2^* = 200/3 \approx 66.67,
        \end{align*}
        d'où:
        \begin{enumerate}[-]
        \item $Q^* = q_1^* +q_2^* = 400/3\approx 133.33$,
        \item $p^* = p(Q^*) = 16$,
        \item $\pi_1(q_1^*) = \pi_2(q_2^*) \approx 43.67$.
        \end{enumerate}
        \item \textbf{\underline{Commentaire:}} c'est un équilibre de Nash qui résulte de ce que les deux firmes maximisent leurs profits et il est supposé 
        que les deux firmes savent cela et connaissent la forme de leurs fonctions de réaction. 
        Dans ce modèle les quantités sont des \underline{\textbf{substituts stratégiques}}.
    \end{itemize}
\end{frame}   

\begin{frame}[allowframebreaks]{\insertsection}
\framesubtitle{Cournot avec asymmétrie sur les coûts}
    \begin{itemize}
        \item Fonctions de coûts: 
        \begin{align*}
            c_1(q_1) &= 1+2q_1\\
            c_2(q_2) &= 1+\frac{5q_1}{2}.
        \end{align*}
        \item 1 présente un avantage en termes de coût avec un coût marginal inférieur à celui de 2.
        \item La meilleure réponse de 1 est donné par \eqref{eq4}(fonction de coût inchangée par rapport à cette question). 
        \item Celle de 2 peut être obtenue selon la mêmes démarche que dans la question précédente. On obtient: 
        \begin{align*}
            q_2^{mr}(q_1) &= 75 - \frac{q_1}{2}.
        \end{align*}
        \item Le vecteur de prix d'équilibre $(q_1^{*a}, q_2^{*a})$ est alors obtenu comme solution dun système: 
        \begin{align*}
        \left\{
        \begin{array}{l}
        q_1^{*a} = 100 - \frac{q_2^{*a}}{2}\\
        q_2^{*a} = 75 - \frac{q_1^{*a}}{2}
        \end{array}
        \right.
        &\Rightarrow 
        \left\{
        \begin{array}{l}
        q_1^{*a} =83.33,\\
         q_2^{*a} = 33.34.
        \end{array} \right.
        \end{align*}
       d'où 
       $Q^{*a} = q_1^{*a} + q_2^{*a} = 116.67$, $p^{*a} = p(Q^{*a}) = 2.83$, $\pi_1(q_1^{*a}) = 68.16$, $\pi_2(q_2^{*a}) = 10$.
 
    \end{itemize}

\end{frame}

\section{Exercice 3}
\frame{\sectionpage}
\begin{frame}[allowframebreaks]{\insertsection}
\framesubtitle{Modèle oligopolistique séquentiel}
\begin{itemize}
\item Deux firmes produisent des biens homogènes où: 
\begin{enumerate}[-]
\item la firme 1 est leader:  elle fixe son niveau de production $q_1$,
\item la firme 2 est suiveuse : elle choisit son niveau de production $q_2$ après avoir observé $q_1$
\end{enumerate}
\item Fonction de demande inverse: 
\begin{align}
    P(Q) &= 200 - Q, \ Q=q_1+q_2.
\label{eq1}
\end{align}
\item Coûts fixes nuls, et coûts marginal constant:
\begin{align}
    c_i(q_i) &= 60q_i \Rightarrow c^{m}_i(q_i) :=\frac{\partial c_i}{\partial q_1}(q_i) = 60, \ \text{pour} \ i= 1, 2.
    \label{eq2}
\end{align}
\end{itemize}
\end{frame}


\begin{frame}[allowframebreaks]{\insertsection}
\framesubtitle{(a)}

    \begin{itemize}
        \item \underline{\textbf{Fonction de meilleure réponse de la firme 2}}
        \begin{enumerate}[$\cdot$]
        \item Profit de $2$: 
        \begin{align}
            \pi_2(q_1, q_2) &= Pq_2 - c_2(q_2) = \underbrace{\left(200-(q_1+q_2)\right)}_{\text{en raison de \eqref{eq1}}}q_2 - 
            \underbrace{60q_2}_{\text{par \eqref{eq2}}},
            \label{eq3}
        \end{align}
        \item $2$ décide du niveau $q_2$ de sorte à maximiser $\pi_2$. Notons $q_2^*$ ce niveau qui est défini par:
        \begin{align*}
            q^*_2 &= \argmax_{q_2} \pi_2(q_1, q_2) ,
        \end{align*}
         et vérifie la c.p.o.,
         \begin{align}
             \frac{\partial \pi_2(q_1, q_2^*)}{\partial q_2} = 0 \Leftrightarrow 140-q_1-2q^*_2 = 0,
             \label{eq4}
         \end{align}
        qui défini $q^*_2$ comme fonction de $q_1$, c.à.d., \textbf{\underline{la fonction de meilleure réponse de 2}}. 
        On la note $q^{mr}_2(q_1)$ et elle est donc donnée en utilisant \eqref{eq4} par,
        \begin{align}
            q^*_2 & = 70 - \frac{q_1}{2}=: q^{mr}_2(q_1)
           \label{eq5}
        \end{align}
        \end{enumerate}
    \end{itemize}
\end{frame}

\begin{frame}[allowframebreaks]{\insertsection}
\framesubtitle{(b)}

    \begin{itemize}
\item \underline{\textbf{ Équilibre}}
        \begin{enumerate}[$\cdot$]
            \item La fonction de profit de 1 présente une forme similaire à celle de 2:
            \begin{align}
                \pi_1\left(q_1, q_2\right) &= Pq_1 - c_1(q_1) = \left(200-(q_1+q_2)\right)q_1 - 60q_1.
                \label{eq6}
            \end{align}
            \item En outre 1 connaît le niveau de production que 2 fixe en fonction du sien, 
            donné par \eqref{eq5} et ce faisant \eqref{eq6} peut s'écrire:
            \begin{align}
                \pi_1\left(q_1, q^{mr}_2(q_1)\right) &= \left(200-(q_1+q_2)\right)q_1 - 60q_1 \nonumber\\
                &= 140q_1 - q_1^2 - q^{mr}_2(q_1)q_1 \nonumber \\
                &= 140q_1 - q_1^2 - \underbrace{\left(70 - \frac{q_1}{2}\right)}_{=q^{mr}_2(q_1)}q_1\nonumber\\
                &= 70q_1 - \frac{q_1^2}{2} =: \pi(q_1),
                \label{eq6}
            \end{align}
            qui ne dépends que de $q_1$. 
            \item $1$ maximise donc \eqref{eq6}. Le maximum est atteint pour $q_1^*$ qui vérifie la c.p.o.:  
            \begin{align}
                \frac{\partial \pi_1(q_1^*)}{\partial q_1} =0 & \Leftrightarrow 
                70 - q_1^* =0 \Rightarrow q_1^* = 70.
                \label{eq7}
            \end{align}
            \item Il en résulte par \eqref{eq5}:
            \begin{align*}
                q^*_2 &= q^{mr}_2(q_1^*) = 70 - \frac{q_1^*}{2}=35.
            \end{align*}
            \item Ceci qui implique une quantité produite à l'équilibre de $Q^* = q_1^* + q_2^* = 105$, 
            un prix de $P^* = P(Q^*) = 95$, pour de profits $\pi_1(q_1^*) = 2450$, et 
            $\pi_2(q_2^*) = 1225$.
        \end{enumerate}
    \end{itemize}
\end{frame}
\begin{frame}[allowframebreaks]{\insertsection}
\framesubtitle{(c)}
   \begin{itemize}
\item \underline{\textbf{Comparaison avec Cournot}}
 \begin{enumerate}[$\cdot$]
\item Pour rappel dans un Cournot de base(voir cours, et TD1) avec:
\begin{enumerate}[$\cdot$]
\item des coûts $c_i(q_i) = cq_i$, $c>0$, $i=1, 2$,
\item et une demande $P(Q) = a - bQ$, $Q=q_1+q_2$, $a, b > 0$,
\end{enumerate}
\item L'équilibre est (où l'indice sup "ec" est pour "équilibre de Cournot"):
\begin{align*}
    q_1^{(ec)} = q_2^{(ec)} = q^{(ec)} = \frac{a-c}{3b},
\end{align*}
d'où ici(avec $a=300, b=1, c=20$): 
\begin{enumerate}[-]
\item $q_1^{(ec)}=q_2^{(ec)}\approx 46.67$, 
\item $Q^{(ec)} = q_1^{(ec)}+q_2^{(ec)} \approx 93.33$,
\item $P^{(ec)} \approx 106.67$.
\item $\pi^{(ec)}_1 = \pi^{(ec)}_2 = 2178$.
\end{enumerate}

\item \textbf{\underline{Remarque}}: en Cournot les meilleures réponses sont pour $i=1, 2$:
\begin{align*}
    q_i^{mr}&=\frac{(a-c)}{2b} - \frac{q_j}{2}, \ j=1, 2, j\neq i.
\end{align*} 
\item Par rapport au modèle de Cournot la firme 1 possède l'avantage de jouer en premier ce que la firme 2 doit anticiper.
\end{enumerate}
\end{itemize}
\end{frame}
\begin{frame}[allowframebreaks]{Références}
\bibliographystyle{jpe}
\bibliography{../../../Biblio}
\end{frame}

\end{document}
