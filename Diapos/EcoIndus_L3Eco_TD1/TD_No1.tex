%\documentclass[ignorenonframetext, compress, 9pt, xcolor=svgnames]{beamer} 
\input{../Config_diapos}
\usepackage{color}
\usepackage{tikz}

\usepackage{enumerate}   


%\setbeamersize{text margin left=1.5em,text margin right=1.5em} 
%\setbeamersize{text margin left=1.2cm,text margin right=1.2cm} 
\setbeamersize{text margin left=1.5em,text margin right=1.5em} 
%\usepackage{xr}
%\externaldocument{Econometrie1_UGA_P2e}
  \usepackage{eso-pic}
%\newcommand\AtPagemyUpperLeft[1]{\AtPageLowerLeft{%
%\put(\LenToUnit{0.9\paperwidth},\LenToUnit{0.85\paperheight}){#1}}}
%\AddToShipoutPictureFG{
 % \AtPagemyUpperLeft{{\includegraphics[width=1.1cm,keepaspectratio]{logoUGA2020.pdf}}}
%}%

%\setbeamercolor{title}{fg=black}
%\setbeamercolor{frametitle}{fg=black}
%\setbeamercolor{section in head/foot}{fg=black}
%\setbeamercolor{author in head/foot}{bg=Brown}
%\setbeamercolor{date in head/foot}{fg=Brown}
\setbeamertemplate{section page}
{
    \begin{centering}
    \begin{beamercolorbox}[sep=11pt,center]{part title}
    \usebeamerfont{section title}\insertsection\par
    \end{beamercolorbox}
    \end{centering}
}
%\titlegraphic{\includegraphics[width=1cm]{logoUGA2020.pdf}}
\title[]{ \textbf{Économie Industrielle} \\ (UGA, L3 Éco, S2) \\ (responsable du cours: Sylvain Rossiaud)}
\subtitle{Travaux dirigés: No 2\\ 
Fondamentaux de l'économie industrielle\\(éléments de correction d'exercices)}
\date{\today}
\author{Michal W. Urdanivia\inst{*}}
\institute{\inst{*}UGA, Facult\'e d'\'Economie, GAEL, \\
e-mail:
 \href{
     mailto:michal.wong-urdanivia@univ-grenoble-alpes.fr}{michal.wong-urdanivia@univ-grenoble-alpes.fr}}

%\titlegraphic{\includegraphics[width=1cm]{logoUGA2020.pdf}
%}

\begin{document}

%%% TIKZ STUFF
\usetikzlibrary{positioning}
\usetikzlibrary{snakes}
\usetikzlibrary{calc}
\usetikzlibrary{arrows}
\usetikzlibrary{decorations.markings}
\usetikzlibrary{shapes.misc}
\usetikzlibrary{matrix,shapes,arrows,fit,tikzmark}
\usetikzlibrary{shapes}
\tikzset{   
        every picture/.style={remember picture,baseline},
        every node/.style={anchor=base,align=center,outer sep=1.5pt},
        every path/.style={thick},
        }
\newcommand\marktopleft[1]{
    \tikz[overlay,remember picture] 
        \node (marker-#1-a) at (-.3em,.3em) {};%
}
\newcommand\markbottomright[2]{%
    \tikz[overlay,remember picture] 
        \node (marker-#1-b) at (0em,0em) {};%
}
\tikzstyle{every picture}+=[remember picture] 
\tikzstyle{mybox} =[draw=black, very thick, rectangle, inner sep=10pt, inner ysep=20pt]
\tikzstyle{fancytitle} =[draw=black,fill=red, text=white]
\tikzstyle{observed}=[draw,circle,fill=gray!50]



\begin{frame}
\titlepage
\end{frame}

%\begin{frame}
%\frametitle{Contenu}
%\tableofcontents[pausesections, pausesubsections]
%\end{frame}

%\section{Qu'est-ce que l’économétrie ? A quoi (à qui) ça sert ?}
%\frame{\sectionpage}
%\begin{frame}
%  \tableofcontents  
%\end{frame}


\section{Exercice 1}
\frame{\sectionpage}
\begin{frame}[allowframebreaks]{Modèle de base de concurrence à la Bertrand}
\begin{itemize}
    \item 2 firmes identiques produisent un bien homogène.  
    \item Coût de la firme $i=1, 2$:
    \begin{align*}
        c_i(q_i)=cq_i \Rightarrow c^m_i(q_i):= \frac{\partial c_i(q_i)}{\partial q_i} = c, \ \text{avec} c > 0.
    \end{align*}
    \item Demande donnée par: 
    \begin{align*}
        q = \frac{a}{b} -\frac{p}{b}, \ \text{avec} \ a, b > 0.
    \end{align*}
    \item Elles se concurrencent par les prix avec $p_i$ le prix de $i$.
    \item Les biens produits étant homogènes, tous les consommateurs consomment celui produit par la firme 
    qui propose le prix le plus bas. 
    \item Le profit de $i$ est alors(avec $j=1, 2$, $j\neq i$)
    \begin{align*}
        \pi_i &=\left\{
            \begin{array}{ll}
                p_i(p_i-c)\left(  \frac{a}{b} -\frac{p_i}{b} \right) & \ \text{si} \ p_i<p_j.\\
                \frac{1}{2}p_i(p_i-c)\left(  \frac{a}{b} -\frac{p_i}{b} \right) & \ \text{si} \ p_i = p_j.\\
                 0& \ \text{si} \ p_i > p_j.
            \end{array}
        \right.
    \end{align*}
    \item \textbf{\underline{Meilleure réponse de $i$:}}
    \begin{align*}
         p^{mr}_i(p_j) &=
         \left\{
         \begin{array}{ll}
            p_j - \epsilon & \ \text{si} \ p_j  -\epsilon> c\\
            c & \ \text{sinon},
         \end{array}
         \right.
    \end{align*}
    où $\epsilon > 0$.
    \item \textbf{\underline{Résultat(Paradoxe de Bertrand)}}: Le prix d'équilibre $p^*$ est tel que $p^* = p^{mr}_1(p_2) = p^{mr}_2(p_1) =c=c^m_1(q_1) = c^m_2(q_2)$.
\end{itemize}
\end{frame}

\section{Exercice 2}
\frame{\sectionpage}
\begin{frame}[allowframebreaks]{Modèle de base de concurrence à la Cournot}
\begin{itemize}
\item Deux firmes produisent un bien homogène. 
\item Coûts identiques avec: 
\begin{align}
    c_i(q_i) &= 1 + 4q_i \Rightarrow c_i^m(q_i):=\frac{\partial c_i(q_i)}{\partial q_i} = 4, \ i=1, 2.
    \label{eq1}
\end{align}
\item Demande sur le marché: 
\begin{align}
Q(p) &= 400 - 10p \Rightarrow p(Q) = 40-\frac{Q}{10} \ (\text{Fonction de demande inverse}),
    \label{eq2}
\end{align}
où $Q=q_1+q_2$.
\item Concurrence par les quantités, c.à.d., $q_i$ est la variable de décision de $i$.
\end{itemize}
\end{frame}

\begin{frame}[allowframebreaks]{(1) Équilibre}
    \begin{itemize}
        \item Profit $i=1, 2$:
        \begin{align}
            \pi_i(q_i) &= pq_i - c_i(q_i) = \left(\underbrace{40-\frac{Q}{10}}_{ =  p(Q) \ \text{par \eqref{eq2}}}\right)q_i 
            - (\underbrace{1+4q_i}_{=c_i(q_i) \ \text{par \eqref{eq1}}}) \nonumber\\
            &= \left(40-\frac{q_i + q_j}{10}\right)q_i - (1+ 4q_i) \nonumber\\
            &= 36q_i - 1 - \frac{(q_j + q_i)}{10}q_i.
            \label{eq3}
        \end{align}
        où $j=1,2$, $j\neq i$.
        \item $i$ maximise \eqref{eq3} par rapport à $q_i$ pour $q_j$ donné. 
        \item Soit $q_i*$ la valeur de $q_i$ où  \eqref{eq3} est maximisée. Elle peut être obtenu à partir de la c.p.o.:
        \begin{align}
            \frac{\partial \pi_i(q_i^*)}{\partial q_i} = 0 \Leftrightarrow 36 -\frac{q_j}{10} - \frac{q_i^*}{5} = 0
             \Rightarrow q_i^* = 180 - \frac{q_j}{2} =: q_i^{mr}(q_j)  \ (\text{Meilleure réponse de $i$}).
             \label{eq4}
        \end{align}
        \item L'équilibre est tel que les deux firmes maximisent leur profit, donc telle $q_1^*$ et $q_2^*$ soient définies d'après 
        \eqref{eq4}: 
        \begin{align*}
            \left\{
            \begin{array}{l}
            q_1^* = 180 - \frac{q_2^*}{2}\\
            q_2^* = 180 - \frac{q_1^*}{2}
            \end{array}
            \right.
            &\Rightarrow q_1^* = q_2^* = 120,
        \end{align*}
        d'où:
        \begin{enumerate}[-]
        \item $Q^* = q_1^* +q_2^* = 240$,
        \item $p^* = p(Q^*) = 16$,
        \item $\pi_1(q_1^*) = \pi_2(q_2^*) = 1439$.
        \end{enumerate}
        \item \textbf{\underline{Commentaire:}} c'est un équilibre de Nash qui résulte de ce que les deux firmes maximisent leurs profits et il est supposé 
        que les deux firmes savent cela et connaissent la forme de leurs fonctions de réaction. 
        Dans ce modèle les quantités sont des \underline{\textbf{substituts stratégiques}}.
    \end{itemize}
\end{frame}   


    \begin{frame}[allowframebreaks]{(2) Asymmétrie sur les coûts}
    \begin{itemize}
        \item Fonctions de coûts:
        \begin{align*}
            c_1(q_1) &= 1+4q_1\\
            c_2(q_2) &= 1+5q_1.
        \end{align*}
        \item 1 présente un avantage en termes de coût avec un coût marginal inférieur à celui de 2.
        \item La meilleure réponse de 1 est donné par \eqref{eq4}(fonction de coût inchangée par rapport à cette question). 
        \item Celle de 2 peut être obtenue selon la mêmes démarche que dans la question précédente. On obtient: 
        \begin{align*}
            q_2^{mr}(q_1) &= 175 - \frac{q_1}{2}.
        \end{align*}
        \item Le vecteur de prix d'équilibre $(q_1^{*a}, q_2^{*a})$ est alors obtenu comme solution dun système: 
        \begin{align*}
        \left\{
        \begin{array}{l}
        q_1^{*a} = 180 - \frac{q_2^{*a}}{2}\\
        q_2^{*a} = 175 - \frac{q_1^{*a}}{2}
        \end{array}
        \right.
        &\Rightarrow 
        \left\{
        \begin{array}{l}
        q_1^{*a} =123.3,\\
         q_2^{*a} = 113.3.
        \end{array} \right.
        \end{align*}
       d'où 
       $Q^{*a} = q_1^{*a} + q_2^{*a} = 236.6$, $p^{*a} = p(Q^{*a}) = 16.3$, $\pi_1(q_1^{*a}) = 1515.59$, $\pi_2(q_2^{*a}) = 1279.29$.
 
    \end{itemize}
\end{frame}   

\end{document}