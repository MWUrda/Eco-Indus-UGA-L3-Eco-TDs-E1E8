%\documentclass[ignorenonframetext, compress, 9pt, xcolor=svgnames]{beamer} 
\input{../Config_diapos}
\usepackage{color}
\usepackage{tikz}
\usetikzlibrary{shapes.geometric, arrows}
\usepackage{enumerate}   
\usepackage{multirow}
%\setbeamersize{text margin left=1.5em,text margin right=1.5em} 
%\setbeamersize{text margin left=1.2cm,text margin right=1.2cm} 
\setbeamersize{text margin left=1.5em,text margin right=1.5em} 
%\usepackage{xr}
%\externaldocument{Econometrie1_UGA_P2e}
  \usepackage{eso-pic}
%\newcommand\AtPagemyUpperLeft[1]{\AtPageLowerLeft{%
%\put(\LenToUnit{0.9\paperwidth},\LenToUnit{0.85\paperheight}){#1}}}
%\AddToShipoutPictureFG{
 % \AtPagemyUpperLeft{{\includegraphics[width=1.1cm,keepaspectratio]{logoUGA2020.pdf}}}
%}%

%\setbeamercolor{title}{fg=black}
%\setbeamercolor{frametitle}{fg=black}
%\setbeamercolor{section in head/foot}{fg=black}
%\setbeamercolor{author in head/foot}{bg=Brown}
%\setbeamercolor{date in head/foot}{fg=Brown}
\setbeamertemplate{section page}
{
    \begin{centering}
    \begin{beamercolorbox}[sep=11pt,center]{part title}
    \usebeamerfont{section title}\thesection.~\insertsection\par
    \end{beamercolorbox}
    \end{centering}
}
%\titlegraphic{\includegraphics[width=1cm]{logoUGA2020.pdf}}
\title[]{ \textbf{Économie Industrielle}\footnote{responsable du cours: Sylvain Rossiaud} \\ (UGA, L3 Éco, S2) \\ }
\subtitle{Travaux dirigés: No 6\\ FUSIONS\\
(éléments de correction)}
\date{\today}
\author{Michal W. Urdanivia\inst{*}}
\institute{\inst{*}UGA, Facult\'e d'\'Economie, GAEL, \\
e-mail:
 \href{
     mailto:michal.wong-urdanivia@univ-grenoble-alpes.fr}{michal.wong-urdanivia@univ-grenoble-alpes.fr}}

%\titlegraphic{\includegraphics[width=1cm]{logoUGA2020.pdf}
%}

\begin{document}

%%% TIKZ STUFF
\usetikzlibrary{positioning}
\usetikzlibrary{snakes}
\usetikzlibrary{calc}
\usetikzlibrary{arrows}
\usetikzlibrary{decorations.markings}
\usetikzlibrary{shapes.misc}
\usetikzlibrary{matrix,shapes,arrows,fit,tikzmark}
\usetikzlibrary{shapes}
\usetikzlibrary{shapes.geometric, arrows}
\tikzset{   
        every picture/.style={remember picture,baseline},
        every node/.style={anchor=base,align=center,outer sep=1.5pt},
        every path/.style={thick},
        }
\newcommand\marktopleft[1]{
    \tikz[overlay,remember picture] 
        \node (marker-#1-a) at (-.3em,.3em) {};%
}
\newcommand\markbottomright[2]{%
    \tikz[overlay,remember picture] 
        \node (marker-#1-b) at (0em,0em) {};%
}
\tikzstyle{every picture}+=[remember picture] 
\tikzstyle{mybox} =[draw=black, very thick, rectangle, inner sep=10pt, inner ysep=20pt]
\tikzstyle{fancytitle} =[draw=black,fill=red, text=white]
\tikzstyle{observed}=[draw,circle,fill=gray!50]

\begin{frame}
\titlepage
\end{frame}
\begin{frame}
 \tableofcontents
    \end{frame}
%\begin{frame}
%\frametitle{Contenu}
%\tableofcontents[pausesections, pausesubsections]
%\end{frame}

%\section{Qu'est-ce que l’économétrie ? A quoi (à qui) ça sert ?}
%\frame{\sectionpage}
%\begin{frame}
%  \tableofcontents  
%\end{frame}


\section{Rappels sur le modèle de Cournot(de base) et à $N$ firmes}
\frame{\sectionpage}
\begin{frame}[allowframebreaks]{Modèle}
\begin{itemize}
\item \textbf{\underline{Référence}}: \cite{belleflamme_peitz_2015}, chapitre 3.
\item $N$ firmes avec des fonctions de coûts symétriques à coûts marginaux constants:
\begin{align*}
    c_i(q_i) &=cq_i \Rightarrow c^m_i(q_i):= \frac{\partial c}{\partial q_i}(q_i) = c, \ c > 0, i=1, \ldots, N.
\end{align*}
\item La demande est représenté par une fonction demande inverse linéaire:
\begin{align*}
p(Q) &= a-bQ, \ a, b > 0, Q = \sum_{i=1}^N q_i
\end{align*}
\item Le profit de chaque firme s'écrit alors: 
\begin{align*}
\pi_i(q_i, q_{(-i)}) &= p(Q)q_i - c_i(q_i) =  (a-bQ)q_i - cq_i
\end{align*}
où on note $q_{(-i)}$ le vecteur $(q_1, q_2, \ldots, q_N)$ sans sa ième composante $q_i$.
\item Bien que $\pi_i$ dépende des quantités offertes par les autres firmes, 
la firme $i$ maximise $\pi_i$ par rapport à $q_i$ qui est sa seule variable de décision.
\item Autrement dit, chaque firme maximise son profit étant donné les décisions des autres.
\end{itemize}
\end{frame}

\begin{frame}[allowframebreaks]{Fonctions de meilleure réponse}
    \begin{itemize}
        \item La c.p.o. devant être vérifiée par le choix optimal de la firme $i$, $q_i^*$ s'écrit: 
        \begin{align}
            \frac{\partial \pi_i}{\partial q_i}(q_i^*) =0 &\Leftrightarrow a-2bq_i -b\sum_{j=1, j\neq i }^N q_j - c = 0\\
            &\Leftrightarrow  a-2bq_i^* -b\left(Q - q_i^*\right) - c= 0,
            \label{eq1}
        \end{align}
         qui définit le choix optimal de la firme $i$ comme une fonction(implicite) de $q_{(-i)}$ qui apparaît 
         dans $Q$. 
         \item Cette fonction est la meilleure réponse de la firme $i$ qu'on note $q^{mr}_i(q_{(-i)})$ avec 
         $q_i^* = q^{mr}_i(q_{(-i)})$. Elle nous est donnée en utilisant \eqref{eq1} par: 
        \begin{align} 
            q_i^{mr}(q_{(-i)}) &= \frac{(a-c)}{b} - Q, \ i=1, \ldots, N.
            \label{eq2}
        \end{align}
           \item L'équilibre du marché est un équilibre de Nash du jeux en information complète, où
           la stratégie de chaque joueur est donné par \eqref{eq2}. 
           Autrement dit le vecteur des quantités d'équilibre $q_1^*, \ldots, q_N^*$ vérifie 
           \begin{align}
               q_i^*&= q_i^{mr}(q_{(-i)}^*), \ i=1, \ldots, N.
               \label{eq3}
           \end{align}
           où $q_i^{mr}$ est donnée par \eqref{eq2}. En notant $Q^*$ la quantité totale offerte 
           à l'équilibre \eqref{eq3} s'écrit:
           \begin{align*}
            q_i^*&=\frac{(a-c)}{b} - Q^*, \ i=1, \ldots, N.
           \end{align*}
           \item On peut utiliser cette dernière égalité et voir que $Q^*$ vérifie:
           \begin{align*}
            \sum_{i=1}^N  q_i^* = N\left(\frac{(a-c)}{b} - Q^*\right)
            &\Leftrightarrow Q^* = N\frac{(a-c)}{b} - NQ^*
            \Rightarrow  Q^* = \frac{N}{(N+1)}\frac{(a-c)}{b}.
           \end{align*}
           \item On obtient alors les quantités offertes par chaque firme à l'équilibre en utilisant
            \eqref{eq2}:
            \begin{align*} 
                q_i^{mr}(q_{(-i)^*}) &= \frac{(a-c)}{b} - Q^* = \frac{(a-c)}{(N+1)b}, \ i=1, \ldots, N.
            \end{align*}
            ce qui permets de calculer le prix et profits à l'équilibre: 
            \begin{align*}
                p^* &=p(Q^*) = a-bQ^* = a-b\left(\frac{N}{(N+1)}\frac{(a-c)}{b}\right)=\frac{a+Nc}{N+1},\\
                \pi_i^* &=\pi_i(q_i^*,q_{(-i)}^*)=p^*q_i^* - cq_i^* = (p^*-c)q_i^* =\frac{1}{b}\left(\frac{a-c}{N+1}\right)^2
            \end{align*}    
    \end{itemize}
\end{frame}

\begin{frame}[allowframebreaks]{Fusion}
    \begin{itemize}
    \item \textbf{\underline{Référence}}: \cite{belleflamme_peitz_2015}, chapitre 15.
    \item Supposons que $K<N$  parmi les $N$ firmes fusionnent.
    \item Le nombre de firmes est alors:
    \begin{align*}
        \tilde{N} &= \underbrace{N-K}_{\substack{\text{firmes qui}\\\text{ne fusionnent pas}}} + \underbrace{1}_{\text{firme fusionnée}}
    \end{align*}
    \item Dans ce cas d'après les résultats précédents le profit de chaque firme est à l'équilibre:
    \begin{align*}
        \pi_{i, \tilde{N}}^* &= \frac{1}{b}\left(\frac{a-c}{\tilde{N}+1}\right)^2
        = \frac{1}{b}\left(\frac{a-c}{N-K+2}\right)^2.
    \end{align*}
    \item On se rappelle que leur profit avant la fusion est pour chacune:
    \begin{align*}
        \pi_i^* &= \frac{1}{b}\left(\frac{a-c}{N+1}\right)^2
    \end{align*}
    \item Ce qui donne un profit total pour les ayant fusionné avant la fusion de:
    \begin{align*}
        K\pi_i^* &= K\frac{1}{b}\left(\frac{a-c}{N+1}\right)^2
    \end{align*}
    \item Pour que la fusion soit profitable il faut alors que:
    \begin{align*}
        \pi_{i, \tilde{N}}^* \geq K\pi_i^*& \Leftrightarrow
         \frac{1}{b}\left(\frac{a-c}{N-K+2}\right)^2  \geq K\frac{1}{b}\left(\frac{a-c}{N+1}\right)^2\\
         &\Leftrightarrow \left(K-1)(-K^2+(2N-3)K-(N+1)^2\right) \geq 0\\
         &\Leftrightarrow K > \frac{1}{2}\left(2N+3-\sqrt{4N+5}\right) > 0.8N
    \end{align*}
    \item C'est la \textbf{règle du 80\%}.
\end{itemize}
\end{frame}

\section{TD: Partie 1}
\frame{\sectionpage}
\begin{frame}[allowframebreaks]{Exercice}
    \begin{itemize}
        \item Fonction de demande inverse: $p(Q) = 200-Q$, $Q=q_1+\ldots q_{10}$, 
        \item Fonction de coût: $c_i(q_i) = 40q_i$, pour $i=1, \ldots, 10$.
    \end{itemize}
    \begin{enumerate}
        \item $N = 10$: d'après ce qui précède:
        \begin{enumerate}[$\star$]
        \item $Q^*_{(N=10)} = 145.5$, 
        \item $p^*_{(N=10)}=54.5$, 
        \item $q_{(i, N=10)}^* = 14.5$,  
        \item $\pi_{(i, N=10)}^*= 211.6$. 
        \item Ce dernier chiffre donnant un surplus des producteurs de $SP_{(i, N=10)}^* = N\pi_{(N=10)}^*=2116$.
        \item  On calcule aussi le surplus du consommateur qui dans le cas d'une fonction de demande inverse 
        linéaire $p(Q) = a-bQ$ est donné par $SC(Q) = \frac{bQ^2}{2}$(voir \citet{belleflamme_peitz_2015}, chapitre 2).\\ Ici à l'équilibre 
        pour $Q^*_{(N=10)} = 145.5$,
         ce surplus est $SC_{(N=10)}^* = SC(Q_{(N=10)}^*) = \frac{145.5^2}{2}  = 10585.1$.
        \end{enumerate}
        
        \item 3 firmes fusionnent et par conséquent $N=8$ désormais, d'où:
        \begin{enumerate}[$\star$]
            \item $Q^*_{N=8} = 142.2$, 
            \item $p^*_{(N=8)}=57.8$, 
            \item $q_{(i, N=8)}^* = 17.8$,  
            \item $\pi_{(i, N=8)}^*= 316$. 
            \item Ce dernier chiffre donnant un surplus des producteurs de 
            $SP^* = N\pi_{(i, N=8)}^*=2528$.
            \item On calcule aussi le surplus du consommateur  
            $SC_{(N=8)}^* = SC(Q_{(N=8)}^*) = \frac{142.3^2}{2} = 10110.4$.
            \framebreak
            \item Ces résultats illustrent la règle du 80\% puisque le profit de la firme fusionnée 
            est inférieur au profit total des trois firmes avant la fusion(soit $316$ 
            après fusion contre $3\times 211.6$). 
            \item Le profit des firmes ne participant pas à la fusion augmente. 
            Elles bénéficient indirectement de celle-ci.
            \item On note aussi que le surplus du consommateur diminue 
            avec la fusion tandis que celui des firmes augmente.
        \end{enumerate}

        \item 3 firmes fusionnent et deviennent leader: concurrence du type Stackelberg. 
        \begin{enumerate}[$\star$]
            \item Jeux séquentiel: le leader joue d'abord, les autres firmes suivent. 
            \item Résolution par induction à rebours: d'abord le problème d'optimisation 
            des firmes qui suivent étant donné le choix du leader, ensuite le problème d'optimisation
            du leader.
            \item Supposons que les trois firmes qui fusionnent produisent la quantité $q_L$. 
            \item \textbf{Dernière étape}: considérons une firme $i$ parmi les firmes qui n'ont pas fusionné, 
            et notons $q_i$ la quantité qu'elle offre. 
            \item La quantité totale offerte peut s'écrire:
            \begin{align*}
                Q&= q_i + q_L + \sum_{j=1; j\neq i}^7 q_j =  q_i + q_L + \underbrace{q_{-j}}_{:=\sum_{j=1; j\neq i}^7 q_j}
            \end{align*}
            \item La demande peut s'écrire alors:
            \begin{align*}
                p(Q) &= 200 - ( q_{-i} + q_L + q_i ).
            \end{align*}
            \item Et le profit de la firme $i$ s'écrit:
            \begin{align*}
                \pi_i^s(q_i) &=  \left[200 - ( q_{-i} + q_L + q_i ) - 40\right]q_i.
            \end{align*}
            \item La c.p.o. associée à la maximisation du profit permet de définir la fonction
            de meilleure réponse de la firme $i$:
            \begin{align*}
             \frac{\partial \pi_i^s}{\partial q_i}(q_i^{s*}) = 0 &\Rightarrow 
             q_i^{s*} = q_i^{mr^s}(q_{-i}, q_L) = 80 - \frac{(q_{-i} + q_L)}{2}.
             \end{align*}
             \item À l'équilibre toutes les firmes jouent leur meilleure stratégie et on note 
             $q_{-i}^{s*}$ l'analogue à l'équilibre de $q_{-i}$, et $q_L^*$ la quantité produite 
             à l'équilibre pas la firme leader(le trois ayant fusionné). Donc:
             \begin{align*}
                q_i^{s*} = 80 - \frac{(q_{-i}^{s*} + q_L^*)}{2} 
             \end{align*}
            \item Notons $Q^{s*}$ la quantité totale offerte par toutes les firmes 
            qui n'ont pas fusionné à l'équilibre: 
            \begin{align*}
                Q^{s*} &= \sum_{i=1}^7 q_i^{s*}.
            \end{align*}
            \item La condition précédente peut alors s'écrire:
            \begin{align*}
                q_i^{s*} &= 80 - \frac{(Q^{s*} - q_i^{s*} + q_L^*)}{2} \\
                &\Leftrightarrow  q_i^{s*} = 160 - Q^{s*} - q_L^* \\
                &\Leftrightarrow \sum_{i=1}^7 q_i^{s*} = 7(160 - Q^{s*} - q_L^* )\\
                &\Leftrightarrow  Q^{s*} = 7(160 - Q^{s*} - q_L^* )\\
                &\Leftrightarrow 8Q^{s*} = 7(160 - q_L^*)\\
                &\Leftrightarrow 8Q^{s*} = 7(160 - q_L^*)\\
                &\Leftrightarrow Q^{s*} = \frac{7}{8}(160 - q_L^*)
             \end{align*}
             \item Cette quantité est la quantité produite par toutes les firmes qui n'ont pas 
             fusionné à l'équilibre. 
             \item \textbf{1ère étape}: maximisation du profit du leader. 
             \item Son profit peut s'écrire:
             \begin{align*}
                 \pi_L(q_L) &= (160 - (q_L + \sum_{i=1}^7 q_i))q_L.
             \end{align*}
             \item En procédant comme habituellement(calcul de la c.p.o) et en notant $q_L^*$ la quantité optimale 
             de la firme leader et $q^{mr}_L(\tilde{q})$ sa fonction de meilleure réponse avec $\tilde{q}:=(q_1, \ldots, q_7)$ on obtient: 
             \begin{align*}
              q^{mr}_L(\tilde{q}) = 80 - \frac{\sum_{i=1}^7 q_i}{2}
             \end{align*}
             \item L'équilibre correspond alors à:
             \begin{align*}
               q_L^* = q^{mr}_L(\tilde{q}^{s*}) = 80 - \frac{\sum_{i=1}^7 q_i^{s*}}{2} = 80 - \frac{Q^{s*}}{2}&=\frac{1}{2}\left[ \frac{7}{8}(160 - q_L^*)\right]\\
               &\Rightarrow q_L^* 
            \end{align*}

        \end{enumerate}
        



    \end{enumerate}

\end{frame}

\begin{frame}[allowframebreaks]{References}
    \bibliographystyle{jpe}
    \bibliography{../Biblio}
    \end{frame}
    
    \end{document}
    