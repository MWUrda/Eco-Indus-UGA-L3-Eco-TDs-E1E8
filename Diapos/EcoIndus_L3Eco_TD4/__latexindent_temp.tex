%\documentclass[ignorenonframetext, compress, 9pt, xcolor=svgnames]{beamer} 
\input{../Config_diapos}
\usepackage{color}
\usepackage{tikz}
\usetikzlibrary{shapes.geometric, arrows}
\usepackage{enumerate}   
\usepackage{multirow}
%\setbeamersize{text margin left=1.5em,text margin right=1.5em} 
%\setbeamersize{text margin left=1.2cm,text margin right=1.2cm} 
\setbeamersize{text margin left=1.5em,text margin right=1.5em} 
%\usepackage{xr}
%\externaldocument{Econometrie1_UGA_P2e}
  \usepackage{eso-pic}
%\newcommand\AtPagemyUpperLeft[1]{\AtPageLowerLeft{%
%\put(\LenToUnit{0.9\paperwidth},\LenToUnit{0.85\paperheight}){#1}}}
%\AddToShipoutPictureFG{
 % \AtPagemyUpperLeft{{\includegraphics[width=1.1cm,keepaspectratio]{logoUGA2020.pdf}}}
%}%

%\setbeamercolor{title}{fg=black}
%\setbeamercolor{frametitle}{fg=black}
%\setbeamercolor{section in head/foot}{fg=black}
%\setbeamercolor{author in head/foot}{bg=Brown}
%\setbeamercolor{date in head/foot}{fg=Brown}
\setbeamertemplate{section page}
{
    \begin{centering}
    \begin{beamercolorbox}[sep=11pt,center]{part title}
    \usebeamerfont{section title}\thesection.~\insertsection\par
    \end{beamercolorbox}
    \end{centering}
}
%\titlegraphic{\includegraphics[width=1cm]{logoUGA2020.pdf}}
\title[]{ \textbf{Économie Industrielle}\footnote{responsable du cours: Sylvain Rossiaud} \\ (UGA, L3 Éco, S2) \\ }
\subtitle{Travaux dirigés: No 3\\ Ententes Verticales\\
(éléments de correction)}
\date{\today}
\author{Michal W. Urdanivia\inst{*}}
\institute{\inst{*}UGA, Facult\'e d'\'Economie, GAEL, \\
e-mail:
 \href{
     mailto:michal.wong-urdanivia@univ-grenoble-alpes.fr}{michal.wong-urdanivia@univ-grenoble-alpes.fr}}

%\titlegraphic{\includegraphics[width=1cm]{logoUGA2020.pdf}
%}

\begin{document}

%%% TIKZ STUFF
\usetikzlibrary{positioning}
\usetikzlibrary{snakes}
\usetikzlibrary{calc}
\usetikzlibrary{arrows}
\usetikzlibrary{decorations.markings}
\usetikzlibrary{shapes.misc}
\usetikzlibrary{matrix,shapes,arrows,fit,tikzmark}
\usetikzlibrary{shapes}
\usetikzlibrary{shapes.geometric, arrows}
\tikzset{   
        every picture/.style={remember picture,baseline},
        every node/.style={anchor=base,align=center,outer sep=1.5pt},
        every path/.style={thick},
        }
\newcommand\marktopleft[1]{
    \tikz[overlay,remember picture] 
        \node (marker-#1-a) at (-.3em,.3em) {};%
}
\newcommand\markbottomright[2]{%
    \tikz[overlay,remember picture] 
        \node (marker-#1-b) at (0em,0em) {};%
}
\tikzstyle{every picture}+=[remember picture] 
\tikzstyle{mybox} =[draw=black, very thick, rectangle, inner sep=10pt, inner ysep=20pt]
\tikzstyle{fancytitle} =[draw=black,fill=red, text=white]
\tikzstyle{observed}=[draw,circle,fill=gray!50]

\begin{frame}
\titlepage
\end{frame}
\begin{frame}
 \tableofcontents
    \end{frame}
%\begin{frame}
%\frametitle{Contenu}
%\tableofcontents[pausesections, pausesubsections]
%\end{frame}

%\section{Qu'est-ce que l’économétrie ? A quoi (à qui) ça sert ?}
%\frame{\sectionpage}
%\begin{frame}
%  \tableofcontents  
%\end{frame}


\section{Exercice d'application}
\frame{\sectionpage}
\begin{frame}[allowframebreaks]{Rappels}
\begin{itemize}
\item \textbf{\underline{Référence}}: \cite{belleflamme_peitz_2015}, chapitre 17(section 17.1).
\item Cadre: des firmes dans une chaîne de production/offre verticale. 
\item Exemple:
\begin{enumerate}[$\star$]
    \item Une firme produit un bien intermédiaire(c.à.d, l'input) dans la production d'un bien final 
    par une autre firme. 
    \item Une firme produit un bien final qui est distribué sur le marché par une autre firme. 
    \item \ldots
\end{enumerate} 
\item Chaque firme associée à un niveau de la chaîne possède potentiellement un pouvoir de marché 
et la modélisation doit en tenir compte pour pouvoir décrire les marchés considérés et répondre à des questions telles que: 
\begin{enumerate}[$\star$]
\item L'effet de l'intégration verticale?
\item Une firme en amont(e.g., productrice d'un bien) peut-elle empêcher des concurrents 
d'accéder en faisant des accords d'exclusivité avec des firmes en aval(e.g., les distributeurs)?
\end{enumerate} 

\framebreak

\item L'inefficacité du fonctionnement de ces marchés est le problème de la  \textbf{\underline{Double marge}}:
\begin{enumerate}[$\star$]
\item Sur un marché dans lequel les firmes n'opèrent qu'à un seul niveau de la chaîne verticale d'offre 
les prix de distribution sont plus élevés que sur un marché verticalement intégré car une firme 
en aval(c.à.d., le distributeur) applique une marge sur le prix de vente du bien final qui intègre la marge appliquée 
par la firme en amont(c.à.d., le producteur).
\end{enumerate}

\item \textbf{\underline{Prix linéaires et double marge}}:
\begin{enumerate}[$\star$]
\item Marché monopolistique pour un bien final. 
\item Deux firmes à deux niveaux de la chaîne: une firme en amont appelée "producteur", firme en aval appelée "distributeur".
\item Demande sur le marché: 
\begin{align*}
    q(p) &= a-bp, \quad a, b > 0
\end{align*}
\item Les coûts de production(pour la firme en amont) correspondent à un coût marginal constant:
\begin{align*}
    c(q) &= cq \Rightarrow c^{m}(q) := \frac{\partial c}{\partial q}(q) = c, \quad c<\frac{a}{b}.
\end{align*}
\item Les coûts de distribution sont supposés nuls(pour la firme en aval). 
\item Jeux en deux étapes: à la première le producteur fixe le prix de vente $w$, et à la deuxième le distributeur fixe 
le prix à la distribution $p$ ayant observé $w$. 
\item Induction à rebours:

\item \textbf{Dernière étape}: le distributer fixe son prix $p^*$ en maximisant son profit: 
\begin{align*}
    \pi_d(p, w) &= (a-bp)(p-w) \Rightarrow \underbrace{\frac{\partial \pi_d}{\partial p}(p^*) = 0}_{c.p.o.} \Leftrightarrow p^*(w) = \frac{a+bw}{2b}
\end{align*}
\item \textbf{Première étape}: le producteur maximise sont profit sachant que la demande est:
\begin{align*}
    q(p^*) &= a-b\left(\underbrace{\frac{a+bw}{2b}}_{=p^*(w)}\right) = \frac{a-bw}{2},
\end{align*}
et le profit à maximiser est:
\begin{align*}
    \pi_p(w) &= \left(\frac{a-bw}{2}\right)(w - c) \Rightarrow \underbrace{\frac{\partial \pi_p}{\partial w}(w^*) = 0}_{c.p.o.} 
    \Leftrightarrow w^* = \
\end{align*}


\end{enumerate}
\framebreak

\tikzstyle{startstop} = [rectangle, rounded corners, minimum width=3cm, minimum height=1cm,text centered, draw=black,
 fill=red!30]
\tikzstyle{io} = [trapezium, trapezium left angle=70, trapezium right angle=110, minimum width=3cm, minimum height=1cm, 
text centered, 
draw=black, fill=blue!30]
\tikzstyle{process} = [rectangle, minimum width=3cm, minimum height=1cm, text centered, draw=black]
\tikzstyle{decision} = [diamond, minimum width=3cm, minimum height=1cm, text centered, draw=black, fill=green!30]
\tikzstyle{arrow} = [thick,->,>=stealth]
\begin{figure}
\begin{tikzpicture}[node distance=2cm]

    \node (start) [startstop] {Start};
    \node (in1) [io, below of=start] {Input};
    \node (pro1) [process, below of=in1] {Process 1};
    \node (dec1) [decision, below of=pro1, yshift=-0.5cm] {Decision 1};
    \draw [arrow] (start) -- (in1);
\draw [arrow] (in1) -- (pro1);
\draw [arrow] (pro1) -- (dec1);
\end{tikzpicture}
\end{figure}

\end{itemize}
\end{frame}
   
\begin{frame}[allowframebreaks]{References}
\bibliographystyle{jpe}
\bibliography{../Biblio}
\end{frame}

\end{document}
