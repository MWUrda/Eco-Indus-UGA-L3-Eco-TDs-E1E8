%\documentclass[ignorenonframetext, compress, 9pt, xcolor=svgnames]{beamer} 
\documentclass[notes, ignorenonframetext, compress, 10pt, xcolor=svgnames, aspectratio=169]{beamer} 
\usepackage{pgfpages}
\usepackage{pdfpages}
% These slides also contain speaker notes. You can print just the slides,
% just the notes, or both, depending on the setting below. Comment out the want
% you want.
\setbeameroption{hide notes} % Only slide
%\setbeameroption{show only notes} % Only notes
%\setbeameroption{show notes on second screen=right} % Both
\usepackage{amsmath}
\usepackage{amsfonts}
\usepackage{amssymb}
\setbeamercolor{frametitle}{fg=MidnightBlue}

\setbeamercolor{sectionpage title}{bg=MidnightBlue}
\setbeamertemplate{frametitle}[default][center]
%\setbeamertemplate{headline}{\vskip2cm}
%\setbeamertemplate{frametitle}{\color{MidnightBlue}\centering\bfseries\insertframetitle\par\vskip-6pt}
\setbeamerfont{frametitle}{series=\bfseries}
\setbeamerfont{title}{series=\bfseries}
\setbeamerfont{sectionpage}{series=\bfseries}
%\setbeamercolor{section in head/foot}{bg=MidnightBlueBlue}
%\setbeamercolor{author in head/foot}{bg=DarkBlue}
\setbeamercolor{author in head/foot}{fg=MidnightBlue}
%\setbeamercolor{title in head/foot}{bg=White}
\setbeamercolor{title in head/foot}{fg=MidnightBlue}
\setbeamercolor{title}{fg=MidnightBlue}
%\setbeamercolor{date in head/foot}{fg=Brown}
%\setbeamercolor{alerted text}{fg=DarkBlue}
%\usecolortheme[named=DarkBlue]{structure} 
%\usepackage{bbm}
%\usepackage{bbold}
\usepackage{eurosym}
\usepackage{graphicx}
%\usepackage{epstopdf}
\usepackage{hyperref}
\hypersetup{
  colorlinks   = true, %Colours links instead of ugly boxes
  urlcolor     = gray, %Colour for external hyperlinks
  linkcolor    = MidnightBlue, %Colour of internal links
  citecolor   = DarkRed %Colour of citations
}
\usepackage{multirow}
\usepackage{xspace}
\usepackage{listings}
\usepackage{natbib}
%\usepackage[sort&compress,comma,super]{natbib}
\def\newblock{} % To avoid a compilation error about a function \newblock undefined
\usepackage{bibentry}
\usepackage{booktabs}
\usepackage{dcolumn}
\usepackage[greek,frenchb]{babel}
\usepackage[babel=true,kerning=true]{microtype}
\usepackage[utf8]{inputenc}
\usepackage[T1]{fontenc}
\usepackage{natbib}
\renewcommand{\cite}{\citet}
\usepackage{longtable}
\usepackage{eso-pic}

\usepackage{xcolor}
 \colorlet{linkequation}{DarkRed} 
 \newcommand*{\SavedEqref}{}
 \let\SavedEqref\eqref 
\renewcommand*{\eqref}[1]{%
\begingroup \hypersetup{
      linkcolor=linkequation,
linkbordercolor=linkequation, }%
\SavedEqref{#1}%
 \endgroup
}

\newcommand*{\refeq}[1]{%
 \begingroup
\hypersetup{ 
linkcolor=linkequation, 
linkbordercolor=linkequation,
}%
\ref{#1}%
 \endgroup
}

\setbeamertemplate{caption}[numbered]
\setbeamertemplate{theorem}[ams style]
\setbeamertemplate{theorems}[numbered]
%\usefonttheme{serif}
%\usecolortheme{beaver}
%\usetheme{Hannover}
%\usetheme{CambridgeUS}
%\usetheme{Madrid}
%\usecolortheme{whale}
%\usetheme{Warsaw}
%\usetheme{Luebeck}
%\usetheme{Montpellier}
%\usetheme{Berlin}
%\setbeamercolor{titlelike}{parent=structure}
%\setbeamertemplate{headline}[default]
%\setbeamertemplate{footline}[default]
%\setbeamertemplate{footline}[Malmoe]
%\setbeamercovered{transparent}
%\setbeamercovered{invisible}
%\usecolortheme{crane}
%\usecolortheme{dolphin}
%\usepackage{pxfonts}
%\usepackage{isomath}
%\usepackage{mathpazo}
%\usepackage{arev} %     (Arev/Vera Sans)
%\usepackage{eulervm} %_   (Euler Math)
%\usepackage{fixmath} %  (Computer Modern)
%\usepackage{hvmath} %_   (HV-Math/Helvetica)
%\usepackage{tmmath} %_   (TM-Math/Times)
%\usepackage{tgheros}
%\usepackage{cmbright}
%\usepackage{ccfonts} \usepackage[T1]{fontenc}
%\usepackage[garamond]{mathdesign}

%\usepackage{color}
%\usepackage{ulem}

%\usepackage[math]{kurier}
%\usepackage[no-math]{fontspec}
%\setmainfont{Fontin Sans}
%\setsansfont{Fontin Sans}
%\setbeamerfont{frametitle}{size=\LARGE,series=\bfseries}
%%%add 19022021
\usepackage{enumerate}    
\usepackage{dcolumn}
\usepackage{verbatim}
\newcolumntype{d}[0]{D{.}{.}{5}}
%\setbeamertemplate{note page}{\pagecolor{yellow!5}\insertnote}
%\usetikzlibrary{positioning}
%\usetikzlibrary{snakes}
%\usetikzlibrary{calc}
%\usetikzlibrary{arrows}
%\usetikzlibrary{decorations.markings}
%\usetikzlibrary{shapes.misc}
%\usetikzlibrary{matrix,shapes,arrows,fit,tikzmark}
%%%
% suppress navigation bar
\beamertemplatenavigationsymbolsempty
%\usetheme{bunsenMod}
%\setbeamercovered{transparent}
%\setbeamertemplate{items}[circle]
%\usecolortheme[named=CadetBlue]{structure}
%\usecolortheme[RGB={225,64,5}]{structure}
%\definecolor{burntRed}{RGB}{225,64,5}
%\setbeamercolor{alerted text}{fg=burntRed} 
%\usecolortheme[RGB={0,40,110}]{structure}
%\hypersetup{linkcolor=burntRed}
%\hypersetup{urlcolor=burntRed}
%\hypersetup{filecolor=burntRed}
%\hypersetup{citecolor=burntRed}

%\usetheme{bunsenMod}
%\setbeamercovered{transparent}
%\setbeamertemplate{items}[circle]
%\usecolortheme[named=CadetBlue]{structure}
%\usecolortheme[RGB={225,64,5}]{structure}
%\definecolor{burntRed}{RGB}{225,64,5}
%\setbeamercolor{alerted text}{fg=burntRed} 
%\usecolortheme[RGB={0,40,110}]{structure}
%\hypersetup{linkcolor=burntRed}
%\hypersetup{urlcolor=burntRed}
%\hypersetup{filecolor=burntRed}
%\hypersetup{citecolor=burntRed}

%\AtBeginSection[] % Do nothing for \section*
%{ \frame{\sectionpage} }
%\setbeamertemplate{frametitle continuation}{}
\newtheorem{lemme}{Lemme}[section]
%\newtheorem{remarque}{Remarque}
\newcommand{\argmax}{\operatornamewithlimits{arg\,max}}
\newcommand{\argmin}{\operatornamewithlimits{arg\,min}}
\def\inprobLOW{\rightarrow_p}
\def\inprobHIGH{\,{\buildrel p \over \rightarrow}\,} 
\def\inprob{\,{\inprobHIGH}\,} 
\def\indist{\,{\buildrel d \over \rightarrow}\,} 
\def\sima{\,{\buildrel a \over \sim}\,} 
\def\F{\mathbb{F}}
\def\R{\mathbb{R}}
\def\N{\mathbb{N}}
\newcommand{\gmatrix}[1]{\begin{pmatrix} {#1}_{11} & \cdots &
    {#1}_{1n} \\ \vdots & \ddots & \vdots \\ {#1}_{m1} & \cdots &
    {#1}_{mn} \end{pmatrix}}
\newcommand{\iprod}[2]{\left\langle {#1} , {#2} \right\rangle}
\newcommand{\norm}[1]{\left\Vert {#1} \right\Vert}
\newcommand{\abs}[1]{\left\vert {#1} \right\vert}
\renewcommand{\det}{\mathrm{det}}
\newcommand{\rank}{\mathrm{rank}}
\newcommand{\spn}{\mathrm{span}}
\newcommand{\row}{\mathrm{Row}}
\newcommand{\col}{\mathrm{Col}}
\renewcommand{\dim}{\mathrm{dim}}
\newcommand{\prefeq}{\succeq}
\newcommand{\pref}{\succ}
\newcommand{\seq}[1]{\{{#1}_n \}_{n=1}^\infty }
\renewcommand{\to}{{\rightarrow}}
\renewcommand{\L}{{\mathcal{L}}}
\newcommand{\Er}{\mathrm{E}}
\renewcommand{\Pr}{\mathrm{P}}
%\newcommand{\Var}{\mathrm{Var}}
%\newcommand{\Cov}{\mathrm{Cov}}
%\newcommand{\corr}{\mathrm{Corr}}
%\newcommand{\Var}{\mathrm{Var}}
\newcommand{\bias}{\mathrm{Bias}}
\newcommand{\mse}{\mathrm{MSE}}
\providecommand{\Pred}{\mathcal{P}}
\providecommand{\plim}{\operatornamewithlimits{plim}}
\providecommand{\avg}{\frac{1}{n} \underset{i=1}{\overset{n}{\sum}}}
\providecommand{\sumin}{{\sum_{i=1}^n}}
\providecommand{\sumiN}{{\sum_{i=1}^N}}
\providecommand{\sumtT}{{\sum_{t=1}^T}}
\providecommand{\limp}{\overset{p}{\rightarrow}}
\providecommand{\liml}{\overset{L}{\rightarrow}}
%\providecommand{\limp}{\underset{n \rightarrow \infty}{\overset{p}{\longrightarrow}}}
%\providecommand{\limp}{\underset{n \rightarrow \infty}{\overset{p}{\longrightarrow}}}
%\providecommand{\limp}{\overset{p}{\longrightarrow}}
%\providecommand{\limd}{\underset{n \rightarrow \infty}{\overset{d}{\longrightarrow}}}
\providecommand{\limd}{\overset{d}{\rightarrow}}
\providecommand{\limps}{\overset{p.s.}{\rightarrow}}
\providecommand{\limlp}{\overset{L^p}{\rightarrow}}
\providecommand{\limprob}{\overset{p}{\underset{N\to +\infty}{\longrightarrow}}}
\providecommand{\limloi}{\overset{L}{\underset{N\to +\infty}{\longrightarrow}}}
\providecommand{\limpsure}{\overset{p.s.}{\underset{N\to +\infty}{\longrightarrow}}}
\def\independenT#1#2{\mathrel{\setbox0\hbox{$#1#2$}%
    \copy0\kern-\wd0\mkern4mu\box0}} 
\newcommand\indep{\protect\mathpalette{\protect\independenT}{\perp}}


\lstset{language=R}
\lstset{keywordstyle=\color[rgb]{0,0,1},                                        % keywords
        commentstyle=\color[rgb]{0.133,0.545,0.133},    % comments
        stringstyle=\color[rgb]{0.627,0.126,0.941}      % strings
}       
\lstset{
  showstringspaces=false,       % not emphasize spaces in strings 
  columns=fixed,
  numbersep=3mm, numbers=left, numberstyle=\tiny,       % number style
  frame=none,
  framexleftmargin=5mm, xleftmargin=5mm         % tweak margins
}
\makeatletter
%\setbeamertemplate{frametitle continuation}{\gdef\beamer@frametitle{}}
\setbeamertemplate{frametitle continuation}{\frametitle{}}
%\setbeamertemplate{frametitle continuation}{\insertcontinuationcount}
\makeatother

\theoremstyle{remark}
\newtheorem{interpretation}{Interprétation}
\newtheorem*{interpretation*}{Interprétation}

\theoremstyle{remark}
\newtheorem{remarque}{Remarque}%[section]
\newtheorem*{remarque*}{Remarque}
\usepackage[framemethod=TikZ]{mdframed} 
\usepackage{showexpl}
%\newtheorem{step}{Step}[section]
%\newtheorem{rem}{Comment}[section]
%\newtheorem{ex}{Example}[section]
%\newtheorem{hist}{History}[section]
%\newtheorem*{ex*}{Example}
%\theoremstyle{plain}
%\newtheorem{propriete}{Propri\'et\'e}
%\renewcommand{\thepropriete}{P\arabic{propriete}}
%\theoremstyle{definition}
%\newtheorem{definition}{Définition}%[section]
%\theoremstyle{remark}
%\newtheorem{exemple}{Exemple}
%\newtheorem*{exemple*}{Exemple}

\newtheorem{theoreme}{Théorème}
\newtheorem{proposition}{Proposition}
%\newtheorem{propriete}{Propri\'et\'e}
\newtheorem{corollaire}{Corollaire}
%\newtheorem{exemple}{Exemple}
%\newtheorem{assumption}{Assumption}
%\renewcommand{\theassumption}{A\arabic{assumption}}
\newtheorem{hypothese}{Hypothèse}
\renewcommand{\thehypothese}{H\arabic{hypothese}}
%\theoremstyle{definition}

%\newtheorem{definitionx}{D\'efinition}%[section]
%\newenvironment{definition}
 %{\pushQED{\qed}\renewcommand{\qedsymbol}{$\triangle$}\definitionx}
 %{\popQED\enddefinitionx}

%\newtheorem{condition}{Condition}
%\renewcommand{\thecondition}{C\arabic{condition}}
%\newcommand{\Var}{\mathbb{V}}
%\newcommand{\Var}{\mathbf{Var}}
%\newcommand{\Exp}{\mathbf{E}}
%\providecommand{\Vr}{\mathrm{Var}}
%\renewcommand{\Er}{\mathbb{E}}
%\newcommand{\LP}{\mathcal{LP}}
%\providecommand{\Id}{\mathbf{I}}
%\providecommand{\Rang}{\mathrm{Rang}}
%\providecommand{\Trace}{\mathrm{Trace}}
%\newcommand{\Cov}{\mathbf{Cov}}
%\newcommand{\Cov}{\mathbb{C}\mathrm{ov}}
\providecommand{\Id}{\mathbf{I}}
\providecommand{\Ind}{\mathbf{1}}
\providecommand{\uvec}{\mathbf{1}}
\providecommand{\vecOnes}{\mathbf{1}}
\DeclareMathOperator{\indfun}{\mathbf{1}}
\DeclareMathOperator{\Exp}{E}
\DeclareMathOperator{\Expn}{\mathbb{E}_n}
\DeclareMathOperator{\EL}{EL}
\DeclareMathOperator{\Var}{Var}
\DeclareMathOperator{\Vr}{V}
\newcommand{\boldVr}{ {\boldsymbol \Vr} }
\DeclareMathOperator{\Cov}{Cov}
\DeclareMathOperator{\corr}{corr}
\DeclareMathOperator{\perps}{\perp_s}
%\DeclareMathOperator{\Prob}{Pr}
\DeclareMathOperator{\Prob}{P}
\DeclareMathOperator{\prob}{p}
\DeclareMathOperator{\loss}{L}
\providecommand{\Corr}{\mathrm{Corr}}
\providecommand{\Diag}{\mathrm{Diag}}
\providecommand{\reg}{\mathrm{r}}
\providecommand{\Likelihood}{\mathrm{L}}
\renewcommand{\Pr}{{\mathbb{P}}}
\providecommand{\set}[1]{\left\{#1\right\}}
\providecommand{\uvec}{\mathbf{1}}
\providecommand{\Rang}{\mathrm{Rang}}
\providecommand{\Trace}{\mathrm{Trace}}
\providecommand{\Tr}{\mathrm{Tr}}
\providecommand{\CI}{\mathrm{CI}}
\providecommand{\asyvar}{\mathrm{AsyVar}}
\DeclareMathOperator{\Supp}{Supp}
\newcommand{\inputslide}[2]{{
    \usebackgroundtemplate{
     \includegraphics[page={#2},width=0.90\textwidth,keepaspectratio=true]
      %\includegraphics[page={#2},width=\paperwidth,keepaspectratio=true]
      {{#1}}}
    \frame[plain]{}
  }}
\newcommand\pperp{\perp\!\!\!\perp}
\newcommand\independent{\protect\mathpalette{\protect\independenT}{\perp}}
\def\independenT#1#2{\mathrel{\rlap{$#1#2$}\mkern2mu{#1#2}}}
\usepackage{bbm}
\providecommand{\Ind}{\mathbf{1}}
\newcommand{\sumjsi}{\underset{i<j}{{\sum}}}
\newcommand{\prodjsi}{\underset{i<j}{{\prod}}}
\newcommand{\sumisj}{\underset{j<i}{{\sum}}}
\newcommand{\prodisj}{\underset{j<i}{{\prod}}}
\newcommand{\sumobs}{\underset{i=1}{\overset{n}{\sum}}}
\newcommand{\sumi}{\underset{i=1}{\overset{n}{\sum}}}
\newcommand{\prodi}{\underset{i=1}{\overset{n}{\prod}}}
\newcommand{\prodobs}{\underset{i=1}{\overset{n}{\prod}}}
\newcommand{\simiid}{{\overset{i.i.d.}{\sim}}}
%\newcommand{\sumobs}{\sum_{i=1}^N}
%\newcommand{\prodobs}{\prod_{i=1}^N}
%\newcommand{\sumjsi}{\sum_{i<j}}
%\newcommand{\prodjsi}{\prod_{i<j}}
%\newcommand{\sumisj}{\sum_{j<i}}
%\newcommand{\prodisj}{\sum_{j<i}}

%\usepackage{appendixnumberbeamer}
\setbeamertemplate{footline}[frame number]
\setbeamertemplate{section in toc}[sections numbered]
\setbeamertemplate{subsection in toc}[subsections numbered]
\setbeamertemplate{subsubsection in toc}[subsubsections numbered]

%\makeatother
%\setbeamertemplate{footline}
%{
%    \leavevmode%
%    \hbox{%
%        \begin{beamercolorbox}[wd=.333333\paperwidth,ht=2.25ex,dp=1ex,center]{author in head/foot}%
%            \usebeamerfont{author in head/foot}\insertshortauthor
%        \end{beamercolorbox}%
%        \begin{beamercolorbox}[wd=.333333\paperwidth,ht=2.25ex,dp=1ex,center]{title in head/foot}%
%            \usebeamerfont{title in head/foot}\insertshorttitle
%        \end{beamercolorbox}%
%        \begin{beamercolorbox}[wd=.333333\paperwidth,ht=2.25ex,dp=1ex,right]{date in head/foot}%
%            \usebeamerfont{date in head/foot}\insertshortdate{}\hspace*{2em}
%            \insertframenumber{} / \inserttotalframenumber\hspace*{2ex} 
%        \end{beamercolorbox}}%
%       \vskip0pt%
 %   }
%   \makeatother
%\setbeamertemplate{navigation symbols}{}
\setbeamertemplate{itemize items}[ball]
%\setbeamertemplate{itemize items}{-}
%\newenvironment{wideitemize}{\itemize\addtolength{\itemsep}{10pt}}{\enditemize}
% \usepackage{eso-pic}
%\newcommand\AtPagemyUpperLeft[1]{\AtPageLowerLeft{%
%\put(\LenToUnit{0.9\paperwidth},\LenToUnit{0.9\paperheight}){#1}}}
%\AddToShipoutPictureFG{
%  \AtPagemyUpperLeft{{\includegraphics[width=1.1cm,keepaspectratio]{../logo-uga.png}}}
%}%
\def\figheight{3in}
\def\figwidth{4in}

%%Commands from Econometric Theory(Slides) by J. Stachurski.

\newcommand{\boldx}{ {\mathbf x} }
\newcommand{\boldu}{ {\mathbf u} }
\newcommand{\boldv}{ {\mathbf v} }
\newcommand{\boldw}{ {\mathbf w} }
\newcommand{\boldy}{ {\mathbf y} }
\newcommand{\boldb}{ {\mathbf b} }
\newcommand{\bolda}{ {\mathbf a} }
\newcommand{\boldc}{ {\mathbf c} }
\newcommand{\boldd}{ {\mathbf d} }
\newcommand{\boldi}{ {\mathbf i} }
\newcommand{\bolde}{ {\mathbf e} }
\newcommand{\boldp}{ {\mathbf p} }
\newcommand{\boldq}{ {\mathbf q} }
\newcommand{\bolds}{ {\mathbf s} }
\newcommand{\boldt}{ {\mathbf t} }
\newcommand{\boldz}{ {\mathbf z} }
\newcommand{\boldr}{ {\mathbf r} }
\newcommand{\boldm}{ {\mathbf m} }

\newcommand{\boldzero}{ {\mathbf 0} }
\newcommand{\boldone}{ {\mathbf 1} }

\newcommand{\boldalpha}{ {\boldsymbol \alpha} }
\newcommand{\boldbeta}{ {\boldsymbol \beta} }
\newcommand{\boldgamma}{ {\boldsymbol \gamma} }
\newcommand{\boldGamma}{ {\boldsymbol \Gamma} }
\newcommand{\boldtheta}{ {\boldsymbol \theta} }
\newcommand{\boldxi}{ {\boldsymbol \xi} }
\newcommand{\boldtau}{ {\boldsymbol \tau} }
\newcommand{\boldepsilon}{ {\boldsymbol \epsilon} }
\newcommand{\boldvepsilon}{ {\boldsymbol \varepsilon} }
\newcommand{\boldmu}{ {\boldsymbol \mu} }
\newcommand{\boldSigma}{ {\boldsymbol \Sigma} }
\newcommand{\boldOmega}{ {\boldsymbol \Omega} }
\newcommand{\boldPhi}{ {\boldsymbol \Phi} }
\newcommand{\boldLambda}{ {\boldsymbol \Lambda} }
\newcommand{\boldphi}{ {\boldsymbol \phi} }
\newcommand{\boldeta}{ {\boldsymbol \eta} }

\newcommand{\Sigmax}{ {\boldsymbol \Sigma_{\boldx}}}
\newcommand{\Sigmau}{ {\boldsymbol \Sigma_{\boldu}}}
\newcommand{\Sigmaxinv}{ {\boldsymbol \Sigma_{\boldx}^{-1}}}
\newcommand{\Sigmav}{ {\boldsymbol \Sigma_{\boldv \boldv}}}

\newcommand{\hboldx}{ \hat {\mathbf x} }
\newcommand{\hboldy}{ \hat {\mathbf y} }
\newcommand{\hboldb}{ \hat {\mathbf b} }
\newcommand{\hboldu}{ \hat {\mathbf u} }
\newcommand{\hboldtheta}{ \hat {\boldsymbol \theta} }
\newcommand{\hboldtau}{ \hat {\boldsymbol \tau} }
\newcommand{\hboldmu}{ \hat {\boldsymbol \mu} }
\newcommand{\hboldbeta}{ \hat {\boldsymbol \beta} }
\newcommand{\hboldgamma}{ \hat {\boldsymbol \gamma} }
\newcommand{\hboldSigma}{ \hat {\boldsymbol \Sigma} }

\newcommand{\boldA}{\mathbf A}
\newcommand{\boldB}{\mathbf B}
\newcommand{\boldC}{\mathbf C}
\newcommand{\boldD}{\mathbf D}
\newcommand{\boldI}{\mathbf I}
\newcommand{\boldL}{\mathbf L}
\newcommand{\boldM}{\mathbf M}
\newcommand{\boldP}{\mathbf P}
\newcommand{\boldQ}{\mathbf Q}
\newcommand{\boldR}{\mathbf R}
\newcommand{\boldX}{\mathbf X}
\newcommand{\boldU}{\mathbf U}
\newcommand{\boldV}{\mathbf V}
\newcommand{\boldW}{\mathbf W}
\newcommand{\boldY}{\mathbf Y}
\newcommand{\boldZ}{\mathbf Z}

\newcommand{\bSigmaX}{ {\boldsymbol \Sigma_{\hboldbeta}} }
\newcommand{\hbSigmaX}{ \mathbf{\hat \Sigma_{\hboldbeta}} }
\newcommand{\betahat}{\hat{\beta}}
\newcommand{\gammahat}{\hat{\gamma}}
\newcommand{\Uhat}{\hat{U}}
\newcommand{\Vhat}{\hat{V}}
\newcommand{\epsilonhat}{\hat{\epsilon}}
\newcommand{\sigmahat}{\hat{\sigma}}
\newcommand{\Sigmahat}{\hat{\Sigma}}
\newcommand{\Gammahat}{\hat{\Gamma}}

\newcommand{\RR}{\mathbbm R}
\newcommand{\CC}{\mathbbm C}
\newcommand{\NN}{\mathbbm N}
\newcommand{\PP}{\mathbbm P}
\newcommand{\EE}{\mathbbm E \nobreak\hspace{.1em}}
\newcommand{\EEP}{\mathbbm E_P \nobreak\hspace{.1em}}
\newcommand{\ZZ}{\mathbbm Z}
\newcommand{\QQ}{\mathbbm Q}


\newcommand{\XX}{\mathbbm X}

\newcommand{\aA}{\mathcal A}
\newcommand{\fF}{\mathscr F}
\newcommand{\bB}{\mathscr B}
\newcommand{\iI}{\mathscr I}
\newcommand{\rR}{\mathscr R}
\newcommand{\dD}{\mathcal D}
\newcommand{\lL}{\mathcal L}
\newcommand{\llL}{\mathcal{H}_{\ell}}
\newcommand{\gG}{\mathcal G}
\newcommand{\hH}{\mathcal H}
\newcommand{\nN}{\textrm{\sc n}}
\newcommand{\lN}{\textrm{\sc ln}}
\newcommand{\pP}{\mathscr P}
\newcommand{\qQ}{\mathscr Q}
\newcommand{\xX}{\mathcal X}
\newcommand{\yY}{\mathcal Y}
\newcommand{\ddD}{\mathscr D}


%\newcommand{\R}{{\texttt R}}
\newcommand{\risk}{\mathcal R}
\newcommand{\Remp}{R_{{\rm emp}}}

\newcommand*\diff{\mathop{}\!\mathrm{d}}
\newcommand{\ess}{ \textrm{{\sc ess}} }
\newcommand{\tss}{ \textrm{{\sc tss}} }
\newcommand{\rss}{ \textrm{{\sc rss}} }
\newcommand{\rssr}{ \textrm{{\sc rssr}} }
\newcommand{\ussr}{ \textrm{{\sc ussr}} }
\newcommand{\zdata}{\mathbf{z}_{\mathcal D}}
\newcommand{\Pdata}{P_{\mathcal D}}
\newcommand{\Pdatatheta}{P^{\mathcal D}_{\theta}}
\newcommand{\Zdata}{Z_{\mathcal D}}


\newcommand{\e}[1]{\mathbbm{E}[{#1}]}
\newcommand{\p}[1]{\mathbbm{P}({#1})}

% condition
\theoremstyle{definition}
\newtheorem{condition}{Condition}
\renewcommand{\thecondition}{C\arabic{condition}}
\BeforeBeginEnvironment{condition}{
  \setbeamerfont{block title}{series=\bfseries}
  \setbeamercolor{block title}{fg=MidnightBlue,bg=white}
  \setbeamercolor{block body}{fg=black, bg=gray!10}
}
\newtheorem*{condition*}{Condition}
\BeforeBeginEnvironment{condition*}{
  \setbeamerfont{block title}{series=\bfseries}
  \setbeamercolor{block title}{fg=MidnightBlue,bg=white}
  \setbeamercolor{block body}{fg=black, bg=gray!10}
}

% assumption
\theoremstyle{definition}
\newtheorem{assumption}{Assumption}
\BeforeBeginEnvironment{assumption}{
  \setbeamerfont{block title}{series=\bfseries}
  \setbeamercolor{block title}{fg=MidnightBlue,bg=white}
  \setbeamercolor{block body}{fg=black, bg=gray!10}
}
\newtheorem*{assumption*}{Assumption}
\BeforeBeginEnvironment{assumption*}{
  \setbeamerfont{block title}{series=\bfseries}
  \setbeamercolor{block title}{fg=MidnightBlue,bg=white}
  \setbeamercolor{block body}{fg=black, bg=gray!10}
}

% definition
\BeforeBeginEnvironment{definition}{
  \setbeamerfont{block title}{series=\bfseries}
  \setbeamercolor{block title}{fg=MidnightBlue,bg=white}
  \setbeamercolor{block body}{fg=black, bg=gray!10}
}
\newtheorem*{definition*}{Definition}
\BeforeBeginEnvironment{definition*}{
  \setbeamerfont{block title}{series=\bfseries}
  \setbeamercolor{block title}{fg=MidnightBlue,bg=white}
  \setbeamercolor{block body}{fg=black, bg=gray!10}
}

% theorem
\theoremstyle{plain}
\BeforeBeginEnvironment{theorem}{
  \setbeamerfont{block body}{shape=\itshape}
  \setbeamerfont{block title}{series=\bfseries}
  \setbeamercolor{block title}{fg=MidnightBlue,bg=white}
  \setbeamercolor{block body}{fg=black, bg=gray!10}
}
\newtheorem*{theorem*}{Theorem}
\BeforeBeginEnvironment{theorem*}{
  \setbeamerfont{block body }{shape=\itshape}
  \setbeamerfont{block title}{series=\bfseries}
  \setbeamercolor{block title}{fg=MidnightBlue,bg=white}
  \setbeamercolor{block body}{fg=black, bg=gray!10}
}

% definition_fr
\theoremstyle{definition}
\newtheorem{definition_fr}{Définition}%[section]
\BeforeBeginEnvironment{definition_fr}{
  \setbeamerfont{block title}{series=\bfseries}
  \setbeamercolor{block title}{fg=MidnightBlue,bg=white}
  \setbeamercolor{block body}{fg=black, bg=gray!10}
}
\newtheorem*{definition_fr*}{Définition}
\BeforeBeginEnvironment{definition_fr*}{
  \setbeamerfont{block title}{series=\bfseries}
  \setbeamercolor{block title}{fg=MidnightBlue,bg=white}
  \setbeamercolor{block body}{fg=black, bg=gray!10}
}
% theorem_fr
\newtheorem{theorem_fr}{Théorème}%[section]
\BeforeBeginEnvironment{theorem_fr}{
  \setbeamerfont{block body}{shape=\itshape}
  \setbeamerfont{block title}{series=\bfseries, shape = \upshape}
  \setbeamercolor{block title}{fg=MidnightBlue,bg=white}
  \setbeamercolor{block body}{fg=black, bg=gray!10}
}
\newtheorem*{theorem_fr*}{Théorème}
\BeforeBeginEnvironment{theorem_fr*}{
  \setbeamerfont{block body}{shape=\itshape}
  \setbeamerfont{block title}{series=\bfseries, shape = \upshape}
  \setbeamercolor{block title}{fg=MidnightBlue,bg=white}
  \setbeamercolor{block body}{fg=black, bg=gray!10}
}

% remark_fr
\theoremstyle{remark}
\newtheorem{remark_fr}{Remarque}%[section]
\BeforeBeginEnvironment{remark_fr}{
  \setbeamerfont{block title}{series=\bfseries, shape=\itshape}
  \setbeamercolor{block title}{fg=MidnightBlue,bg=white}
  \setbeamercolor{block body}{fg=black, bg=gray!10}
}
\newtheorem*{remark_fr*}{Remarque}
\BeforeBeginEnvironment{remark_fr*}{
  \setbeamerfont{block title}{series=\bfseries, shape=\itshape}
  \setbeamercolor{block title}{fg=MidnightBlue,bg=white}
  \setbeamercolor{block body}{fg=black, bg=gray!10}
}

% exemple
\theoremstyle{remark}
\newtheorem{exemple}{Exemple}%[section]
\BeforeBeginEnvironment{exemple}{
  \setbeamerfont{block title}{series=\bfseries, shape=\itshape}
  \setbeamercolor{block title}{fg=MidnightBlue,bg=white}
  \setbeamercolor{block body}{fg=black, bg=gray!10}
}
\newtheorem*{exemple*}{}
\BeforeBeginEnvironment{exemple*}{
  \setbeamerfont{block title}{series=\bfseries, shape=\itshape}
  \setbeamercolor{block title}{fg=MidnightBlue,bg=white}
  \setbeamercolor{block body}{fg=black, bg=gray!10}
}


% propriete
\theoremstyle{plain}
\newtheorem{propriete}{Propri\'et\'e}%[section]
\BeforeBeginEnvironment{propriete}{
  \setbeamerfont{block body}{shape=\itshape}
  \setbeamerfont{block title}{series=\bfseries, shape = \upshape}
  \setbeamercolor{block title}{fg=MidnightBlue,bg=white}
  \setbeamercolor{block body}{fg=black, bg=gray!10}
}
\newtheorem*{propriete*}{Propri\'et\'e}
\BeforeBeginEnvironment{propriete*}{
  \setbeamerfont{block body}{shape=\itshape}
  \setbeamerfont{block title}{series=\bfseries, shape = \upshape}
  \setbeamercolor{block title}{fg=MidnightBlue,bg=white}
  \setbeamercolor{block body}{fg=black, bg=gray!10}
}


% remark
\theoremstyle{remark}
\newtheorem{remark}{Remark}%[section]
\BeforeBeginEnvironment{remark}{
  \setbeamerfont{block body}{shape=\itshape}
  \setbeamerfont{block title}{series=\bfseries}
  \setbeamercolor{block title}{fg=MidnightBlue,bg=white}
  \setbeamercolor{block body}{fg=black, bg=gray!10}
}
\newtheorem*{remark*}{Remark}
\BeforeBeginEnvironment{remark*}{
  \setbeamerfont{block body }{shape=\itshape}
  \setbeamerfont{block title}{series=\bfseries}
  \setbeamercolor{block title}{fg=MidnightBlue,bg=white}
  \setbeamercolor{block body}{fg=black, bg=gray!10}
}


\usepackage{color}
\usepackage{tikz}
\usetikzlibrary{positioning}
\usepackage{enumerate}   
\usepackage{multirow}
%\setbeamersize{text margin left=1.5em,text margin right=1.5em} 
%\setbeamersize{text margin left=1.2cm,text margin right=1.2cm} 
\setbeamersize{text margin left=1.5em,text margin right=1.5em} 
%\usepackage{xr}
%\externaldocument{Econometrie1_UGA_P2e}
  \usepackage{eso-pic}
%\newcommand\AtPagemyUpperLeft[1]{\AtPageLowerLeft{%
%\put(\LenToUnit{0.9\paperwidth},\LenToUnit{0.85\paperheight}){#1}}}
%\AddToShipoutPictureFG{
 % \AtPagemyUpperLeft{{\includegraphics[width=1.1cm,keepaspectratio]{logoUGA2020.pdf}}}
%}%

%\setbeamercolor{title}{fg=black}
%\setbeamercolor{frametitle}{fg=black}
%\setbeamercolor{section in head/foot}{fg=black}
%\setbeamercolor{author in head/foot}{bg=Brown}
%\setbeamercolor{date in head/foot}{fg=Brown}
\setbeamertemplate{section page}
{
    \begin{centering}
    \begin{beamercolorbox}[sep=11pt,center]{part title}
    \usebeamerfont{section title}\thesection.~\insertsection\par
    \end{beamercolorbox}
    \end{centering}
}
%\titlegraphic{\includegraphics[width=1cm]{logoUGA2020.pdf}}
\title[]{ \textbf{Économie Industrielle} \\ (UGA, L3 Éco, S2) \\ (responsable du cours: Sylvain Rossiaud)}
\subtitle{Travaux dirigés: No 3\\ 
Les Cartels\\(éléments de correction d'exercices)}
\date{\today}
\author{Michal W. Urdanivia\inst{*}}
\institute{\inst{*}UGA, Facult\'e d'\'Economie, GAEL, \\
e-mail:
 \href{
     mailto:michal.wong-urdanivia@univ-grenoble-alpes.fr}{michal.wong-urdanivia@univ-grenoble-alpes.fr}}

%\titlegraphic{\includegraphics[width=1cm]{logoUGA2020.pdf}
%}

\begin{document}

%%% TIKZ STUFF
\usetikzlibrary{positioning}
\usetikzlibrary{snakes}
\usetikzlibrary{calc}
\usetikzlibrary{arrows}
\usetikzlibrary{decorations.markings}
\usetikzlibrary{shapes.misc}
\usetikzlibrary{matrix,shapes,arrows,fit,tikzmark}
\usetikzlibrary{shapes}
\tikzset{   
        every picture/.style={remember picture,baseline},
        every node/.style={anchor=base,align=center,outer sep=1.5pt},
        every path/.style={thick},
        }
\newcommand\marktopleft[1]{
    \tikz[overlay,remember picture] 
        \node (marker-#1-a) at (-.3em,.3em) {};%
}
\newcommand\markbottomright[2]{%
    \tikz[overlay,remember picture] 
        \node (marker-#1-b) at (0em,0em) {};%
}
\tikzstyle{every picture}+=[remember picture] 
\tikzstyle{mybox} =[draw=black, very thick, rectangle, inner sep=10pt, inner ysep=20pt]
\tikzstyle{fancytitle} =[draw=black,fill=red, text=white]
\tikzstyle{observed}=[draw,circle,fill=gray!50]



\begin{frame}
\titlepage
\end{frame}
\begin{frame}
 \tableofcontents
    \end{frame}
%\begin{frame}
%\frametitle{Contenu}
%\tableofcontents[pausesections, pausesubsections]
%\end{frame}

%\section{Qu'est-ce que l’économétrie ? A quoi (à qui) ça sert ?}
%\frame{\sectionpage}
%\begin{frame}
%  \tableofcontents  
%\end{frame}


\section{Équilibres de concurrence à la Cournot et d'entente}
\frame{\sectionpage}
\begin{frame}[allowframebreaks]{(1a) Concurrence à la Cournot}
\begin{itemize}
    \item Il s'agit d'un modèle de concurrence à la Cournot. 
        \item Chaque firme a une fonction objectif qui est son profit: 
        \begin{align*}
            \pi_i(q_i, q_j) &= P(Q)q_i -c_i(q_i),  \ i, j= 1, 2, \ i\neq j.
        \end{align*}
        où $P(Q)$ est la fonction de demande inverse avec $Q = q_1 + q_2$ est $c_i(q_i)$ 
        est la fonction de coût de la firme $i$.
        \item La variable de décision est la quantité à produire $q_i$. 
        \item L'équilibre est une paire $(q_1^*, q_2^*)$ qui est un équilibre de Nash dans un jeu d'information complète.
        \item Avec : 
        \begin{align*}
            c_i(q_i)&= 40 q_i \Rightarrow c^m_i(q_i) :=\frac{\partial c}{\partial q_i}(q_i) = 40,
        \end{align*}
        et,
        \begin{align*}
           P(Q) &= 400 - 4Q, 
        \end{align*}
        $(q_1^*, q_2^*)$ est obtenu comme solution du système: 
        \begin{align*}
            \begin{array}{l}
            q_1^* = q_1^{mr}(q_2^*) = 45 - \frac{q_2^*}{2}\\
            q_2^* = q_2^{mr}(q_1^*) = 45 - \frac{q_1^*}{2}
            \end{array}
            &\Rightarrow q_1^* = q_2^* = 30.
        \end{align*}
    où rappelons que $q_i^{mr}(q_j)$ est la fonction de réponse de la firme $i$ définie implicitement par la c.p.o 
    dans la maximisation de $\pi(q_i, q_j)$ par rapport à $q_i$(c.f., cours, TDs précédents): 
    \begin{align*}
        \frac{\partial \pi}{\partial q_i}\left(q_i^{mr}(q_j), q_j\right)&=0.
    \end{align*}
    \item On calcule aussi que $Q^* = q_1^* + q_2^* = 60$, $P^*=P(Q^*) = 160$,  $\pi_i(q_i^*, q_j^*) = 3600$.
\end{itemize}
\end{frame}
\begin{frame}[allowframebreaks]{(1b) Profit d'entente}
    \begin{itemize}
        \item Les deux firmes fixent la quantité de monopole.
        \item Pour un niveau décidé de produit $q^{ca}$ chacune produit $q_i^{ca} = \frac{q^{ca}}{2}$, $i=1, 2$.
        \item Le profit est donné par,
        \begin{align*}
            \pi(q^{ca}) &= p(q^{ca})q^{ca} - c(q^{ca}) = (400-4q^{ca})q^{ca} - 40q^{ca}
        \end{align*}
        \item La quantité optimale/de monopole $q^{ca^*}$ est obtenue comme:
        \begin{align*}
            q^{ca^*} &=\argmin_{q^{ca}}  \pi(q) \Rightarrow \frac{\partial \pi}{\partial q}( q^{ca^*}) = 0 
            \Leftrightarrow 360 -8  q^{ca^*} = 0  \Rightarrow q^{ca^*} = 45,
        \end{align*}
        et ainsi $q_i^{ca^*} =  \frac{q^{ca^*}}{2} = 22.5$,  $P^{ca^*} = P(q^{ca^*}) = 220$, $\pi_i(q_i^{ca^*}) = 4050$.
    \end{itemize}
\end{frame}    

\section{Stratégie de déviation d'une firme(c.à.d., de tricherie)}
\frame{\sectionpage}

\begin{frame}[allowframebreaks]{La firme 1 triche, la firme 2 respecte l'entente}
\begin{itemize}
    \item 2 produit donc $q_2^{ca^*} = 22.5$(c.f., question précédente).  
    \item La meilleure réponse de 1 est $q^{mr}_1(q_2^{ca^*}) = 45 - \frac{q_2^{ca^*}}{2} = 33.75=: q_1^{d^*}$.
    \item On a alors $Q^{d} = q_1^{d^*} + q_2^{ca^*}  = 56.25$, $P^d = P(Q^{d}) = 400 - 4 Q^{d} = 175$, et 
    $\pi_1(q_1^{d^*}, q_2^{ca^*} ) = 4556.25$, $\pi_2(q_1^{d^*}, q_2^{ca^*} ) = 3037.5$.
\end{itemize}
\end{frame}

\section{Stabilité d'ententes: discussion}
\frame {\sectionpage}
\begin{frame}[allowframebreaks]{Rappels de cours}
\begin{itemize}
    \item Concurrence entre deux agents/firmes.
    \item La technologie de la firme $i = 1, 2$ est représentée par la fonction de coût:
    \begin{align*}
        c_i(q_i) & c q_i \Rightarrow  c^m_i(q_i) :=\frac{\partial c}{\partial q_i}(q_i) = c,  \ i = 1, 2, \text{pour une constante $c$ strictement positive}.\  
    \end{align*}
    \item Notons la demande inverse,
    \begin{align*}
        P(Q) &= a - b Q, \ Q = q_1+q_2, \ \text{pour de constantes $a$ et $b$ strictement positives}.
    \end{align*}
    \item La fonction objectif/de profit de la firme $i$ est:
    \begin{align*}
        \pi_i(q_i, q_j) &=P(Q)q_i - c q_i, \ i=1, 2.
    \end{align*}
    \framebreak 
    \item En \textbf{\underline{concurrence à la Cournot(c.à.d., par les quantités)}} :
    \begin{enumerate}[-]
        \item Chaque firme maximise son profit par rapport à la quantité qu'elle produit et son choix optimal étant donnée 
        la quantité produite par son concurrent est donné par sa meilleure réponse:
        \begin{align*}
            q_i^{mr}(q_j) &=\frac{a-c}{2b} - \frac{q_j}{2}, \ i, j = 1, 2; \ i\neq j.
        \end{align*}
        \item A l'équilibre $i$ produit $q_i^*$ qui correspond à un équilibre  de Nash tel que:
        \begin{align*}
            q_i^* &=q_i^{mr}(q_j^*), \  i, j = 1, 2; \ i\neq j.
        \end{align*}
        \item On obtient ici $q_i^* = \frac{a-c}{3b}$, $i = 1, 2$.
        \item Prix d'équilibre: $P^*(Q^*) = a - b Q^* = \frac{1}{3}a - \frac{2}{3}c$, avec $Q^* = 2 \frac{a-c}{3b}$.
        \item Profits: $pi^* = \pi_i(q_i^*, q_j^*) = \frac{(a-c)^2}{9b}$.
    \end{enumerate}
    \framebreak
    \item En \textbf{\underline{entente}} les firmes ayant des coûts identiques, elles fixent la quantité $q^{ca}$ qui maximise le profit de monopole:
    \begin{align*}
        \pi(q^{ca}) &= P(q^{ca})q^{ca} - c(q^{ca}) = (a-b q^{ca})q^{ca} - cq^{ca}, \ \text{et $q_i^{ca} = \frac{q^{ca}}{2}$}.
    \end{align*}
    \item A l'équilibre on a alors:
    \begin{enumerate}[-]
        \item Quantités d'équilibre: $q^{ca^*} = \frac{a-c}{2b}$, $q_i^{ca^*} = \frac{q^{ca^*}}{2}$.
        \item Prix d'équilibre: $P^{ca^*} = P(q^{ca^*}) = \frac{a-c}{2}$.
        \item Profits: $\pi_i^{ca^*} = \pi_i(q_i^{ca^*}, q_j^{ca^*}) = P^{ca^*} q_i^{ca^*} - c q_i^{ca^*} = \frac{(a-c)^2}{8b}$.
    \end{enumerate}

    \framebreak
    \item Dans le cas où un des agents, \textbf{\underline{dévie de l'entente(c.à.d., triche)}}, il obtient le profit:
    \begin{align*}
     \pi_i^d &= \pi_i\left(q_i^{mr}(q_j^{ca^*}), q_j^{ca^*}\right) = \frac{9(a-c)^2}{64b}, \ i, j=1, 2, \ i\neq j.
    \end{align*}
    Autrement dit, $i$ joue sa meilleure réponse par rapport à la quantité produite par $j$ en cartel. Ce dernier obtient, 
    \begin{align*}
        \pi_j^d&= \pi_j\left(q_i^{mr}(q_j^{ca^*}), q_j^{ca^*}\right) =\frac{3(a-c)^2}{32b}.
    \end{align*}

\framebreak

\item En fait le jeu consistant à tricher ou s'entendre s'apparente au
 \textbf{\underline{jeu du prisonnier}} avec la représentation de forme normale et extensives suivantes:



		\begin{figure}[!h]
\begin{tabular}{*4c} 
	& &\multicolumn{2}{c}{Firme 2}\\
	& & Tricher &Cartel \\ \cline{3-4}
	\multirow{2}{*}{Firme 1}& Tricher & \multicolumn{1}{|c|}{$\underbrace{\frac{(a-c)^2}{9b}, \frac{(a-c)^2}{9b}}_{\text{Équilibre de Nash}}$} & \multicolumn{1}{c|}{$\frac{9(a-c)^2}{64b},\frac{3(a-c)^2}{32b}$} \\[10pt] \cline{3-4} 
	& Cartel & \multicolumn{1}{|c|}{$\frac{3(a-c)^2}{32b}, \frac{9(a-c)^2}{64b}$} &\multicolumn{1}{c|}{$\frac{(a-c)^2}{8b}, \frac{(a-c)^2}{8b}$}\\[10pt] \cline{3-4}
	\end{tabular}
    \caption{Représentation de la concurrence à la Cournot sous forme normale.}
\end{figure}

\begin{figure}
    \begin{center}
   \small
   \begin{tikzpicture}[thin,
     level 1/.style={sibling distance=60mm},
     level 2/.style={sibling distance=30mm},
     level 3/.style={sibling distance=15mm},
     every circle node/.style={minimum size=1.5mm,inner sep=0mm}]
     
     \node[circle,draw,label=above: Firme 1] (root) {}
       child { node [circle,fill,label=above: Firme 2] {}
         child { 
           node {$\left(\frac{(a-c)^2}{8b}, \frac{(a-c)^2}{8b}\right)$}
           edge from parent
             node[left] {Cartel}}
         child { 
           node {$\left(\frac{3(a-c)^2}{32b}, \frac{9(a-c)^2}{64b}\right)$}
           edge from parent
             node[right] {Triche}}
         edge from parent
           node[left] {Cartel}}
       child { node [circle,fill,label=above: Firme 2] {}
       child { 
        node {$\left(\frac{9(a-c)^2}{64b},\frac{3(a-c)^2}{32b}\right)$}
        edge from parent
          node[left] {Cartel}}
          child { 
            node {$\left(\frac{(a-c)^2}{9b}, \frac{(a-c)^2}{9b}\right)$}
            edge from parent
              node[right] {Triche}}
          edge from parent
            node[right] {Triche}};
   \end{tikzpicture}
   \end{center}
   \caption{Représentation de la concurrence à la Cournot sous forme extensive}
 \end{figure}
    \end{itemize}
\end{frame}

\begin{frame}[allowframebreaks]{Répétition du jeu}
    \begin{itemize}
        \item Supposons que le jeu se répète deux fois: la firme $i$ choisit 
        la quantité $q_{it}$ à la période $t$, pour $i=1, 2$, et $t=1, 2$.
        \item \textbf{\underline{Quel est l'équilibre en sous-jeu parfait?}} 
        \item On résout le jeu(par induction) à rebours:
        \begin{enumerate}[-]
            \item En $t=2$ l'unique équilibre de Nash est $(triche, triche)$.
            \item En $t=1$ l'unique équilibre de Nash est $(triche, triche)$.
            \item Par conséquent, l'unique équilibre en sous-jeux parfait est $\left( 
                (triche, triche), (triche, triche)
            \right)$.
        \end{enumerate}
        \item \textbf{\underline{Autres questions}} : 
        \begin{enumerate}[-]
         \item qu'en est-il de $\left((cartel,cartel), (cartel,cartel)\right)$? 
         \item qu'en est-il de $\left((cartel, triche), (cartel, triche)\right)$?
         \item qu'en est-il de: la firm 1 joue $(cartel; triche \ \text{si} \ triche, cartel  \ \text{si} \ cartel)$,
          et la firme 2 joue $(cartel; triche \ \text{si} \ triche, cartel  \ \text{si} \ cartel)$?
        \item du jeu à 3,..., $N$ périodes?
\end{enumerate}
\end{itemize}
\end{frame}

\section{Jeu/concurrence à la Cournot infiniment répété}
\frame{\sectionpage}

\begin{frame}[allowframebreaks]{Actualisation}
\begin{itemize}
    \item Le jeu se répète un nombre infini de fois et on se demande 
    s'il existe des équilibre en sous-jeu parfait où les firmes jouent $cartel$
     toutes les deux à chaque fois?
     \item Pour y répondre on a besoin du concept d'actualisation:
     \begin{enumerate}[-]
         \item Le taux d'actualisation $\delta \in [0, 1]$, mesure "l'impatience" de la firme.
         \item Par exemple, la valeur actualisée de  $10$ euros reçus aujourd'hui et demain 
         est $10 + \delta 10$.
         \item Si $\delta = 1$, il n'y a pas de différence entre recevoir $10$ euros aujourd'hui et les 
         recevoir demain.
         \item On peut poursuivre le raisonnement avec  10 euros reçus aujourd'hui, demain, et après-demain, soit
         $10 + \delta 10 + \delta^2 10$, raisonnement que l'on peut encore poursuivre \ldots à l'infini.
         \item En fait, il s'agit d'une \textbf{série géométrique}. Elle a notamment la propriété:
         \begin{align*}
             \sum_{k = 0}^\infty \delta^k x &= \frac{x}{1-\delta}.
         \end{align*}
     \end{enumerate}
\end{itemize}
\end{frame}

\begin{frame}[allowframebreaks]{Cournot infiniment répété}
\begin{itemize}
    \item On note comme précédemment $q^{ca}_i$ la quantité de cartel pour la firme $i$(maximisant le profit des deux firmes), et $q_i^*$ 
    la quantité en concurrence à la Cournot.
    \item Nous avons le résultat suivant:  
    \begin{proposition}
        Si le taux d'actualisation $\delta$ est "suffisamment"  élevé 
         alors les stratégies suivantes constituent des équilibre de sous-jeu parfaits pour  
         le jeu de Cournot infiniment répété:  
         \begin{enumerate}[(a)]
             \item En $t$, la firme $i$ joue  $q_{it} =q^{ca}_i$ si $q_{j,t-1} =q^{ca}_j$ pour $j=1$ et $j=2$. 
             \item Jouer $q^*_i$ si $q_{j,t-1} \neq q^{ca}_j$ pour soit $j=1$ ou $j=2$.
         \end{enumerate}
    \end{proposition}
    \item La firme $i$ coopère tant que $j$ coopère.
    \item Une fois que $j$ triche $i$ produit la quantité d'équilibre de Nash-Cournot 
    pour toutes les périodes suivantes: \textbf{Nash reversion}.
    \item “Grim strategy”: pas de deuxième chance.
    \item Pour démontrer que ces strategies constituent des équilibres en sous-jeu parfaits il   
    faut obtenir des conditions qui "prescrivent" que la meilleure réponse 
    de la firme $i$ étant donné celle de la firme $j$ est aussi la meilleure réponse 
    dans chaque sous-jeu.
\end{itemize}

\end{frame}

\begin{frame}[allowframebreaks]{Éléments de démonstration}
\begin{itemize}
    \item Pour la firme $i=1, 2$, deux types de sous-jeu sont à considérer:
    \item \textbf{\underline{Sous-jeu de type 1}}:
    Après une période où un des joueurs à triché (dont soit $i$, ou $j=1, 2$, $i\neq j$):
    \begin{enumerate}[-]
        \item La stratégie proposée indique de jouer $q_i^*$ pour toujours étant donné que $j$ joue aussi 
        cette stratégie.  
        \item C'est un équilibre de Nash du sous-jeu: jouer "$q_i^*$ pour toujours" est la meilleure 
        réponse à la stratégie de $j$ de jouer "$q_j^*$ pour toujours".
        \item Ceci vérifie les critères d'un équilibre de sous-jeu parfait.
    \end{enumerate}
    \framebreak
    \item \textbf{\underline{Sous-jeu de type 2}}: Après une période sans triche.
    \begin{enumerate}[-]
        \item La stratégie proposée indique de coopérer et jouer $q_i^{ca}$, avec le profit actualisé de $\frac{\pi_i^{ca}}{1-\delta}$
        \item La meilleure stratégie alternative est de jouer $q_i^{mr}(q_j^{ca}) =: q_i^d$ à la période en cours,
         mais ceci entraîne $q_j =  q_j^*$ pour toujours. Le profit actualisé est ici 
         $\pi^d_i + \delta\left(\frac{\pi_i^*}{(1-\delta)}\right)$.
         \item Pour que $q_i^{ca}$ soit un équilibre de Nash de ce sous-jeu, il est nécessaire que,
         \begin{align*}
            \underbrace{\frac{\pi_i^{ca}}{1-\delta}}_{\text{profits sous coopération}} &> \underbrace{\pi^d_i + \delta\left(\frac{\pi_i^*}{(1-\delta)}\right)}_{\text{profits sous déviation/triche}}\\
            &\Rightarrow \delta >\frac{9}{17}.
         \end{align*}
    \end{enumerate}
    \item Par conséquent, la \textbf{Nash reversion} indique une meilleure réponse dans ces deux sous-jeux 
    si est "suffisamment élevé" avec $\delta >\frac{9}{17}$. 
    \item Dans ce cas la \textbf{Nash reversion} constitue un équilibre de sous-jeu parfait.
\end{itemize}
\end{frame}
\end{document}