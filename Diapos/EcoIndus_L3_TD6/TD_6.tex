%\documentclass[ignorenonframetext, compress, 9pt, xcolor=svgnames]{beamer} 
\documentclass[notes, ignorenonframetext, compress, 10pt, xcolor=svgnames, aspectratio=169]{beamer} 
\usepackage{pgfpages}
\usepackage{pdfpages}
% These slides also contain speaker notes. You can print just the slides,
% just the notes, or both, depending on the setting below. Comment out the want
% you want.
\setbeameroption{hide notes} % Only slide
%\setbeameroption{show only notes} % Only notes
%\setbeameroption{show notes on second screen=right} % Both
\usepackage{amsmath}
\usepackage{amsfonts}
\usepackage{amssymb}
\setbeamercolor{frametitle}{fg=MidnightBlue}

\setbeamercolor{sectionpage title}{bg=MidnightBlue}
\setbeamertemplate{frametitle}[default][center]
%\setbeamertemplate{headline}{\vskip2cm}
%\setbeamertemplate{frametitle}{\color{MidnightBlue}\centering\bfseries\insertframetitle\par\vskip-6pt}
\setbeamerfont{frametitle}{series=\bfseries}
\setbeamerfont{title}{series=\bfseries}
\setbeamerfont{sectionpage}{series=\bfseries}
%\setbeamercolor{section in head/foot}{bg=MidnightBlueBlue}
%\setbeamercolor{author in head/foot}{bg=DarkBlue}
\setbeamercolor{author in head/foot}{fg=MidnightBlue}
%\setbeamercolor{title in head/foot}{bg=White}
\setbeamercolor{title in head/foot}{fg=MidnightBlue}
\setbeamercolor{title}{fg=MidnightBlue}
%\setbeamercolor{date in head/foot}{fg=Brown}
%\setbeamercolor{alerted text}{fg=DarkBlue}
%\usecolortheme[named=DarkBlue]{structure} 
%\usepackage{bbm}
%\usepackage{bbold}
\usepackage{eurosym}
\usepackage{graphicx}
%\usepackage{epstopdf}
\usepackage{hyperref}
\hypersetup{
  colorlinks   = true, %Colours links instead of ugly boxes
  urlcolor     = gray, %Colour for external hyperlinks
  linkcolor    = MidnightBlue, %Colour of internal links
  citecolor   = DarkRed %Colour of citations
}
\usepackage{multirow}
\usepackage{xspace}
\usepackage{listings}
\usepackage{natbib}
%\usepackage[sort&compress,comma,super]{natbib}
\def\newblock{} % To avoid a compilation error about a function \newblock undefined
\usepackage{bibentry}
\usepackage{booktabs}
\usepackage{dcolumn}
\usepackage[greek,frenchb]{babel}
\usepackage[babel=true,kerning=true]{microtype}
\usepackage[utf8]{inputenc}
\usepackage[T1]{fontenc}
\usepackage{natbib}
\renewcommand{\cite}{\citet}
\usepackage{longtable}
\usepackage{eso-pic}

\usepackage{xcolor}
 \colorlet{linkequation}{DarkRed} 
 \newcommand*{\SavedEqref}{}
 \let\SavedEqref\eqref 
\renewcommand*{\eqref}[1]{%
\begingroup \hypersetup{
      linkcolor=linkequation,
linkbordercolor=linkequation, }%
\SavedEqref{#1}%
 \endgroup
}

\newcommand*{\refeq}[1]{%
 \begingroup
\hypersetup{ 
linkcolor=linkequation, 
linkbordercolor=linkequation,
}%
\ref{#1}%
 \endgroup
}

\setbeamertemplate{caption}[numbered]
\setbeamertemplate{theorem}[ams style]
\setbeamertemplate{theorems}[numbered]
%\usefonttheme{serif}
%\usecolortheme{beaver}
%\usetheme{Hannover}
%\usetheme{CambridgeUS}
%\usetheme{Madrid}
%\usecolortheme{whale}
%\usetheme{Warsaw}
%\usetheme{Luebeck}
%\usetheme{Montpellier}
%\usetheme{Berlin}
%\setbeamercolor{titlelike}{parent=structure}
%\setbeamertemplate{headline}[default]
%\setbeamertemplate{footline}[default]
%\setbeamertemplate{footline}[Malmoe]
%\setbeamercovered{transparent}
%\setbeamercovered{invisible}
%\usecolortheme{crane}
%\usecolortheme{dolphin}
%\usepackage{pxfonts}
%\usepackage{isomath}
%\usepackage{mathpazo}
%\usepackage{arev} %     (Arev/Vera Sans)
%\usepackage{eulervm} %_   (Euler Math)
%\usepackage{fixmath} %  (Computer Modern)
%\usepackage{hvmath} %_   (HV-Math/Helvetica)
%\usepackage{tmmath} %_   (TM-Math/Times)
%\usepackage{tgheros}
%\usepackage{cmbright}
%\usepackage{ccfonts} \usepackage[T1]{fontenc}
%\usepackage[garamond]{mathdesign}

%\usepackage{color}
%\usepackage{ulem}

%\usepackage[math]{kurier}
%\usepackage[no-math]{fontspec}
%\setmainfont{Fontin Sans}
%\setsansfont{Fontin Sans}
%\setbeamerfont{frametitle}{size=\LARGE,series=\bfseries}
%%%add 19022021
\usepackage{enumerate}    
\usepackage{dcolumn}
\usepackage{verbatim}
\newcolumntype{d}[0]{D{.}{.}{5}}
%\setbeamertemplate{note page}{\pagecolor{yellow!5}\insertnote}
%\usetikzlibrary{positioning}
%\usetikzlibrary{snakes}
%\usetikzlibrary{calc}
%\usetikzlibrary{arrows}
%\usetikzlibrary{decorations.markings}
%\usetikzlibrary{shapes.misc}
%\usetikzlibrary{matrix,shapes,arrows,fit,tikzmark}
%%%
% suppress navigation bar
\beamertemplatenavigationsymbolsempty
%\usetheme{bunsenMod}
%\setbeamercovered{transparent}
%\setbeamertemplate{items}[circle]
%\usecolortheme[named=CadetBlue]{structure}
%\usecolortheme[RGB={225,64,5}]{structure}
%\definecolor{burntRed}{RGB}{225,64,5}
%\setbeamercolor{alerted text}{fg=burntRed} 
%\usecolortheme[RGB={0,40,110}]{structure}
%\hypersetup{linkcolor=burntRed}
%\hypersetup{urlcolor=burntRed}
%\hypersetup{filecolor=burntRed}
%\hypersetup{citecolor=burntRed}

%\usetheme{bunsenMod}
%\setbeamercovered{transparent}
%\setbeamertemplate{items}[circle]
%\usecolortheme[named=CadetBlue]{structure}
%\usecolortheme[RGB={225,64,5}]{structure}
%\definecolor{burntRed}{RGB}{225,64,5}
%\setbeamercolor{alerted text}{fg=burntRed} 
%\usecolortheme[RGB={0,40,110}]{structure}
%\hypersetup{linkcolor=burntRed}
%\hypersetup{urlcolor=burntRed}
%\hypersetup{filecolor=burntRed}
%\hypersetup{citecolor=burntRed}

%\AtBeginSection[] % Do nothing for \section*
%{ \frame{\sectionpage} }
%\setbeamertemplate{frametitle continuation}{}
\newtheorem{lemme}{Lemme}[section]
%\newtheorem{remarque}{Remarque}
\newcommand{\argmax}{\operatornamewithlimits{arg\,max}}
\newcommand{\argmin}{\operatornamewithlimits{arg\,min}}
\def\inprobLOW{\rightarrow_p}
\def\inprobHIGH{\,{\buildrel p \over \rightarrow}\,} 
\def\inprob{\,{\inprobHIGH}\,} 
\def\indist{\,{\buildrel d \over \rightarrow}\,} 
\def\sima{\,{\buildrel a \over \sim}\,} 
\def\F{\mathbb{F}}
\def\R{\mathbb{R}}
\def\N{\mathbb{N}}
\newcommand{\gmatrix}[1]{\begin{pmatrix} {#1}_{11} & \cdots &
    {#1}_{1n} \\ \vdots & \ddots & \vdots \\ {#1}_{m1} & \cdots &
    {#1}_{mn} \end{pmatrix}}
\newcommand{\iprod}[2]{\left\langle {#1} , {#2} \right\rangle}
\newcommand{\norm}[1]{\left\Vert {#1} \right\Vert}
\newcommand{\abs}[1]{\left\vert {#1} \right\vert}
\renewcommand{\det}{\mathrm{det}}
\newcommand{\rank}{\mathrm{rank}}
\newcommand{\spn}{\mathrm{span}}
\newcommand{\row}{\mathrm{Row}}
\newcommand{\col}{\mathrm{Col}}
\renewcommand{\dim}{\mathrm{dim}}
\newcommand{\prefeq}{\succeq}
\newcommand{\pref}{\succ}
\newcommand{\seq}[1]{\{{#1}_n \}_{n=1}^\infty }
\renewcommand{\to}{{\rightarrow}}
\renewcommand{\L}{{\mathcal{L}}}
\newcommand{\Er}{\mathrm{E}}
\renewcommand{\Pr}{\mathrm{P}}
%\newcommand{\Var}{\mathrm{Var}}
%\newcommand{\Cov}{\mathrm{Cov}}
%\newcommand{\corr}{\mathrm{Corr}}
%\newcommand{\Var}{\mathrm{Var}}
\newcommand{\bias}{\mathrm{Bias}}
\newcommand{\mse}{\mathrm{MSE}}
\providecommand{\Pred}{\mathcal{P}}
\providecommand{\plim}{\operatornamewithlimits{plim}}
\providecommand{\avg}{\frac{1}{n} \underset{i=1}{\overset{n}{\sum}}}
\providecommand{\sumin}{{\sum_{i=1}^n}}
\providecommand{\sumiN}{{\sum_{i=1}^N}}
\providecommand{\sumtT}{{\sum_{t=1}^T}}
\providecommand{\limp}{\overset{p}{\rightarrow}}
\providecommand{\liml}{\overset{L}{\rightarrow}}
%\providecommand{\limp}{\underset{n \rightarrow \infty}{\overset{p}{\longrightarrow}}}
%\providecommand{\limp}{\underset{n \rightarrow \infty}{\overset{p}{\longrightarrow}}}
%\providecommand{\limp}{\overset{p}{\longrightarrow}}
%\providecommand{\limd}{\underset{n \rightarrow \infty}{\overset{d}{\longrightarrow}}}
\providecommand{\limd}{\overset{d}{\rightarrow}}
\providecommand{\limps}{\overset{p.s.}{\rightarrow}}
\providecommand{\limlp}{\overset{L^p}{\rightarrow}}
\providecommand{\limprob}{\overset{p}{\underset{N\to +\infty}{\longrightarrow}}}
\providecommand{\limloi}{\overset{L}{\underset{N\to +\infty}{\longrightarrow}}}
\providecommand{\limpsure}{\overset{p.s.}{\underset{N\to +\infty}{\longrightarrow}}}
\def\independenT#1#2{\mathrel{\setbox0\hbox{$#1#2$}%
    \copy0\kern-\wd0\mkern4mu\box0}} 
\newcommand\indep{\protect\mathpalette{\protect\independenT}{\perp}}


\lstset{language=R}
\lstset{keywordstyle=\color[rgb]{0,0,1},                                        % keywords
        commentstyle=\color[rgb]{0.133,0.545,0.133},    % comments
        stringstyle=\color[rgb]{0.627,0.126,0.941}      % strings
}       
\lstset{
  showstringspaces=false,       % not emphasize spaces in strings 
  columns=fixed,
  numbersep=3mm, numbers=left, numberstyle=\tiny,       % number style
  frame=none,
  framexleftmargin=5mm, xleftmargin=5mm         % tweak margins
}
\makeatletter
%\setbeamertemplate{frametitle continuation}{\gdef\beamer@frametitle{}}
\setbeamertemplate{frametitle continuation}{\frametitle{}}
%\setbeamertemplate{frametitle continuation}{\insertcontinuationcount}
\makeatother

\theoremstyle{remark}
\newtheorem{interpretation}{Interprétation}
\newtheorem*{interpretation*}{Interprétation}

\theoremstyle{remark}
\newtheorem{remarque}{Remarque}%[section]
\newtheorem*{remarque*}{Remarque}
\usepackage[framemethod=TikZ]{mdframed} 
\usepackage{showexpl}
%\newtheorem{step}{Step}[section]
%\newtheorem{rem}{Comment}[section]
%\newtheorem{ex}{Example}[section]
%\newtheorem{hist}{History}[section]
%\newtheorem*{ex*}{Example}
%\theoremstyle{plain}
%\newtheorem{propriete}{Propri\'et\'e}
%\renewcommand{\thepropriete}{P\arabic{propriete}}
%\theoremstyle{definition}
%\newtheorem{definition}{Définition}%[section]
%\theoremstyle{remark}
%\newtheorem{exemple}{Exemple}
%\newtheorem*{exemple*}{Exemple}

\newtheorem{theoreme}{Théorème}
\newtheorem{proposition}{Proposition}
%\newtheorem{propriete}{Propri\'et\'e}
\newtheorem{corollaire}{Corollaire}
%\newtheorem{exemple}{Exemple}
%\newtheorem{assumption}{Assumption}
%\renewcommand{\theassumption}{A\arabic{assumption}}
\newtheorem{hypothese}{Hypothèse}
\renewcommand{\thehypothese}{H\arabic{hypothese}}
%\theoremstyle{definition}

%\newtheorem{definitionx}{D\'efinition}%[section]
%\newenvironment{definition}
 %{\pushQED{\qed}\renewcommand{\qedsymbol}{$\triangle$}\definitionx}
 %{\popQED\enddefinitionx}

%\newtheorem{condition}{Condition}
%\renewcommand{\thecondition}{C\arabic{condition}}
%\newcommand{\Var}{\mathbb{V}}
%\newcommand{\Var}{\mathbf{Var}}
%\newcommand{\Exp}{\mathbf{E}}
%\providecommand{\Vr}{\mathrm{Var}}
%\renewcommand{\Er}{\mathbb{E}}
%\newcommand{\LP}{\mathcal{LP}}
%\providecommand{\Id}{\mathbf{I}}
%\providecommand{\Rang}{\mathrm{Rang}}
%\providecommand{\Trace}{\mathrm{Trace}}
%\newcommand{\Cov}{\mathbf{Cov}}
%\newcommand{\Cov}{\mathbb{C}\mathrm{ov}}
\providecommand{\Id}{\mathbf{I}}
\providecommand{\Ind}{\mathbf{1}}
\providecommand{\uvec}{\mathbf{1}}
\providecommand{\vecOnes}{\mathbf{1}}
\DeclareMathOperator{\indfun}{\mathbf{1}}
\DeclareMathOperator{\Exp}{E}
\DeclareMathOperator{\Expn}{\mathbb{E}_n}
\DeclareMathOperator{\EL}{EL}
\DeclareMathOperator{\Var}{Var}
\DeclareMathOperator{\Vr}{V}
\newcommand{\boldVr}{ {\boldsymbol \Vr} }
\DeclareMathOperator{\Cov}{Cov}
\DeclareMathOperator{\corr}{corr}
\DeclareMathOperator{\perps}{\perp_s}
%\DeclareMathOperator{\Prob}{Pr}
\DeclareMathOperator{\Prob}{P}
\DeclareMathOperator{\prob}{p}
\DeclareMathOperator{\loss}{L}
\providecommand{\Corr}{\mathrm{Corr}}
\providecommand{\Diag}{\mathrm{Diag}}
\providecommand{\reg}{\mathrm{r}}
\providecommand{\Likelihood}{\mathrm{L}}
\renewcommand{\Pr}{{\mathbb{P}}}
\providecommand{\set}[1]{\left\{#1\right\}}
\providecommand{\uvec}{\mathbf{1}}
\providecommand{\Rang}{\mathrm{Rang}}
\providecommand{\Trace}{\mathrm{Trace}}
\providecommand{\Tr}{\mathrm{Tr}}
\providecommand{\CI}{\mathrm{CI}}
\providecommand{\asyvar}{\mathrm{AsyVar}}
\DeclareMathOperator{\Supp}{Supp}
\newcommand{\inputslide}[2]{{
    \usebackgroundtemplate{
     \includegraphics[page={#2},width=0.90\textwidth,keepaspectratio=true]
      %\includegraphics[page={#2},width=\paperwidth,keepaspectratio=true]
      {{#1}}}
    \frame[plain]{}
  }}
\newcommand\pperp{\perp\!\!\!\perp}
\newcommand\independent{\protect\mathpalette{\protect\independenT}{\perp}}
\def\independenT#1#2{\mathrel{\rlap{$#1#2$}\mkern2mu{#1#2}}}
\usepackage{bbm}
\providecommand{\Ind}{\mathbf{1}}
\newcommand{\sumjsi}{\underset{i<j}{{\sum}}}
\newcommand{\prodjsi}{\underset{i<j}{{\prod}}}
\newcommand{\sumisj}{\underset{j<i}{{\sum}}}
\newcommand{\prodisj}{\underset{j<i}{{\prod}}}
\newcommand{\sumobs}{\underset{i=1}{\overset{n}{\sum}}}
\newcommand{\sumi}{\underset{i=1}{\overset{n}{\sum}}}
\newcommand{\prodi}{\underset{i=1}{\overset{n}{\prod}}}
\newcommand{\prodobs}{\underset{i=1}{\overset{n}{\prod}}}
\newcommand{\simiid}{{\overset{i.i.d.}{\sim}}}
%\newcommand{\sumobs}{\sum_{i=1}^N}
%\newcommand{\prodobs}{\prod_{i=1}^N}
%\newcommand{\sumjsi}{\sum_{i<j}}
%\newcommand{\prodjsi}{\prod_{i<j}}
%\newcommand{\sumisj}{\sum_{j<i}}
%\newcommand{\prodisj}{\sum_{j<i}}

%\usepackage{appendixnumberbeamer}
\setbeamertemplate{footline}[frame number]
\setbeamertemplate{section in toc}[sections numbered]
\setbeamertemplate{subsection in toc}[subsections numbered]
\setbeamertemplate{subsubsection in toc}[subsubsections numbered]

%\makeatother
%\setbeamertemplate{footline}
%{
%    \leavevmode%
%    \hbox{%
%        \begin{beamercolorbox}[wd=.333333\paperwidth,ht=2.25ex,dp=1ex,center]{author in head/foot}%
%            \usebeamerfont{author in head/foot}\insertshortauthor
%        \end{beamercolorbox}%
%        \begin{beamercolorbox}[wd=.333333\paperwidth,ht=2.25ex,dp=1ex,center]{title in head/foot}%
%            \usebeamerfont{title in head/foot}\insertshorttitle
%        \end{beamercolorbox}%
%        \begin{beamercolorbox}[wd=.333333\paperwidth,ht=2.25ex,dp=1ex,right]{date in head/foot}%
%            \usebeamerfont{date in head/foot}\insertshortdate{}\hspace*{2em}
%            \insertframenumber{} / \inserttotalframenumber\hspace*{2ex} 
%        \end{beamercolorbox}}%
%       \vskip0pt%
 %   }
%   \makeatother
%\setbeamertemplate{navigation symbols}{}
\setbeamertemplate{itemize items}[ball]
%\setbeamertemplate{itemize items}{-}
%\newenvironment{wideitemize}{\itemize\addtolength{\itemsep}{10pt}}{\enditemize}
% \usepackage{eso-pic}
%\newcommand\AtPagemyUpperLeft[1]{\AtPageLowerLeft{%
%\put(\LenToUnit{0.9\paperwidth},\LenToUnit{0.9\paperheight}){#1}}}
%\AddToShipoutPictureFG{
%  \AtPagemyUpperLeft{{\includegraphics[width=1.1cm,keepaspectratio]{../logo-uga.png}}}
%}%
\def\figheight{3in}
\def\figwidth{4in}

%%Commands from Econometric Theory(Slides) by J. Stachurski.

\newcommand{\boldx}{ {\mathbf x} }
\newcommand{\boldu}{ {\mathbf u} }
\newcommand{\boldv}{ {\mathbf v} }
\newcommand{\boldw}{ {\mathbf w} }
\newcommand{\boldy}{ {\mathbf y} }
\newcommand{\boldb}{ {\mathbf b} }
\newcommand{\bolda}{ {\mathbf a} }
\newcommand{\boldc}{ {\mathbf c} }
\newcommand{\boldd}{ {\mathbf d} }
\newcommand{\boldi}{ {\mathbf i} }
\newcommand{\bolde}{ {\mathbf e} }
\newcommand{\boldp}{ {\mathbf p} }
\newcommand{\boldq}{ {\mathbf q} }
\newcommand{\bolds}{ {\mathbf s} }
\newcommand{\boldt}{ {\mathbf t} }
\newcommand{\boldz}{ {\mathbf z} }
\newcommand{\boldr}{ {\mathbf r} }
\newcommand{\boldm}{ {\mathbf m} }

\newcommand{\boldzero}{ {\mathbf 0} }
\newcommand{\boldone}{ {\mathbf 1} }

\newcommand{\boldalpha}{ {\boldsymbol \alpha} }
\newcommand{\boldbeta}{ {\boldsymbol \beta} }
\newcommand{\boldgamma}{ {\boldsymbol \gamma} }
\newcommand{\boldGamma}{ {\boldsymbol \Gamma} }
\newcommand{\boldtheta}{ {\boldsymbol \theta} }
\newcommand{\boldxi}{ {\boldsymbol \xi} }
\newcommand{\boldtau}{ {\boldsymbol \tau} }
\newcommand{\boldepsilon}{ {\boldsymbol \epsilon} }
\newcommand{\boldvepsilon}{ {\boldsymbol \varepsilon} }
\newcommand{\boldmu}{ {\boldsymbol \mu} }
\newcommand{\boldSigma}{ {\boldsymbol \Sigma} }
\newcommand{\boldOmega}{ {\boldsymbol \Omega} }
\newcommand{\boldPhi}{ {\boldsymbol \Phi} }
\newcommand{\boldLambda}{ {\boldsymbol \Lambda} }
\newcommand{\boldphi}{ {\boldsymbol \phi} }
\newcommand{\boldeta}{ {\boldsymbol \eta} }

\newcommand{\Sigmax}{ {\boldsymbol \Sigma_{\boldx}}}
\newcommand{\Sigmau}{ {\boldsymbol \Sigma_{\boldu}}}
\newcommand{\Sigmaxinv}{ {\boldsymbol \Sigma_{\boldx}^{-1}}}
\newcommand{\Sigmav}{ {\boldsymbol \Sigma_{\boldv \boldv}}}

\newcommand{\hboldx}{ \hat {\mathbf x} }
\newcommand{\hboldy}{ \hat {\mathbf y} }
\newcommand{\hboldb}{ \hat {\mathbf b} }
\newcommand{\hboldu}{ \hat {\mathbf u} }
\newcommand{\hboldtheta}{ \hat {\boldsymbol \theta} }
\newcommand{\hboldtau}{ \hat {\boldsymbol \tau} }
\newcommand{\hboldmu}{ \hat {\boldsymbol \mu} }
\newcommand{\hboldbeta}{ \hat {\boldsymbol \beta} }
\newcommand{\hboldgamma}{ \hat {\boldsymbol \gamma} }
\newcommand{\hboldSigma}{ \hat {\boldsymbol \Sigma} }

\newcommand{\boldA}{\mathbf A}
\newcommand{\boldB}{\mathbf B}
\newcommand{\boldC}{\mathbf C}
\newcommand{\boldD}{\mathbf D}
\newcommand{\boldI}{\mathbf I}
\newcommand{\boldL}{\mathbf L}
\newcommand{\boldM}{\mathbf M}
\newcommand{\boldP}{\mathbf P}
\newcommand{\boldQ}{\mathbf Q}
\newcommand{\boldR}{\mathbf R}
\newcommand{\boldX}{\mathbf X}
\newcommand{\boldU}{\mathbf U}
\newcommand{\boldV}{\mathbf V}
\newcommand{\boldW}{\mathbf W}
\newcommand{\boldY}{\mathbf Y}
\newcommand{\boldZ}{\mathbf Z}

\newcommand{\bSigmaX}{ {\boldsymbol \Sigma_{\hboldbeta}} }
\newcommand{\hbSigmaX}{ \mathbf{\hat \Sigma_{\hboldbeta}} }
\newcommand{\betahat}{\hat{\beta}}
\newcommand{\gammahat}{\hat{\gamma}}
\newcommand{\Uhat}{\hat{U}}
\newcommand{\Vhat}{\hat{V}}
\newcommand{\epsilonhat}{\hat{\epsilon}}
\newcommand{\sigmahat}{\hat{\sigma}}
\newcommand{\Sigmahat}{\hat{\Sigma}}
\newcommand{\Gammahat}{\hat{\Gamma}}

\newcommand{\RR}{\mathbbm R}
\newcommand{\CC}{\mathbbm C}
\newcommand{\NN}{\mathbbm N}
\newcommand{\PP}{\mathbbm P}
\newcommand{\EE}{\mathbbm E \nobreak\hspace{.1em}}
\newcommand{\EEP}{\mathbbm E_P \nobreak\hspace{.1em}}
\newcommand{\ZZ}{\mathbbm Z}
\newcommand{\QQ}{\mathbbm Q}


\newcommand{\XX}{\mathbbm X}

\newcommand{\aA}{\mathcal A}
\newcommand{\fF}{\mathscr F}
\newcommand{\bB}{\mathscr B}
\newcommand{\iI}{\mathscr I}
\newcommand{\rR}{\mathscr R}
\newcommand{\dD}{\mathcal D}
\newcommand{\lL}{\mathcal L}
\newcommand{\llL}{\mathcal{H}_{\ell}}
\newcommand{\gG}{\mathcal G}
\newcommand{\hH}{\mathcal H}
\newcommand{\nN}{\textrm{\sc n}}
\newcommand{\lN}{\textrm{\sc ln}}
\newcommand{\pP}{\mathscr P}
\newcommand{\qQ}{\mathscr Q}
\newcommand{\xX}{\mathcal X}
\newcommand{\yY}{\mathcal Y}
\newcommand{\ddD}{\mathscr D}


%\newcommand{\R}{{\texttt R}}
\newcommand{\risk}{\mathcal R}
\newcommand{\Remp}{R_{{\rm emp}}}

\newcommand*\diff{\mathop{}\!\mathrm{d}}
\newcommand{\ess}{ \textrm{{\sc ess}} }
\newcommand{\tss}{ \textrm{{\sc tss}} }
\newcommand{\rss}{ \textrm{{\sc rss}} }
\newcommand{\rssr}{ \textrm{{\sc rssr}} }
\newcommand{\ussr}{ \textrm{{\sc ussr}} }
\newcommand{\zdata}{\mathbf{z}_{\mathcal D}}
\newcommand{\Pdata}{P_{\mathcal D}}
\newcommand{\Pdatatheta}{P^{\mathcal D}_{\theta}}
\newcommand{\Zdata}{Z_{\mathcal D}}


\newcommand{\e}[1]{\mathbbm{E}[{#1}]}
\newcommand{\p}[1]{\mathbbm{P}({#1})}

% condition
\theoremstyle{definition}
\newtheorem{condition}{Condition}
\renewcommand{\thecondition}{C\arabic{condition}}
\BeforeBeginEnvironment{condition}{
  \setbeamerfont{block title}{series=\bfseries}
  \setbeamercolor{block title}{fg=MidnightBlue,bg=white}
  \setbeamercolor{block body}{fg=black, bg=gray!10}
}
\newtheorem*{condition*}{Condition}
\BeforeBeginEnvironment{condition*}{
  \setbeamerfont{block title}{series=\bfseries}
  \setbeamercolor{block title}{fg=MidnightBlue,bg=white}
  \setbeamercolor{block body}{fg=black, bg=gray!10}
}

% assumption
\theoremstyle{definition}
\newtheorem{assumption}{Assumption}
\BeforeBeginEnvironment{assumption}{
  \setbeamerfont{block title}{series=\bfseries}
  \setbeamercolor{block title}{fg=MidnightBlue,bg=white}
  \setbeamercolor{block body}{fg=black, bg=gray!10}
}
\newtheorem*{assumption*}{Assumption}
\BeforeBeginEnvironment{assumption*}{
  \setbeamerfont{block title}{series=\bfseries}
  \setbeamercolor{block title}{fg=MidnightBlue,bg=white}
  \setbeamercolor{block body}{fg=black, bg=gray!10}
}

% definition
\BeforeBeginEnvironment{definition}{
  \setbeamerfont{block title}{series=\bfseries}
  \setbeamercolor{block title}{fg=MidnightBlue,bg=white}
  \setbeamercolor{block body}{fg=black, bg=gray!10}
}
\newtheorem*{definition*}{Definition}
\BeforeBeginEnvironment{definition*}{
  \setbeamerfont{block title}{series=\bfseries}
  \setbeamercolor{block title}{fg=MidnightBlue,bg=white}
  \setbeamercolor{block body}{fg=black, bg=gray!10}
}

% theorem
\theoremstyle{plain}
\BeforeBeginEnvironment{theorem}{
  \setbeamerfont{block body}{shape=\itshape}
  \setbeamerfont{block title}{series=\bfseries}
  \setbeamercolor{block title}{fg=MidnightBlue,bg=white}
  \setbeamercolor{block body}{fg=black, bg=gray!10}
}
\newtheorem*{theorem*}{Theorem}
\BeforeBeginEnvironment{theorem*}{
  \setbeamerfont{block body }{shape=\itshape}
  \setbeamerfont{block title}{series=\bfseries}
  \setbeamercolor{block title}{fg=MidnightBlue,bg=white}
  \setbeamercolor{block body}{fg=black, bg=gray!10}
}

% definition_fr
\theoremstyle{definition}
\newtheorem{definition_fr}{Définition}%[section]
\BeforeBeginEnvironment{definition_fr}{
  \setbeamerfont{block title}{series=\bfseries}
  \setbeamercolor{block title}{fg=MidnightBlue,bg=white}
  \setbeamercolor{block body}{fg=black, bg=gray!10}
}
\newtheorem*{definition_fr*}{Définition}
\BeforeBeginEnvironment{definition_fr*}{
  \setbeamerfont{block title}{series=\bfseries}
  \setbeamercolor{block title}{fg=MidnightBlue,bg=white}
  \setbeamercolor{block body}{fg=black, bg=gray!10}
}
% theorem_fr
\newtheorem{theorem_fr}{Théorème}%[section]
\BeforeBeginEnvironment{theorem_fr}{
  \setbeamerfont{block body}{shape=\itshape}
  \setbeamerfont{block title}{series=\bfseries, shape = \upshape}
  \setbeamercolor{block title}{fg=MidnightBlue,bg=white}
  \setbeamercolor{block body}{fg=black, bg=gray!10}
}
\newtheorem*{theorem_fr*}{Théorème}
\BeforeBeginEnvironment{theorem_fr*}{
  \setbeamerfont{block body}{shape=\itshape}
  \setbeamerfont{block title}{series=\bfseries, shape = \upshape}
  \setbeamercolor{block title}{fg=MidnightBlue,bg=white}
  \setbeamercolor{block body}{fg=black, bg=gray!10}
}

% remark_fr
\theoremstyle{remark}
\newtheorem{remark_fr}{Remarque}%[section]
\BeforeBeginEnvironment{remark_fr}{
  \setbeamerfont{block title}{series=\bfseries, shape=\itshape}
  \setbeamercolor{block title}{fg=MidnightBlue,bg=white}
  \setbeamercolor{block body}{fg=black, bg=gray!10}
}
\newtheorem*{remark_fr*}{Remarque}
\BeforeBeginEnvironment{remark_fr*}{
  \setbeamerfont{block title}{series=\bfseries, shape=\itshape}
  \setbeamercolor{block title}{fg=MidnightBlue,bg=white}
  \setbeamercolor{block body}{fg=black, bg=gray!10}
}

% exemple
\theoremstyle{remark}
\newtheorem{exemple}{Exemple}%[section]
\BeforeBeginEnvironment{exemple}{
  \setbeamerfont{block title}{series=\bfseries, shape=\itshape}
  \setbeamercolor{block title}{fg=MidnightBlue,bg=white}
  \setbeamercolor{block body}{fg=black, bg=gray!10}
}
\newtheorem*{exemple*}{}
\BeforeBeginEnvironment{exemple*}{
  \setbeamerfont{block title}{series=\bfseries, shape=\itshape}
  \setbeamercolor{block title}{fg=MidnightBlue,bg=white}
  \setbeamercolor{block body}{fg=black, bg=gray!10}
}


% propriete
\theoremstyle{plain}
\newtheorem{propriete}{Propri\'et\'e}%[section]
\BeforeBeginEnvironment{propriete}{
  \setbeamerfont{block body}{shape=\itshape}
  \setbeamerfont{block title}{series=\bfseries, shape = \upshape}
  \setbeamercolor{block title}{fg=MidnightBlue,bg=white}
  \setbeamercolor{block body}{fg=black, bg=gray!10}
}
\newtheorem*{propriete*}{Propri\'et\'e}
\BeforeBeginEnvironment{propriete*}{
  \setbeamerfont{block body}{shape=\itshape}
  \setbeamerfont{block title}{series=\bfseries, shape = \upshape}
  \setbeamercolor{block title}{fg=MidnightBlue,bg=white}
  \setbeamercolor{block body}{fg=black, bg=gray!10}
}


% remark
\theoremstyle{remark}
\newtheorem{remark}{Remark}%[section]
\BeforeBeginEnvironment{remark}{
  \setbeamerfont{block body}{shape=\itshape}
  \setbeamerfont{block title}{series=\bfseries}
  \setbeamercolor{block title}{fg=MidnightBlue,bg=white}
  \setbeamercolor{block body}{fg=black, bg=gray!10}
}
\newtheorem*{remark*}{Remark}
\BeforeBeginEnvironment{remark*}{
  \setbeamerfont{block body }{shape=\itshape}
  \setbeamerfont{block title}{series=\bfseries}
  \setbeamercolor{block title}{fg=MidnightBlue,bg=white}
  \setbeamercolor{block body}{fg=black, bg=gray!10}
}


\usepackage{color}
\usepackage{tikz}
\usetikzlibrary{shapes.geometric, arrows}
\usepackage{enumerate}   
\usepackage{multirow}
%\setbeamersize{text margin left=1.5em,text margin right=1.5em} 
%\setbeamersize{text margin left=1.2cm,text margin right=1.2cm} 
\setbeamersize{text margin left=1.5em,text margin right=1.5em} 
%\usepackage{xr}
%\externaldocument{Econometrie1_UGA_P2e}
  \usepackage{eso-pic}
%\newcommand\AtPagemyUpperLeft[1]{\AtPageLowerLeft{%
%\put(\LenToUnit{0.9\paperwidth},\LenToUnit{0.85\paperheight}){#1}}}
%\AddToShipoutPictureFG{
 % \AtPagemyUpperLeft{{\includegraphics[width=1.1cm,keepaspectratio]{logoUGA2020.pdf}}}
%}%

%\setbeamercolor{title}{fg=black}
%\setbeamercolor{frametitle}{fg=black}
%\setbeamercolor{section in head/foot}{fg=black}
%\setbeamercolor{author in head/foot}{bg=Brown}
%\setbeamercolor{date in head/foot}{fg=Brown}
\setbeamertemplate{section page}
{
    \begin{centering}
    \begin{beamercolorbox}[sep=11pt,center]{part title}
    \usebeamerfont{section title}\thesection.~\insertsection\par
    \end{beamercolorbox}
    \end{centering}
}
%\titlegraphic{\includegraphics[width=1cm]{logoUGA2020.pdf}}
\title[]{ \textbf{Économie Industrielle}\footnote{responsable du cours: Sylvain Rossiaud} \\ (UGA, L3 Éco, S2) \\ }
\subtitle{Travaux dirigés: No 6\\ FUSIONS\\
(éléments de correction)}
\date{\today}
\author{Michal W. Urdanivia\inst{*}}
\institute{\inst{*}UGA, Facult\'e d'\'Economie, GAEL, \\
e-mail:
 \href{
     mailto:michal.wong-urdanivia@univ-grenoble-alpes.fr}{michal.wong-urdanivia@univ-grenoble-alpes.fr}}

%\titlegraphic{\includegraphics[width=1cm]{logoUGA2020.pdf}
%}

\begin{document}

%%% TIKZ STUFF
\usetikzlibrary{positioning}
\usetikzlibrary{snakes}
\usetikzlibrary{calc}
\usetikzlibrary{arrows}
\usetikzlibrary{decorations.markings}
\usetikzlibrary{shapes.misc}
\usetikzlibrary{matrix,shapes,arrows,fit,tikzmark}
\usetikzlibrary{shapes}
\usetikzlibrary{shapes.geometric, arrows}
\tikzset{   
        every picture/.style={remember picture,baseline},
        every node/.style={anchor=base,align=center,outer sep=1.5pt},
        every path/.style={thick},
        }
\newcommand\marktopleft[1]{
    \tikz[overlay,remember picture] 
        \node (marker-#1-a) at (-.3em,.3em) {};%
}
\newcommand\markbottomright[2]{%
    \tikz[overlay,remember picture] 
        \node (marker-#1-b) at (0em,0em) {};%
}
\tikzstyle{every picture}+=[remember picture] 
\tikzstyle{mybox} =[draw=black, very thick, rectangle, inner sep=10pt, inner ysep=20pt]
\tikzstyle{fancytitle} =[draw=black,fill=red, text=white]
\tikzstyle{observed}=[draw,circle,fill=gray!50]

\begin{frame}
\titlepage
\end{frame}
\begin{frame}
 \tableofcontents
    \end{frame}
%\begin{frame}
%\frametitle{Contenu}
%\tableofcontents[pausesections, pausesubsections]
%\end{frame}

%\section{Qu'est-ce que l’économétrie ? A quoi (à qui) ça sert ?}
%\frame{\sectionpage}
%\begin{frame}
%  \tableofcontents  
%\end{frame}


\section{Rappels sur le modèle de Cournot(de base) et à $N$ firmes}
\frame{\sectionpage}
\begin{frame}[allowframebreaks]{Modèle}
\begin{itemize}
\item \textbf{\underline{Référence}}: \cite{belleflamme_peitz_2015}, chapitre 3.
\item $N$ firmes avec des fonctions de coûts symétriques à coûts marginaux constants:
\begin{align*}
    c_i(q_i) &=cq_i \Rightarrow c^m_i(q_i):= \frac{\partial c}{\partial q_i}(q_i) = c, \ c > 0, i=1, \ldots, N.
\end{align*}
\item La demande est représenté par une fonction demande inverse linéaire:
\begin{align*}
p(Q) &= a-bQ, \ a, b > 0, Q = \sum_{i=1}^N q_i
\end{align*}
\item Le profit d'une firme $i$ s'écrit alors: 
\begin{align*}
\pi_i(q_i, q_{(-i)}) = p(Q)q_i - c_i(q_i) =  (a-bQ)q_i - cq_i &= (a-b\sum_{i=1}^N q_i)q_i - cq_i\\
& = \left(a-b(q_i + \sum_{j=1, j\neq i}^N q_j)\right)q_i - cq_i\\
&= \left(a-b(q_i + q_{-i})\right)q_i - cq_i
\end{align*}
où on note $q_{-i}:= \sum_{j=1, j\neq i}^N q_j $ la quantité(totale) offerte par les autres firmes que la firme $i$.
\item Bien que $\pi_i$ dépende des quantités offertes par les autres firmes, 
la firme $i$ maximise $\pi_i$ par rapport à $q_i$ qui est sa seule variable de décision.
\item Autrement dit, chaque firme maximise son profit étant donné les décisions des autres.
\end{itemize}
\end{frame}

\begin{frame}[allowframebreaks]{Fonctions de meilleure réponse}
    \begin{itemize}
        \item La c.p.o. devant être vérifiée par le choix optimal de la firme $i$, $q_i^*$ s'écrit: 
        \begin{align}
            \frac{\partial \pi_i}{\partial q_i}(q_i^*) =0 &\Leftrightarrow a-2bq_i -b\sum_{j=1, j\neq i }^N q_j - c = 0\\
            &\Leftrightarrow  a-2bq_i^* -b\left(Q - q_i^*\right) - c= 0,
            \label{eq1}
        \end{align}
         qui définit le choix optimal de la firme $i$ comme une fonction(implicite) de $q_{(-i)}$ qui apparaît 
         dans $Q$. 
         \item Cette fonction est la meilleure réponse de la firme $i$ qu'on note $q^{mr}_i(q_{-i})$ avec 
         $q_i^* = q^{mr}_i(q_{-i})$. Elle nous est donnée en utilisant \eqref{eq1} par: 
        \begin{align} 
            q_i^{mr}(q_{(-i)}) &= \frac{(a-c)}{b} - Q, \ i=1, \ldots, N.
            \label{eq2}
        \end{align}
           \item L'équilibre du marché est un équilibre de Nash du jeux en information complète, où
           la stratégie de chaque joueur est donné par \eqref{eq2}. 
           Autrement dit le vecteur des quantités d'équilibre $q_1^*, \ldots, q_N^*$ vérifie 
           \begin{align}
               q_i^*&= q_i^{mr}(q_{-i}^*), \ i=1, \ldots, N.
               \label{eq3}
           \end{align}
           où $q_i^{mr}$ est donnée par \eqref{eq2}. En notant $Q^*$ la quantité totale offerte 
           à l'équilibre \eqref{eq3} s'écrit:
           \begin{align*}
            q_i^*&=\frac{(a-c)}{b} - Q^*, \ i=1, \ldots, N.
           \end{align*}
           \item On peut utiliser cette dernière égalité et voir que $Q^*$ vérifie:
           \begin{align*}
            \sum_{i=1}^N  q_i^* = N\left(\frac{(a-c)}{b} - Q^*\right)
            &\Leftrightarrow Q^* = N\frac{(a-c)}{b} - NQ^*
            \Rightarrow  Q^* = \frac{N}{(N+1)}\frac{(a-c)}{b}.
           \end{align*}
           \item On obtient alors les quantités offertes par chaque firme à l'équilibre en utilisant
            \eqref{eq2}:
            \begin{align*} 
                q_i^{mr}(q_{-i^*}) &= \frac{(a-c)}{b} - Q^* = \frac{(a-c)}{(N+1)b}, \ i=1, \ldots, N.
            \end{align*}
            ce qui permets de calculer le prix et profits à l'équilibre: 
            \begin{align*}
                p^* &=p(Q^*) = a-bQ^* = a-b\left(\frac{N}{(N+1)}\frac{(a-c)}{b}\right)=\frac{a+Nc}{N+1},\\
                \pi_i^* &=\pi_i(q_i^*,q_{(-i)}^*)=p^*q_i^* - cq_i^* = (p^*-c)q_i^* =\frac{1}{b}\left(\frac{a-c}{N+1}\right)^2
            \end{align*}    
    \end{itemize}
\end{frame}

\begin{frame}[allowframebreaks]{Fusion}
    \begin{itemize}
    \item \textbf{\underline{Référence}}: \cite{belleflamme_peitz_2015}, chapitre 15.
    \item Supposons que $K<N$  parmi les $N$ firmes fusionnent.
    \item Le nombre de firmes est alors:
    \begin{align*}
        \tilde{N} &= \underbrace{N-K}_{\substack{\text{firmes qui}\\\text{ne fusionnent pas}}} + \underbrace{1}_{\text{firme fusionnée}}
    \end{align*}
    \item Dans ce cas d'après les résultats précédents le profit de chaque firme est à l'équilibre:
    \begin{align*}
        \pi_{i, \tilde{N}}^* &= \frac{1}{b}\left(\frac{a-c}{\tilde{N}+1}\right)^2
        = \frac{1}{b}\left(\frac{a-c}{N-K+2}\right)^2.
    \end{align*}
    \item On se rappelle que leur profit avant la fusion est pour chacune:
    \begin{align*}
        \pi_i^* &= \frac{1}{b}\left(\frac{a-c}{N+1}\right)^2
    \end{align*}
    \item Ce qui donne un profit total pour les ayant fusionné avant la fusion de:
    \begin{align*}
        K\pi_i^* &= K\frac{1}{b}\left(\frac{a-c}{N+1}\right)^2
    \end{align*}
    \item Pour que la fusion soit profitable il faut alors que:
    \begin{align*}
        \pi_{i, \tilde{N}}^* \geq K\pi_i^*& \Leftrightarrow
         \frac{1}{b}\left(\frac{a-c}{N-K+2}\right)^2  \geq K\frac{1}{b}\left(\frac{a-c}{N+1}\right)^2\\
         & \Leftrightarrow (N+1)^2 \geq K(N-K+2)^2\\
    \end{align*}
    \item On peut alors considérer l'équation:
    \begin{align*}
        f(K) &= (N+1)^2 - K(N-K+2)^2,
    \end{align*}
    pour chercher les racines  de $f(K) = 0$, et ce qui permettra de définir le seuil $\tilde{K}$ 
    tel que pour $K > \tilde{K}$ $f(K) > 0$ et la fusion est profitable aux entreprises qui fusionnent.
    \item $f(K) = 0$ est une équation polynomiale de degré 3 dont les racines sont :
    \begin{align*}
        K_1, =1, K_2 = N - \sqrt{(4N + 5)}/2 + 3/2, K_3 = N + \sqrt{(4N + 5)}/2 + 3/2
    \end{align*}
    \item \textbf{Remarque}: 
    \begin{enumerate}[$\star$]
    \item il existe des méthodes connues pour résoudre ces équations, si vous n'en avez 
    vu commencez \href{https://www.lucaswillems.com/fr/articles/58/equations-troisieme-degre?cache=update}{ici par exemple}.
    \item ici, pour faire vite j'ai utilisé un outil de calcul symbolique avec dans Python avec 
    \href{https://www.sympy.org/en/index.html}{SimPy}.
    \end{enumerate}
    \item Parmi ces racines seulement la 2ème donne une région admissible donc $\tilde{K}=K_2$
    \item Si dans $\tilde{K}$ on fait varier $N$ on obtiendra qu'en général $\tilde{K}=0.8$.
    \item C'est ce qui est souvent appelé \textbf{règle du 80\%}: il faut que plus de 80\% des firmes fusionnent 
    pour que dans ce modèle cette fusion soit profitable à ces entreprises. Ici cela correspond à 9 firmes, 
    soit une situation de quasi monopole(en fait un duopole avec les firmes fusionnées et la firme restante)
\end{itemize}
\end{frame}

\section{TD: Partie 1}
\frame{\sectionpage}
\begin{frame}[allowframebreaks]{Exercice}
    \begin{itemize}
        \item Fonction de demande inverse: $p(Q) = 200-Q$, $Q=q_1+\ldots + q_{10}$, 
        \item Fonction de coût: $c_i(q_i) = 40q_i$, pour $i=1, \ldots, 10$.
    \end{itemize}
    \begin{enumerate}
        \item $N = 10$: d'après ce qui précède:
        \begin{enumerate}[$\star$]
        \item $Q^*_{(N=10)} = 145.5$, 
        \item $p^*_{(N=10)}=54.5$, 
        \item $q_{(i, N=10)}^* = 14.5$,  
        \item $\pi_{(i, N=10)}^*= 211.6$. 
        \item Ce dernier chiffre donnant un surplus des producteurs de $SP_{(i, N=10)}^* = N\pi_{(N=10)}^*=2116$.
        \item  On calcule aussi le surplus du consommateur qui dans le cas d'une fonction de demande inverse 
        linéaire $p(Q) = a-bQ$ est donné par $SC(Q) = \frac{bQ^2}{2}$(voir \citet{belleflamme_peitz_2015}, chapitre 2).\\ Ici à l'équilibre 
        pour $Q^*_{(N=10)} = 145.5$,
         ce surplus est $SC_{(N=10)}^* = SC(Q_{(N=10)}^*) = \frac{145.5^2}{2}  = 10585.1$.
        \end{enumerate}
        
        \item 3 firmes fusionnent et par conséquent $N=8$ désormais, d'où:
        \begin{enumerate}[$\star$]
            \item $Q^*_{N=8} = 142.2$, 
            \item $p^*_{(N=8)}=57.8$, 
            \item $q_{(i, N=8)}^* = 17.8$,  
            \item $\pi_{(i, N=8)}^*= 316$. 
            \item Ce dernier chiffre donnant un surplus des producteurs de 
            $SP^* = N\pi_{(i, N=8)}^*=2528$.
            \item On calcule aussi le surplus du consommateur  
            $SC_{(N=8)}^* = SC(Q_{(N=8)}^*) = \frac{142.3^2}{2} = 10110.4$.
            \framebreak
            \item Ces résultats illustrent la règle du 80\% puisque le profit de la firme fusionnée 
            est inférieur au profit total des trois firmes avant la fusion(soit $316$ 
            après fusion contre $3\times 211.6$). 
            \item Le profit des firmes ne participant pas à la fusion augmente. 
            Elles bénéficient indirectement de celle-ci.
            \item On note aussi que le surplus du consommateur diminue 
            avec la fusion tandis que celui des firmes augmente.
        \end{enumerate}

        \item 3 firmes fusionnent et deviennent leader: concurrence du type Stackelberg. 
        \begin{enumerate}[$\star$]
            \item Jeux séquentiel: le leader joue d'abord, les autres firmes suivent. 
            \item On utilise l'indice "$L$" pour le leader et $i$ pour les firmes qui suivent avec donc 
            ici $i=1, \ldots, 7$.
            \item La résolution  se fait par induction à rebours, c.à.d. selon la procédure suivante:
            \begin{enumerate}[(i)]
            \item d'abord le problème d'optimisation des firmes qui suivent étant donné le choix du leader(dernière étape du jeux),
            \item ensuite le problème d'optimisation du leader(première étape du jeux).
            \end{enumerate}
            \item Notons donc la quantité offerte par la firme leader(lc.à.d., issue des trois firmes qui ont fusionné) $q_L$. 
            \item \textbf{Dernière/2ème étape}: considérons une firme $i$ parmi les firmes qui n'ont pas fusionné, 
            et notons $q_i$ la quantité qu'elle offre. 
            \item La quantité totale offerte peut s'écrire:
            \begin{align*}
                Q&= \underbrace{q_i}_{\text{qté de $i$}} + \underbrace{q_L}_{\text{qté de $L$}} 
                + \underbrace{\sum_{j=1; j\neq i}^7 q_j}_{\substack{\text{qté des autres}\\ \text{firmes que $i$ et $L$}}}
                 =  q_i + q_L + \underbrace{q_{-j}}_{:=\sum_{j=1; j\neq i}^7 q_j}
            \end{align*}
            \item La demande peut s'écrire alors:
            \begin{align*}
                p(Q) &= 200 - ( q_{-i} + q_L + q_i ).
            \end{align*}
            \item Et le profit de la firme $i$ s'écrit:
            \begin{align*}
                \pi_i(q_i) &= \left[200 - (q_{-i} + q_L + q_i ) \right] - 40q_i = \left[160 - ( q_{-i} + q_L + q_i )\right]q_i.
            \end{align*}
            \item La c.p.o. associée à la maximisation du profit permet de définir la fonction
            de meilleure réponse de la firme $i$:
            \begin{align*}
             \frac{\partial \pi_i}{\partial q_i}(q_i^{*}) = 0 &\Rightarrow 
             q_i^{*} = q_i^{mr}(q_{-i}, q_L) = 80 - \frac{(q_{-i} + q_L)}{2}.
             \end{align*}
             \begin{enumerate}[$\star$]
                \item Notons $q_{-i}^{*}$ l'analogue à l'équilibre de $q_{-i}$, et $q_L^*$ la quantité offeret
               à l'équilibre par la firme leader(le trois ayant fusionné). 
               \item Pour simplifier utilisons le fait que les firmes qui suivent 
               sont supposées identiques(mêmes fonction de coût) et en conséquence 
               à l'équilibre elles sont caractérisés par les mêmes valeurs quant aux quantités offertes, profits, etc.  
               \item Par conséquent $q_{-i}^{*} = \sum_{j=1; j\neq i}^7 q_i^* = 6q_i^*$.
              \end{enumerate}
              \item En utilisant la meilleure réponse de $i$ on obtient alors:
              \begin{align*}
                  q_i^{*}= 80 - \frac{(6q_i^* + q_L^*)}{2} &\Rightarrow q_i^* = 20 -\frac{q_L^*}{8}.
                  \end{align*}

             \framebreak
             \item \textbf{1ère étape}:
             \item On considère le profit de la firme leader qui connaît les meilleures réponses des firmes 
             qui suivent obtenues à l'étape précédente.  
             \item Le profit de la firme leader s'écrit:
             \begin{align*}
                \pi_L(q_L) &= P(Q)-cq_L =  \left(200 -q_L - \sum_{i=1}^7 q_i\right)q_L - 40q_L 
                = \left(160 -q_L + \sum_{i=1}^7 q_i\right) q_L
             \end{align*}
             \item En particulier, à l'optimum ce profit peut s'écrire:
             \begin{enumerate}[$\star$]
                \item Peut s'écrire:
                \begin{align*}
                \pi_L(q_L^*) = \left(160 -q_L^* - \sum_{i=1}^7 \underbrace{q_i^*}_{20 -\frac{q_L^*}{8}}\right)q_L^* 
                &= \left(160 -q_L^* - 7\left(20 -\frac{q_L^*}{8}\right)\right)q_L^*\\
                &= \left(20-\frac{q_L^*}{8}\right)q_L^*\\
                &= 20q_L^* -\frac{q_L{^{*^2}}}{8}
               \end{align*}
               \item Et par définition de $q_L^*$ comme maximisant $\pi_L(\cdot)$, il doit vérifier(c.p.o.):  
               \begin{align*}
                \frac{\partial \pi_L}{\partial q_L}(q_L^*) = 0 &\Rightarrow q_L^* = 80,
               \end{align*}
               et on en déduit alors: 
               
               \item la quantité offerte par la firme $i$ à l'équilibre: 
               \begin{align*}
                q_i^* &= q_i^* = 20 -\frac{q_L^*}{8} = 10, \ \text{pour} \ i=1, \ldots, 7.
               \end{align*}
               \item La quantité totale offerte et le prix d'équilibre:
               \begin{align*}
                Q^* = 7\times q_i^* + q_L^* = 150 &\Rightarrow p^* = P(Q^*) = 200-Q^* = 50.
               \end{align*}
               \item Les profits des firmes à l'équilibre:
               \begin{align*}
               \pi_L^* = (p^* - 40)q_L^* = 800&, \ \pi_i^* = (p^* - 40)q_i^* = 100.
               \end{align*}
               \item  Les surplus des agents sur le marché(c.à.d., firmes et consommateurs):
               \begin{align*}
                   SP^* = \underbrace{7\times \pi_i^* + \pi_L^*}_{\text{surplus des firmes}} = 1500&, \ 
                   SC^* =  \underbrace{\frac{Q^{*^2}}{2}}_{\substack{\text{surplus des }\\
                   \text{consommateurs}}} = 11250.
               \end{align*}
            
             \end{enumerate}

            \end{enumerate}
            \end{enumerate}    
\end{frame}
    \begin{frame}[allowframebreaks]{Commentaires/Résumé}
    \begin{itemize}
        \item La fusion est considérée dans le cadre d'un modèle de Cournot avec des firmes
         symétriques quant à leurs caractéristiques(mêmes coûts), et une demande linéaire. 
        \item Le premier point de l'exercice est simplement une extension du duopole 
        de Cournot classique à $N>2$ firmes(ici $N = 10$) qui en est un cas particulier. 
        \item Le deuxième point illustre les conditions restrictives 
        d'un fusion profitable(aux firmes) dans ce type de modèle qui est résumé par la \textbf{règle du 80\%}. 
        \item Le dernier point introduit la possibilités de "synergies" entre 
        entreprises fusionnées laquelle se traduirait par l'acquisition d'un 
        statut de leader de la firme issue de la fusion(remarque: 
        sans qu'il ne soit dit d'où proviennent ces synergies)
    \end{itemize}
    \end{frame}
\begin{frame}[allowframebreaks]{References}
    \bibliographystyle{jpe}
    \bibliography{../Biblio}
    \end{frame}
    
    \end{document}
    