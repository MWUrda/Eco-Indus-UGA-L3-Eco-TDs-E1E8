%\documentclass[ignorenonframetext, compress, 9pt, xcolor=svgnames]{beamer} 
\documentclass[notes, ignorenonframetext, compress, 10pt, xcolor=svgnames, aspectratio=169]{beamer} 
\usepackage{pgfpages}
\usepackage{pdfpages}
% These slides also contain speaker notes. You can print just the slides,
% just the notes, or both, depending on the setting below. Comment out the want
% you want.
\setbeameroption{hide notes} % Only slide
%\setbeameroption{show only notes} % Only notes
%\setbeameroption{show notes on second screen=right} % Both
\usepackage{amsmath}
\usepackage{amsfonts}
\usepackage{amssymb}
\setbeamercolor{frametitle}{fg=MidnightBlue}

\setbeamercolor{sectionpage title}{bg=MidnightBlue}
\setbeamertemplate{frametitle}[default][center]
%\setbeamertemplate{headline}{\vskip2cm}
%\setbeamertemplate{frametitle}{\color{MidnightBlue}\centering\bfseries\insertframetitle\par\vskip-6pt}
\setbeamerfont{frametitle}{series=\bfseries}
\setbeamerfont{title}{series=\bfseries}
\setbeamerfont{sectionpage}{series=\bfseries}
%\setbeamercolor{section in head/foot}{bg=MidnightBlueBlue}
%\setbeamercolor{author in head/foot}{bg=DarkBlue}
\setbeamercolor{author in head/foot}{fg=MidnightBlue}
%\setbeamercolor{title in head/foot}{bg=White}
\setbeamercolor{title in head/foot}{fg=MidnightBlue}
\setbeamercolor{title}{fg=MidnightBlue}
%\setbeamercolor{date in head/foot}{fg=Brown}
%\setbeamercolor{alerted text}{fg=DarkBlue}
%\usecolortheme[named=DarkBlue]{structure} 
%\usepackage{bbm}
%\usepackage{bbold}
\usepackage{eurosym}
\usepackage{graphicx}
%\usepackage{epstopdf}
\usepackage{hyperref}
\hypersetup{
  colorlinks   = true, %Colours links instead of ugly boxes
  urlcolor     = gray, %Colour for external hyperlinks
  linkcolor    = MidnightBlue, %Colour of internal links
  citecolor   = DarkRed %Colour of citations
}
\usepackage{multirow}
\usepackage{xspace}
\usepackage{listings}
\usepackage{natbib}
%\usepackage[sort&compress,comma,super]{natbib}
\def\newblock{} % To avoid a compilation error about a function \newblock undefined
\usepackage{bibentry}
\usepackage{booktabs}
\usepackage{dcolumn}
\usepackage[greek,frenchb]{babel}
\usepackage[babel=true,kerning=true]{microtype}
\usepackage[utf8]{inputenc}
\usepackage[T1]{fontenc}
\usepackage{natbib}
\renewcommand{\cite}{\citet}
\usepackage{longtable}
\usepackage{eso-pic}

\usepackage{xcolor}
 \colorlet{linkequation}{DarkRed} 
 \newcommand*{\SavedEqref}{}
 \let\SavedEqref\eqref 
\renewcommand*{\eqref}[1]{%
\begingroup \hypersetup{
      linkcolor=linkequation,
linkbordercolor=linkequation, }%
\SavedEqref{#1}%
 \endgroup
}

\newcommand*{\refeq}[1]{%
 \begingroup
\hypersetup{ 
linkcolor=linkequation, 
linkbordercolor=linkequation,
}%
\ref{#1}%
 \endgroup
}

\setbeamertemplate{caption}[numbered]
\setbeamertemplate{theorem}[ams style]
\setbeamertemplate{theorems}[numbered]
%\usefonttheme{serif}
%\usecolortheme{beaver}
%\usetheme{Hannover}
%\usetheme{CambridgeUS}
%\usetheme{Madrid}
%\usecolortheme{whale}
%\usetheme{Warsaw}
%\usetheme{Luebeck}
%\usetheme{Montpellier}
%\usetheme{Berlin}
%\setbeamercolor{titlelike}{parent=structure}
%\setbeamertemplate{headline}[default]
%\setbeamertemplate{footline}[default]
%\setbeamertemplate{footline}[Malmoe]
%\setbeamercovered{transparent}
%\setbeamercovered{invisible}
%\usecolortheme{crane}
%\usecolortheme{dolphin}
%\usepackage{pxfonts}
%\usepackage{isomath}
%\usepackage{mathpazo}
%\usepackage{arev} %     (Arev/Vera Sans)
%\usepackage{eulervm} %_   (Euler Math)
%\usepackage{fixmath} %  (Computer Modern)
%\usepackage{hvmath} %_   (HV-Math/Helvetica)
%\usepackage{tmmath} %_   (TM-Math/Times)
%\usepackage{tgheros}
%\usepackage{cmbright}
%\usepackage{ccfonts} \usepackage[T1]{fontenc}
%\usepackage[garamond]{mathdesign}

%\usepackage{color}
%\usepackage{ulem}

%\usepackage[math]{kurier}
%\usepackage[no-math]{fontspec}
%\setmainfont{Fontin Sans}
%\setsansfont{Fontin Sans}
%\setbeamerfont{frametitle}{size=\LARGE,series=\bfseries}
%%%add 19022021
\usepackage{enumerate}    
\usepackage{dcolumn}
\usepackage{verbatim}
\newcolumntype{d}[0]{D{.}{.}{5}}
%\setbeamertemplate{note page}{\pagecolor{yellow!5}\insertnote}
%\usetikzlibrary{positioning}
%\usetikzlibrary{snakes}
%\usetikzlibrary{calc}
%\usetikzlibrary{arrows}
%\usetikzlibrary{decorations.markings}
%\usetikzlibrary{shapes.misc}
%\usetikzlibrary{matrix,shapes,arrows,fit,tikzmark}
%%%
% suppress navigation bar
\beamertemplatenavigationsymbolsempty
%\usetheme{bunsenMod}
%\setbeamercovered{transparent}
%\setbeamertemplate{items}[circle]
%\usecolortheme[named=CadetBlue]{structure}
%\usecolortheme[RGB={225,64,5}]{structure}
%\definecolor{burntRed}{RGB}{225,64,5}
%\setbeamercolor{alerted text}{fg=burntRed} 
%\usecolortheme[RGB={0,40,110}]{structure}
%\hypersetup{linkcolor=burntRed}
%\hypersetup{urlcolor=burntRed}
%\hypersetup{filecolor=burntRed}
%\hypersetup{citecolor=burntRed}

%\usetheme{bunsenMod}
%\setbeamercovered{transparent}
%\setbeamertemplate{items}[circle]
%\usecolortheme[named=CadetBlue]{structure}
%\usecolortheme[RGB={225,64,5}]{structure}
%\definecolor{burntRed}{RGB}{225,64,5}
%\setbeamercolor{alerted text}{fg=burntRed} 
%\usecolortheme[RGB={0,40,110}]{structure}
%\hypersetup{linkcolor=burntRed}
%\hypersetup{urlcolor=burntRed}
%\hypersetup{filecolor=burntRed}
%\hypersetup{citecolor=burntRed}

%\AtBeginSection[] % Do nothing for \section*
%{ \frame{\sectionpage} }
%\setbeamertemplate{frametitle continuation}{}
\newtheorem{lemme}{Lemme}[section]
%\newtheorem{remarque}{Remarque}
\newcommand{\argmax}{\operatornamewithlimits{arg\,max}}
\newcommand{\argmin}{\operatornamewithlimits{arg\,min}}
\def\inprobLOW{\rightarrow_p}
\def\inprobHIGH{\,{\buildrel p \over \rightarrow}\,} 
\def\inprob{\,{\inprobHIGH}\,} 
\def\indist{\,{\buildrel d \over \rightarrow}\,} 
\def\sima{\,{\buildrel a \over \sim}\,} 
\def\F{\mathbb{F}}
\def\R{\mathbb{R}}
\def\N{\mathbb{N}}
\newcommand{\gmatrix}[1]{\begin{pmatrix} {#1}_{11} & \cdots &
    {#1}_{1n} \\ \vdots & \ddots & \vdots \\ {#1}_{m1} & \cdots &
    {#1}_{mn} \end{pmatrix}}
\newcommand{\iprod}[2]{\left\langle {#1} , {#2} \right\rangle}
\newcommand{\norm}[1]{\left\Vert {#1} \right\Vert}
\newcommand{\abs}[1]{\left\vert {#1} \right\vert}
\renewcommand{\det}{\mathrm{det}}
\newcommand{\rank}{\mathrm{rank}}
\newcommand{\spn}{\mathrm{span}}
\newcommand{\row}{\mathrm{Row}}
\newcommand{\col}{\mathrm{Col}}
\renewcommand{\dim}{\mathrm{dim}}
\newcommand{\prefeq}{\succeq}
\newcommand{\pref}{\succ}
\newcommand{\seq}[1]{\{{#1}_n \}_{n=1}^\infty }
\renewcommand{\to}{{\rightarrow}}
\renewcommand{\L}{{\mathcal{L}}}
\newcommand{\Er}{\mathrm{E}}
\renewcommand{\Pr}{\mathrm{P}}
%\newcommand{\Var}{\mathrm{Var}}
%\newcommand{\Cov}{\mathrm{Cov}}
%\newcommand{\corr}{\mathrm{Corr}}
%\newcommand{\Var}{\mathrm{Var}}
\newcommand{\bias}{\mathrm{Bias}}
\newcommand{\mse}{\mathrm{MSE}}
\providecommand{\Pred}{\mathcal{P}}
\providecommand{\plim}{\operatornamewithlimits{plim}}
\providecommand{\avg}{\frac{1}{n} \underset{i=1}{\overset{n}{\sum}}}
\providecommand{\sumin}{{\sum_{i=1}^n}}
\providecommand{\sumiN}{{\sum_{i=1}^N}}
\providecommand{\sumtT}{{\sum_{t=1}^T}}
\providecommand{\limp}{\overset{p}{\rightarrow}}
\providecommand{\liml}{\overset{L}{\rightarrow}}
%\providecommand{\limp}{\underset{n \rightarrow \infty}{\overset{p}{\longrightarrow}}}
%\providecommand{\limp}{\underset{n \rightarrow \infty}{\overset{p}{\longrightarrow}}}
%\providecommand{\limp}{\overset{p}{\longrightarrow}}
%\providecommand{\limd}{\underset{n \rightarrow \infty}{\overset{d}{\longrightarrow}}}
\providecommand{\limd}{\overset{d}{\rightarrow}}
\providecommand{\limps}{\overset{p.s.}{\rightarrow}}
\providecommand{\limlp}{\overset{L^p}{\rightarrow}}
\providecommand{\limprob}{\overset{p}{\underset{N\to +\infty}{\longrightarrow}}}
\providecommand{\limloi}{\overset{L}{\underset{N\to +\infty}{\longrightarrow}}}
\providecommand{\limpsure}{\overset{p.s.}{\underset{N\to +\infty}{\longrightarrow}}}
\def\independenT#1#2{\mathrel{\setbox0\hbox{$#1#2$}%
    \copy0\kern-\wd0\mkern4mu\box0}} 
\newcommand\indep{\protect\mathpalette{\protect\independenT}{\perp}}


\lstset{language=R}
\lstset{keywordstyle=\color[rgb]{0,0,1},                                        % keywords
        commentstyle=\color[rgb]{0.133,0.545,0.133},    % comments
        stringstyle=\color[rgb]{0.627,0.126,0.941}      % strings
}       
\lstset{
  showstringspaces=false,       % not emphasize spaces in strings 
  columns=fixed,
  numbersep=3mm, numbers=left, numberstyle=\tiny,       % number style
  frame=none,
  framexleftmargin=5mm, xleftmargin=5mm         % tweak margins
}
\makeatletter
%\setbeamertemplate{frametitle continuation}{\gdef\beamer@frametitle{}}
\setbeamertemplate{frametitle continuation}{\frametitle{}}
%\setbeamertemplate{frametitle continuation}{\insertcontinuationcount}
\makeatother

\theoremstyle{remark}
\newtheorem{interpretation}{Interprétation}
\newtheorem*{interpretation*}{Interprétation}

\theoremstyle{remark}
\newtheorem{remarque}{Remarque}%[section]
\newtheorem*{remarque*}{Remarque}
\usepackage[framemethod=TikZ]{mdframed} 
\usepackage{showexpl}
%\newtheorem{step}{Step}[section]
%\newtheorem{rem}{Comment}[section]
%\newtheorem{ex}{Example}[section]
%\newtheorem{hist}{History}[section]
%\newtheorem*{ex*}{Example}
%\theoremstyle{plain}
%\newtheorem{propriete}{Propri\'et\'e}
%\renewcommand{\thepropriete}{P\arabic{propriete}}
%\theoremstyle{definition}
%\newtheorem{definition}{Définition}%[section]
%\theoremstyle{remark}
%\newtheorem{exemple}{Exemple}
%\newtheorem*{exemple*}{Exemple}

\newtheorem{theoreme}{Théorème}
\newtheorem{proposition}{Proposition}
%\newtheorem{propriete}{Propri\'et\'e}
\newtheorem{corollaire}{Corollaire}
%\newtheorem{exemple}{Exemple}
%\newtheorem{assumption}{Assumption}
%\renewcommand{\theassumption}{A\arabic{assumption}}
\newtheorem{hypothese}{Hypothèse}
\renewcommand{\thehypothese}{H\arabic{hypothese}}
%\theoremstyle{definition}

%\newtheorem{definitionx}{D\'efinition}%[section]
%\newenvironment{definition}
 %{\pushQED{\qed}\renewcommand{\qedsymbol}{$\triangle$}\definitionx}
 %{\popQED\enddefinitionx}

%\newtheorem{condition}{Condition}
%\renewcommand{\thecondition}{C\arabic{condition}}
%\newcommand{\Var}{\mathbb{V}}
%\newcommand{\Var}{\mathbf{Var}}
%\newcommand{\Exp}{\mathbf{E}}
%\providecommand{\Vr}{\mathrm{Var}}
%\renewcommand{\Er}{\mathbb{E}}
%\newcommand{\LP}{\mathcal{LP}}
%\providecommand{\Id}{\mathbf{I}}
%\providecommand{\Rang}{\mathrm{Rang}}
%\providecommand{\Trace}{\mathrm{Trace}}
%\newcommand{\Cov}{\mathbf{Cov}}
%\newcommand{\Cov}{\mathbb{C}\mathrm{ov}}
\providecommand{\Id}{\mathbf{I}}
\providecommand{\Ind}{\mathbf{1}}
\providecommand{\uvec}{\mathbf{1}}
\providecommand{\vecOnes}{\mathbf{1}}
\DeclareMathOperator{\indfun}{\mathbf{1}}
\DeclareMathOperator{\Exp}{E}
\DeclareMathOperator{\Expn}{\mathbb{E}_n}
\DeclareMathOperator{\EL}{EL}
\DeclareMathOperator{\Var}{Var}
\DeclareMathOperator{\Vr}{V}
\newcommand{\boldVr}{ {\boldsymbol \Vr} }
\DeclareMathOperator{\Cov}{Cov}
\DeclareMathOperator{\corr}{corr}
\DeclareMathOperator{\perps}{\perp_s}
%\DeclareMathOperator{\Prob}{Pr}
\DeclareMathOperator{\Prob}{P}
\DeclareMathOperator{\prob}{p}
\DeclareMathOperator{\loss}{L}
\providecommand{\Corr}{\mathrm{Corr}}
\providecommand{\Diag}{\mathrm{Diag}}
\providecommand{\reg}{\mathrm{r}}
\providecommand{\Likelihood}{\mathrm{L}}
\renewcommand{\Pr}{{\mathbb{P}}}
\providecommand{\set}[1]{\left\{#1\right\}}
\providecommand{\uvec}{\mathbf{1}}
\providecommand{\Rang}{\mathrm{Rang}}
\providecommand{\Trace}{\mathrm{Trace}}
\providecommand{\Tr}{\mathrm{Tr}}
\providecommand{\CI}{\mathrm{CI}}
\providecommand{\asyvar}{\mathrm{AsyVar}}
\DeclareMathOperator{\Supp}{Supp}
\newcommand{\inputslide}[2]{{
    \usebackgroundtemplate{
     \includegraphics[page={#2},width=0.90\textwidth,keepaspectratio=true]
      %\includegraphics[page={#2},width=\paperwidth,keepaspectratio=true]
      {{#1}}}
    \frame[plain]{}
  }}
\newcommand\pperp{\perp\!\!\!\perp}
\newcommand\independent{\protect\mathpalette{\protect\independenT}{\perp}}
\def\independenT#1#2{\mathrel{\rlap{$#1#2$}\mkern2mu{#1#2}}}
\usepackage{bbm}
\providecommand{\Ind}{\mathbf{1}}
\newcommand{\sumjsi}{\underset{i<j}{{\sum}}}
\newcommand{\prodjsi}{\underset{i<j}{{\prod}}}
\newcommand{\sumisj}{\underset{j<i}{{\sum}}}
\newcommand{\prodisj}{\underset{j<i}{{\prod}}}
\newcommand{\sumobs}{\underset{i=1}{\overset{n}{\sum}}}
\newcommand{\sumi}{\underset{i=1}{\overset{n}{\sum}}}
\newcommand{\prodi}{\underset{i=1}{\overset{n}{\prod}}}
\newcommand{\prodobs}{\underset{i=1}{\overset{n}{\prod}}}
\newcommand{\simiid}{{\overset{i.i.d.}{\sim}}}
%\newcommand{\sumobs}{\sum_{i=1}^N}
%\newcommand{\prodobs}{\prod_{i=1}^N}
%\newcommand{\sumjsi}{\sum_{i<j}}
%\newcommand{\prodjsi}{\prod_{i<j}}
%\newcommand{\sumisj}{\sum_{j<i}}
%\newcommand{\prodisj}{\sum_{j<i}}

%\usepackage{appendixnumberbeamer}
\setbeamertemplate{footline}[frame number]
\setbeamertemplate{section in toc}[sections numbered]
\setbeamertemplate{subsection in toc}[subsections numbered]
\setbeamertemplate{subsubsection in toc}[subsubsections numbered]

%\makeatother
%\setbeamertemplate{footline}
%{
%    \leavevmode%
%    \hbox{%
%        \begin{beamercolorbox}[wd=.333333\paperwidth,ht=2.25ex,dp=1ex,center]{author in head/foot}%
%            \usebeamerfont{author in head/foot}\insertshortauthor
%        \end{beamercolorbox}%
%        \begin{beamercolorbox}[wd=.333333\paperwidth,ht=2.25ex,dp=1ex,center]{title in head/foot}%
%            \usebeamerfont{title in head/foot}\insertshorttitle
%        \end{beamercolorbox}%
%        \begin{beamercolorbox}[wd=.333333\paperwidth,ht=2.25ex,dp=1ex,right]{date in head/foot}%
%            \usebeamerfont{date in head/foot}\insertshortdate{}\hspace*{2em}
%            \insertframenumber{} / \inserttotalframenumber\hspace*{2ex} 
%        \end{beamercolorbox}}%
%       \vskip0pt%
 %   }
%   \makeatother
%\setbeamertemplate{navigation symbols}{}
\setbeamertemplate{itemize items}[ball]
%\setbeamertemplate{itemize items}{-}
%\newenvironment{wideitemize}{\itemize\addtolength{\itemsep}{10pt}}{\enditemize}
% \usepackage{eso-pic}
%\newcommand\AtPagemyUpperLeft[1]{\AtPageLowerLeft{%
%\put(\LenToUnit{0.9\paperwidth},\LenToUnit{0.9\paperheight}){#1}}}
%\AddToShipoutPictureFG{
%  \AtPagemyUpperLeft{{\includegraphics[width=1.1cm,keepaspectratio]{../logo-uga.png}}}
%}%
\def\figheight{3in}
\def\figwidth{4in}

%%Commands from Econometric Theory(Slides) by J. Stachurski.

\newcommand{\boldx}{ {\mathbf x} }
\newcommand{\boldu}{ {\mathbf u} }
\newcommand{\boldv}{ {\mathbf v} }
\newcommand{\boldw}{ {\mathbf w} }
\newcommand{\boldy}{ {\mathbf y} }
\newcommand{\boldb}{ {\mathbf b} }
\newcommand{\bolda}{ {\mathbf a} }
\newcommand{\boldc}{ {\mathbf c} }
\newcommand{\boldd}{ {\mathbf d} }
\newcommand{\boldi}{ {\mathbf i} }
\newcommand{\bolde}{ {\mathbf e} }
\newcommand{\boldp}{ {\mathbf p} }
\newcommand{\boldq}{ {\mathbf q} }
\newcommand{\bolds}{ {\mathbf s} }
\newcommand{\boldt}{ {\mathbf t} }
\newcommand{\boldz}{ {\mathbf z} }
\newcommand{\boldr}{ {\mathbf r} }
\newcommand{\boldm}{ {\mathbf m} }

\newcommand{\boldzero}{ {\mathbf 0} }
\newcommand{\boldone}{ {\mathbf 1} }

\newcommand{\boldalpha}{ {\boldsymbol \alpha} }
\newcommand{\boldbeta}{ {\boldsymbol \beta} }
\newcommand{\boldgamma}{ {\boldsymbol \gamma} }
\newcommand{\boldGamma}{ {\boldsymbol \Gamma} }
\newcommand{\boldtheta}{ {\boldsymbol \theta} }
\newcommand{\boldxi}{ {\boldsymbol \xi} }
\newcommand{\boldtau}{ {\boldsymbol \tau} }
\newcommand{\boldepsilon}{ {\boldsymbol \epsilon} }
\newcommand{\boldvepsilon}{ {\boldsymbol \varepsilon} }
\newcommand{\boldmu}{ {\boldsymbol \mu} }
\newcommand{\boldSigma}{ {\boldsymbol \Sigma} }
\newcommand{\boldOmega}{ {\boldsymbol \Omega} }
\newcommand{\boldPhi}{ {\boldsymbol \Phi} }
\newcommand{\boldLambda}{ {\boldsymbol \Lambda} }
\newcommand{\boldphi}{ {\boldsymbol \phi} }
\newcommand{\boldeta}{ {\boldsymbol \eta} }

\newcommand{\Sigmax}{ {\boldsymbol \Sigma_{\boldx}}}
\newcommand{\Sigmau}{ {\boldsymbol \Sigma_{\boldu}}}
\newcommand{\Sigmaxinv}{ {\boldsymbol \Sigma_{\boldx}^{-1}}}
\newcommand{\Sigmav}{ {\boldsymbol \Sigma_{\boldv \boldv}}}

\newcommand{\hboldx}{ \hat {\mathbf x} }
\newcommand{\hboldy}{ \hat {\mathbf y} }
\newcommand{\hboldb}{ \hat {\mathbf b} }
\newcommand{\hboldu}{ \hat {\mathbf u} }
\newcommand{\hboldtheta}{ \hat {\boldsymbol \theta} }
\newcommand{\hboldtau}{ \hat {\boldsymbol \tau} }
\newcommand{\hboldmu}{ \hat {\boldsymbol \mu} }
\newcommand{\hboldbeta}{ \hat {\boldsymbol \beta} }
\newcommand{\hboldgamma}{ \hat {\boldsymbol \gamma} }
\newcommand{\hboldSigma}{ \hat {\boldsymbol \Sigma} }

\newcommand{\boldA}{\mathbf A}
\newcommand{\boldB}{\mathbf B}
\newcommand{\boldC}{\mathbf C}
\newcommand{\boldD}{\mathbf D}
\newcommand{\boldI}{\mathbf I}
\newcommand{\boldL}{\mathbf L}
\newcommand{\boldM}{\mathbf M}
\newcommand{\boldP}{\mathbf P}
\newcommand{\boldQ}{\mathbf Q}
\newcommand{\boldR}{\mathbf R}
\newcommand{\boldX}{\mathbf X}
\newcommand{\boldU}{\mathbf U}
\newcommand{\boldV}{\mathbf V}
\newcommand{\boldW}{\mathbf W}
\newcommand{\boldY}{\mathbf Y}
\newcommand{\boldZ}{\mathbf Z}

\newcommand{\bSigmaX}{ {\boldsymbol \Sigma_{\hboldbeta}} }
\newcommand{\hbSigmaX}{ \mathbf{\hat \Sigma_{\hboldbeta}} }
\newcommand{\betahat}{\hat{\beta}}
\newcommand{\gammahat}{\hat{\gamma}}
\newcommand{\Uhat}{\hat{U}}
\newcommand{\Vhat}{\hat{V}}
\newcommand{\epsilonhat}{\hat{\epsilon}}
\newcommand{\sigmahat}{\hat{\sigma}}
\newcommand{\Sigmahat}{\hat{\Sigma}}
\newcommand{\Gammahat}{\hat{\Gamma}}

\newcommand{\RR}{\mathbbm R}
\newcommand{\CC}{\mathbbm C}
\newcommand{\NN}{\mathbbm N}
\newcommand{\PP}{\mathbbm P}
\newcommand{\EE}{\mathbbm E \nobreak\hspace{.1em}}
\newcommand{\EEP}{\mathbbm E_P \nobreak\hspace{.1em}}
\newcommand{\ZZ}{\mathbbm Z}
\newcommand{\QQ}{\mathbbm Q}


\newcommand{\XX}{\mathbbm X}

\newcommand{\aA}{\mathcal A}
\newcommand{\fF}{\mathscr F}
\newcommand{\bB}{\mathscr B}
\newcommand{\iI}{\mathscr I}
\newcommand{\rR}{\mathscr R}
\newcommand{\dD}{\mathcal D}
\newcommand{\lL}{\mathcal L}
\newcommand{\llL}{\mathcal{H}_{\ell}}
\newcommand{\gG}{\mathcal G}
\newcommand{\hH}{\mathcal H}
\newcommand{\nN}{\textrm{\sc n}}
\newcommand{\lN}{\textrm{\sc ln}}
\newcommand{\pP}{\mathscr P}
\newcommand{\qQ}{\mathscr Q}
\newcommand{\xX}{\mathcal X}
\newcommand{\yY}{\mathcal Y}
\newcommand{\ddD}{\mathscr D}


%\newcommand{\R}{{\texttt R}}
\newcommand{\risk}{\mathcal R}
\newcommand{\Remp}{R_{{\rm emp}}}

\newcommand*\diff{\mathop{}\!\mathrm{d}}
\newcommand{\ess}{ \textrm{{\sc ess}} }
\newcommand{\tss}{ \textrm{{\sc tss}} }
\newcommand{\rss}{ \textrm{{\sc rss}} }
\newcommand{\rssr}{ \textrm{{\sc rssr}} }
\newcommand{\ussr}{ \textrm{{\sc ussr}} }
\newcommand{\zdata}{\mathbf{z}_{\mathcal D}}
\newcommand{\Pdata}{P_{\mathcal D}}
\newcommand{\Pdatatheta}{P^{\mathcal D}_{\theta}}
\newcommand{\Zdata}{Z_{\mathcal D}}


\newcommand{\e}[1]{\mathbbm{E}[{#1}]}
\newcommand{\p}[1]{\mathbbm{P}({#1})}

% condition
\theoremstyle{definition}
\newtheorem{condition}{Condition}
\renewcommand{\thecondition}{C\arabic{condition}}
\BeforeBeginEnvironment{condition}{
  \setbeamerfont{block title}{series=\bfseries}
  \setbeamercolor{block title}{fg=MidnightBlue,bg=white}
  \setbeamercolor{block body}{fg=black, bg=gray!10}
}
\newtheorem*{condition*}{Condition}
\BeforeBeginEnvironment{condition*}{
  \setbeamerfont{block title}{series=\bfseries}
  \setbeamercolor{block title}{fg=MidnightBlue,bg=white}
  \setbeamercolor{block body}{fg=black, bg=gray!10}
}

% assumption
\theoremstyle{definition}
\newtheorem{assumption}{Assumption}
\BeforeBeginEnvironment{assumption}{
  \setbeamerfont{block title}{series=\bfseries}
  \setbeamercolor{block title}{fg=MidnightBlue,bg=white}
  \setbeamercolor{block body}{fg=black, bg=gray!10}
}
\newtheorem*{assumption*}{Assumption}
\BeforeBeginEnvironment{assumption*}{
  \setbeamerfont{block title}{series=\bfseries}
  \setbeamercolor{block title}{fg=MidnightBlue,bg=white}
  \setbeamercolor{block body}{fg=black, bg=gray!10}
}

% definition
\BeforeBeginEnvironment{definition}{
  \setbeamerfont{block title}{series=\bfseries}
  \setbeamercolor{block title}{fg=MidnightBlue,bg=white}
  \setbeamercolor{block body}{fg=black, bg=gray!10}
}
\newtheorem*{definition*}{Definition}
\BeforeBeginEnvironment{definition*}{
  \setbeamerfont{block title}{series=\bfseries}
  \setbeamercolor{block title}{fg=MidnightBlue,bg=white}
  \setbeamercolor{block body}{fg=black, bg=gray!10}
}

% theorem
\theoremstyle{plain}
\BeforeBeginEnvironment{theorem}{
  \setbeamerfont{block body}{shape=\itshape}
  \setbeamerfont{block title}{series=\bfseries}
  \setbeamercolor{block title}{fg=MidnightBlue,bg=white}
  \setbeamercolor{block body}{fg=black, bg=gray!10}
}
\newtheorem*{theorem*}{Theorem}
\BeforeBeginEnvironment{theorem*}{
  \setbeamerfont{block body }{shape=\itshape}
  \setbeamerfont{block title}{series=\bfseries}
  \setbeamercolor{block title}{fg=MidnightBlue,bg=white}
  \setbeamercolor{block body}{fg=black, bg=gray!10}
}

% definition_fr
\theoremstyle{definition}
\newtheorem{definition_fr}{Définition}%[section]
\BeforeBeginEnvironment{definition_fr}{
  \setbeamerfont{block title}{series=\bfseries}
  \setbeamercolor{block title}{fg=MidnightBlue,bg=white}
  \setbeamercolor{block body}{fg=black, bg=gray!10}
}
\newtheorem*{definition_fr*}{Définition}
\BeforeBeginEnvironment{definition_fr*}{
  \setbeamerfont{block title}{series=\bfseries}
  \setbeamercolor{block title}{fg=MidnightBlue,bg=white}
  \setbeamercolor{block body}{fg=black, bg=gray!10}
}
% theorem_fr
\newtheorem{theorem_fr}{Théorème}%[section]
\BeforeBeginEnvironment{theorem_fr}{
  \setbeamerfont{block body}{shape=\itshape}
  \setbeamerfont{block title}{series=\bfseries, shape = \upshape}
  \setbeamercolor{block title}{fg=MidnightBlue,bg=white}
  \setbeamercolor{block body}{fg=black, bg=gray!10}
}
\newtheorem*{theorem_fr*}{Théorème}
\BeforeBeginEnvironment{theorem_fr*}{
  \setbeamerfont{block body}{shape=\itshape}
  \setbeamerfont{block title}{series=\bfseries, shape = \upshape}
  \setbeamercolor{block title}{fg=MidnightBlue,bg=white}
  \setbeamercolor{block body}{fg=black, bg=gray!10}
}

% remark_fr
\theoremstyle{remark}
\newtheorem{remark_fr}{Remarque}%[section]
\BeforeBeginEnvironment{remark_fr}{
  \setbeamerfont{block title}{series=\bfseries, shape=\itshape}
  \setbeamercolor{block title}{fg=MidnightBlue,bg=white}
  \setbeamercolor{block body}{fg=black, bg=gray!10}
}
\newtheorem*{remark_fr*}{Remarque}
\BeforeBeginEnvironment{remark_fr*}{
  \setbeamerfont{block title}{series=\bfseries, shape=\itshape}
  \setbeamercolor{block title}{fg=MidnightBlue,bg=white}
  \setbeamercolor{block body}{fg=black, bg=gray!10}
}

% exemple
\theoremstyle{remark}
\newtheorem{exemple}{Exemple}%[section]
\BeforeBeginEnvironment{exemple}{
  \setbeamerfont{block title}{series=\bfseries, shape=\itshape}
  \setbeamercolor{block title}{fg=MidnightBlue,bg=white}
  \setbeamercolor{block body}{fg=black, bg=gray!10}
}
\newtheorem*{exemple*}{}
\BeforeBeginEnvironment{exemple*}{
  \setbeamerfont{block title}{series=\bfseries, shape=\itshape}
  \setbeamercolor{block title}{fg=MidnightBlue,bg=white}
  \setbeamercolor{block body}{fg=black, bg=gray!10}
}


% propriete
\theoremstyle{plain}
\newtheorem{propriete}{Propri\'et\'e}%[section]
\BeforeBeginEnvironment{propriete}{
  \setbeamerfont{block body}{shape=\itshape}
  \setbeamerfont{block title}{series=\bfseries, shape = \upshape}
  \setbeamercolor{block title}{fg=MidnightBlue,bg=white}
  \setbeamercolor{block body}{fg=black, bg=gray!10}
}
\newtheorem*{propriete*}{Propri\'et\'e}
\BeforeBeginEnvironment{propriete*}{
  \setbeamerfont{block body}{shape=\itshape}
  \setbeamerfont{block title}{series=\bfseries, shape = \upshape}
  \setbeamercolor{block title}{fg=MidnightBlue,bg=white}
  \setbeamercolor{block body}{fg=black, bg=gray!10}
}


% remark
\theoremstyle{remark}
\newtheorem{remark}{Remark}%[section]
\BeforeBeginEnvironment{remark}{
  \setbeamerfont{block body}{shape=\itshape}
  \setbeamerfont{block title}{series=\bfseries}
  \setbeamercolor{block title}{fg=MidnightBlue,bg=white}
  \setbeamercolor{block body}{fg=black, bg=gray!10}
}
\newtheorem*{remark*}{Remark}
\BeforeBeginEnvironment{remark*}{
  \setbeamerfont{block body }{shape=\itshape}
  \setbeamerfont{block title}{series=\bfseries}
  \setbeamercolor{block title}{fg=MidnightBlue,bg=white}
  \setbeamercolor{block body}{fg=black, bg=gray!10}
}


\usepackage{color}
\usepackage{tikz}


\usepackage{enumerate}   


%\setbeamersize{text margin left=1.5em,text margin right=1.5em} 
%\setbeamersize{text margin left=1.2cm,text margin right=1.2cm} 
\setbeamersize{text margin left=1.5em,text margin right=1.5em} 
%\usepackage{xr}
%\externaldocument{Econometrie1_UGA_P2e}
  \usepackage{eso-pic}
%\newcommand\AtPagemyUpperLeft[1]{\AtPageLowerLeft{%
%\put(\LenToUnit{0.9\paperwidth},\LenToUnit{0.85\paperheight}){#1}}}
%\AddToShipoutPictureFG{
 % \AtPagemyUpperLeft{{\includegraphics[width=1.1cm,keepaspectratio]{logoUGA2020.pdf}}}
%}%

%\setbeamercolor{title}{fg=black}
%\setbeamercolor{frametitle}{fg=black}
%\setbeamercolor{section in head/foot}{fg=black}
%\setbeamercolor{author in head/foot}{bg=Brown}
%\setbeamercolor{date in head/foot}{fg=Brown}
\setbeamertemplate{section page}
{
    \begin{centering}
    \begin{beamercolorbox}[sep=11pt,center]{part title}
    \usebeamerfont{section title}\thesection.~\insertsection\par
    \end{beamercolorbox}
    \end{centering}
}
%\titlegraphic{\includegraphics[width=1cm]{logoUGA2020.pdf}}
\title[]{ \textbf{Économie Industrielle} \\ (UGA, L3 Éco, S2) \\ (responsable du cours: Sylvain Rossiaud)}
\subtitle{Travaux dirigés: No 2\\ 
Barrières stratégiques à l'entrée\\(éléments de correction d'exercices)}
\date{\today}
\author{Michal W. Urdanivia\inst{*}}
\institute{\inst{*}UGA, Facult\'e d'\'Economie, GAEL, \\
e-mail:
 \href{
     mailto:michal.wong-urdanivia@univ-grenoble-alpes.fr}{michal.wong-urdanivia@univ-grenoble-alpes.fr}}

%\titlegraphic{\includegraphics[width=1cm]{logoUGA2020.pdf}
%}

\begin{document}
%%% TIKZ STUFF
\usetikzlibrary{positioning}
\usetikzlibrary{snakes}
\usetikzlibrary{calc}
\usetikzlibrary{arrows}
\usetikzlibrary{decorations.markings}
\usetikzlibrary{shapes.misc}
\usetikzlibrary{matrix,shapes,arrows,fit,tikzmark}
\usetikzlibrary{shapes}
\tikzset{   
        every picture/.style={remember picture,baseline},
        every node/.style={anchor=base,align=center,outer sep=1.5pt},
        every path/.style={thick},
        }
\newcommand\marktopleft[1]{
    \tikz[overlay,remember picture] 
        \node (marker-#1-a) at (-.3em,.3em) {};%
}
\newcommand\markbottomright[2]{%
    \tikz[overlay,remember picture] 
        \node (marker-#1-b) at (0em,0em) {};%
}
\tikzstyle{every picture}+=[remember picture] 
\tikzstyle{mybox} =[draw=black, very thick, rectangle, inner sep=10pt, inner ysep=20pt]
\tikzstyle{fancytitle} =[draw=black,fill=red, text=white]
\tikzstyle{observed}=[draw,circle,fill=gray!50]



\begin{frame}
\titlepage
\end{frame}

%\begin{frame}
%\frametitle{Contenu}
%\tableofcontents[pausesections, pausesubsections]
%\end{frame}

%\section{Qu'est-ce que l’économétrie ? A quoi (à qui) ça sert ?}
%\frame{\sectionpage}
%\begin{frame}
%  \tableofcontents  
%\end{frame}

\section{Exercice 1}
\frame{\sectionpage}

\begin{frame}[allowframebreaks]{Modèle}
\begin{itemize}
\item Deux firmes produisent des biens homogènes où: 
\begin{enumerate}[-]
\item la firme 1 est leader:  elle fixe son niveau de production $q_1$,
\item la firme 2 est suiveuse : elle choisit son niveau de production $q_2$ après avoir observé $q_1$
\end{enumerate}
\item Fonction de demande inverse: 
\begin{align}
    P(Q) &= 300 - Q, \ Q=q_1+q_2.
\label{eq1}
\end{align}
\item Coûts fixes nuls, et coûts marginal constant:
\begin{align}
    c_i(q_i) &= 20q_i \Rightarrow c^{m}_i(q_i) :=\frac{\partial c_i(q_i)}{\partial q_i} = 20, \ \text{pour} \ i= 1, 2.
    \label{eq2}
\end{align}
\end{itemize}
\end{frame}

\begin{frame}[allowframebreaks]{(a) Fonction de meilleure réponse de la firme 2}
    \begin{itemize}
        \item Profit de $2$: 
        \begin{align}
            \pi_2(q_1, q_2) &= Pq_2 - c_2(q_2) = \underbrace{\left(300-(q_1+q_2)\right)}_{\text{en raison de \eqref{eq1}}}q_2 - 
            \underbrace{20q_2}_{\text{par \eqref{eq2}}},
            \label{eq3}
        \end{align}
        \item $2$ décide du niveau $q_2$ de sorte à maximiser $\pi_2$. Notons $q_2^*$ ce niveau qui est défini par:
        \begin{align*}
            q^*_2 &= \argmax_{q_2}  \left(300-(q_1+q_2)\right)q_2 - 20q_2,
        \end{align*}
         et vérifie la c.p.o.,
         \begin{align}
             \frac{\partial \pi_2(q_1, q_2^*)}{\partial q_2} = 0 \Leftrightarrow 280-q_1-2q^*_2 = 0,
             \label{eq4}
         \end{align}
        qui défini $q^*_2$ comme fonction de $q_1$, c.à.d., \textbf{\underline{la fonction de meilleure réponse de 2}}. 
        On la note $q^{mr}_2(q_1)$ et elle est donc donnée en utilisant \eqref{eq4} par,
        \begin{align}
            q^*_2 & = 140 - \frac{q_1}{2}=: q^{mr}_2(q_1)
           \label{eq5}
        \end{align}
    \end{itemize}

\end{frame}
    
\begin{frame}[allowframebreaks]{(b) Équilibre}
    \begin{enumerate}[(i)]
        \item \textbf{\underline{Quantités}}: 
        \begin{enumerate}[-]
            \item La fonction de profit de 1 présente une forme similaire à celle de 2:
            \begin{align}
                \pi_1\left(q_1, q_2\right) &= Pq_1 - c_1(q_1) = \left(300-(q_1+q_2)\right)q_1 - 20q_1.
                \label{eq6}
            \end{align}
            \item En outre 1 connaît le niveau de production que 2 fixe en fonction du sien, 
            donné par \eqref{eq5} et ce faisant \eqref{eq6} peut s'écrire:
            \begin{align}
                \pi_1\left(q_1, q^{mr}_2(q_1)\right) &= \left(300-(q_1+q_2)\right)q_1 - 20q_1 \nonumber\\
                &= 280q_1 - q_1^2 - q^{mr}_2(q_1)q_1 \nonumber \\
                &= 280q_1 - q_1^2 - \underbrace{\left(140 - \frac{q_1}{2}\right)}_{=q^{mr}_2(q_1)}q_1\nonumber\\
                &= 140 q_1 - \frac{q_1^2}{2} =: \pi(q_1),
                \label{eq6}
            \end{align}
            qui ne dépends que de $q_1$. 
            \item $1$ maximise donc \eqref{eq6}. Le maximum est atteint pour $q_1^*$ qui vérifie la c.p.o.:  
            \begin{align}
                \frac{\partial \pi_1(q_1^*)}{\partial q_1} =0 & \Leftrightarrow 
                140 - q_1^* =0 \Rightarrow q_1^* = 140.
                \label{eq7}
            \end{align}
            \item Il en résulte par \eqref{eq5}:
            \begin{align*}
                q^*_2 &= q^{mr}_2(q_1^*) = 140 - \frac{q_1^*}{2}=70.
            \end{align*}
            \item Ceci qui implique une quantité produite à l'équilibre de $Q^* = q_1^* + q_2^* = 210$, 
            un prix de $P^* = P(Q^*) = 90$, pour de profits $\pi_1(q_1^*) = 9800$, et 
            $\pi_2(q_2^*) = 4900$.
        \end{enumerate}
    \end{enumerate}
\end{frame}

\begin{frame}[allowframebreaks]{(c) Comparaison avec Cournot}
\begin{itemize} 
\item Pour rappel dans un Cournot de base(voir cours, et TD1) avec:
\begin{enumerate}[-]
\item des coûts $c_i(q_i) = cq_i$, $c>0$, $i=1, 2$,
\item et une demande $P(Q) = a - bQ$, $Q=q_1+q_2$, $a, b > 0$,
\end{enumerate}
l'équilibre est (où l'indice sup "ec" est pour "équilibre de Cournot"):
\begin{align*}
    q_1^{(ec)} = q_2^{(ec)} = q^{(ec)} = \frac{a-c}{3b},
\end{align*}
d'où ici(avec $a=300, b=1, c=20$): 
\begin{enumerate}[-]
\item $q_1^{(ec)}=q_2^{(ec)}=280/3 \approx 93.3$, 
\item $Q^{(ec)} = q_1^{(ec)}+q_2^{(ec)} \approx 186.7$,
\item $P^{(ec)} \approx 113.3$.
\item $\pi^{(ec)}_1 = \pi^{(ec)}_2 \approx 8704.9$.
\end{enumerate}

\item \textbf{\underline{Remarque}}: en Cournot les meilleures réponses sont pour $i=1, 2$:
\begin{align*}
    q_i^{mr}&=\frac{(a-b)}{2b} - \frac{q_j}{2}, \ j=1, 2, j\neq i.
\end{align*} 
\end{itemize}
\end{frame}


\section{Exercice 2}
\frame{\sectionpage}
\begin{frame}[allowframebreaks]{(a) Demande résiduelle}
\begin{itemize}
    \item Deux firmes aux coûts respectifs: 
    \begin{align}
     c_1(q_1) = 20q_1 \ (\text{firme 1}) 
     &,  \quad c_2(q_2) = 
     \underbrace{100}_{\substack{\text{part fixe/}\\\text{non récupérable}}} + 20q_2 \ \text{(firme 2)}
     \label{eq18}
    \end{align}
    \item \textbf{\underline{Remarque}}: la part fixe dans le coût de 2 donne un avantage 
    à 1 qui est en place par rapport à 2 qui est l'entrant potentiel.
    \item La demande sur le marché est donné par: 
    \begin{align}
        P&= P(Q) = 200 - \underbrace{(q_1 + q_2)}_{= Q}
        \label{eq19}
    \end{align}
    \item \textbf{\underline{Remarque}}: on change un peu les notations par rapport à l'énoncé où $q_1=Q$, et $q_2 = q$. Ici donc  
    $Q$ est la qté totale et $q_i$ celle produite par la firme $i=1, 2$.
    \item \eqref{eq19} permet d'avoir la demande(inverse) résiduelle qui s'adresse à 2 lorsque 1 produit par exemple $q_1=Q_0\geq 0$: $P=200-Q_0-q_2$.
\end{itemize}
\end{frame}
\begin{frame}[allowframebreaks]{(b) choix de 2}
    \begin{itemize}
    \item \textbf{\underline{Réponse de 2 pour $q_1$ donné}}:  
    \begin{enumerate}[-]
        \item Profit de 2:  
        \begin{align}
            \pi_2(q_2) &= \underbrace{Pq_2}_{=R_2(q_2)(\text{Recette})} -c_2(q_2) 
            = \left(\underbrace{200 - (q_1 + q_2)}_{=P(\text{par \eqref{eq19}})}\right)q_2 -
             \left( \underbrace{100 + 20q_2}_{=c_2(q_2)(\text{par \eqref{eq18}})}\right).
             \label{eq20}
        \end{align}
        \item Notons $q_2^*$ le quantité qui maximise \eqref{eq20}, 
        et peut être définie à partir de la c.p.o.,
        \begin{align}
            \frac{\partial \pi_2(q_2^*)}{\partial q_2} &=0\Leftrightarrow 180 - q_1-2q_2^{*}\Rightarrow 
            q_2^{*} = 90 - \frac{q_1}{2}=: q_2^{mr}(q_1).
            \label{eq21}
        \end{align}
        \item En particulier pour $q_1=Q_0$, le choix optimal de 2 sera $q_2^* = 90 - \frac{Q_0}{2}$.
    \end{enumerate}
\end{itemize}
\end{frame}

\begin{frame}[allowframebreaks]{c) Stratégie de "prix limite" de 1}
    \begin{itemize}
        \item Pour 1 la stratégie consiste à choisir de produire une quantité $q_1^L$ telle qu'elle dissuade 2 
        de décider d'entrer. 
        \item Ce sera le cas(puisque les agents maximisent leur profit) si le profit de 2 est alors nul.
        \item On a d'après \eqref{eq20}:
        \begin{align}
            \pi_2(q_2) &= (200 - (q_1 + q_2))q_2 - (100 + 20q_2) = (180 - q_1-q_2)q_2 - 100.
         \label{eq22}
        \end{align}
        \item D'après \eqref{eq22}:
        \begin{align}
            \pi_2(q_2) &= 0 \Leftrightarrow  (180 - q_1-q_2)q_2 - 100 = 0.
            \label{eq23}
        \end{align}
        \item Lorsque 1 choisit $q_1^L$ telle que \eqref{eq23} le niveau 
        que 2 décide est donné par $q_2^{mr}(q_1^L)$ en \eqref{eq21}. D'où:
        \begin{align*}
            \left(180-q_1^L - \left(\underbrace{90 - \frac{q_1^L}{2}}_{=q_2=:q_2^{mr}(q_1^L)}\right)\right)\left(
                \underbrace{90 - \frac{q_1^L}{2}}_{=q_2=:q_2^{mr}(q_1^L)}\right) - 100 = 0& \Leftrightarrow 
                \left(90-\frac{q_1^L}{2}\right)^2 -100=0\\
                &\Rightarrow q_1^L = 160.
        \end{align*}
        \item Il s'ensuit que $P^L = P(q_1^L) = 40$, $\pi_1(q_1^L) = 3200$.
        \item Lorsque 1 choisit $q_1^L$ elle ne maximise pas son profit 
        mais en dissuadant 2 d'entrer(car son profit est nul) elle peut 
        s'attendre à bénéficier de sa situation de monopole par la suite:
        \begin{enumerate}[-]
            \item En monopole 1 maximise
            \begin{align*}
                \pi_1(q_1) &= \underbrace{P(q_1)q_1}_{=:R_1(q_1)(\text{recette})} -c_1(q_1) = (200-q_1)q_1 - 20q_1.
            \end{align*}
            \item Notons $q_1^M$ le choix optimale de monople qui vérifie(c.p.o.): 
            \begin{align*}
                \frac{\partial \pi(q_1^M)}{\partial q_1} &= 0 \Leftrightarrow 180-2q_1^M = 0\Rightarrow q_1^M = 90,
            \end{align*}
            et alors $P^M=P(q_1^M) = 110$, et $\pi_1(q_1^M) = 8100 > \pi_1(q_1^L)$.
        \end{enumerate}
    \end{itemize}
    \end{frame}
\section{Exercice 3}
\frame{\sectionpage}
\begin{frame}[allowframebreaks]{Modèle}
\begin{itemize}
\item Une firme(la firme 1) produit en monopole un produit en quantité $q_1$.
\item Le demande est donnée par la fonction de demande inverse: 
\begin{align*}
    P(Q) &= 50 - \frac{Q}{10}, 
\end{align*}
où $Q = q_1$ dès lors que la firme est en monopole.
\item Et ce faisant sa recette peut s'écrire,
\begin{align}
    R_1(q_1) &= P(q_1)q_1 = 50q_1- \frac{q_1^2}{10} \Rightarrow R^m_1(q_1) := \frac{\partial R_1(q_1)}{\partial q_1} = 50-\frac{q_1}{5}
\label{eq8}
\end{align}
où $R^m_1(q_1)$ est la recette marginale.
\item Son coût est supposé:
\begin{align}
    c_1(q_1) &= \frac{q_1^2}{40} \Rightarrow c^m_1(q_1) = \frac{q_1}{20} > 0, \ \text{pour tout \ $q_1>0$(coût marginal croissant)}
\label{eq9}
\end{align}
\end{itemize}
    \end{frame}

    \begin{frame}[allowframebreaks]{(a) Équilibre de monopole}
        \begin{itemize}
            \item La firme maximise son profit qui ne dépend que de $q_1$:
            \item Profit de 1: 
            \begin{align}
                \pi_1(q_1) &=R_1(q_1) - c_1(q_1),
                \label{eq10}
            \end{align}
            et il est facile de voir que la quantité $q_1^*$ qui maximise \eqref{eq10} vérifie(c.p.o.)
            \begin{align}
                \frac{\partial \pi_1(q_1^*)}{\partial q_1} &= 0 \Leftrightarrow  R^m_1(q_1^*) = c^m_1(q_1^*),
                \label{eq11}
            \end{align}
            d'où par \eqref{eq8}, \eqref{eq9} et \eqref{eq10}:
            \begin{align*}
                 50-\frac{q_1^*}{5}=\frac{q_1^*}{20} \Rightarrow q_1^* = 200,
            \end{align*}
            le prix d'équilibre étant $P^* := P(q_1^*) = 30$.
        \end{itemize}
    \end{frame}   


\begin{frame}[allowframebreaks]{(b) Marché contestable(2ème firme)}
    \begin{itemize}
        \item Produire est plus coûteux pour 2 avec: 
        \begin{align}
            c_2(q_2) &= 10q_2 + \frac{q_2^2}{40} \Rightarrow c^m_2(q_2) = 10 + \frac{q_2}{20}.
            \label{eq12}
        \end{align}
        \item  \textbf{\underline{Demande résiduelle}} pour 2 quand 1 conserve le niveau de production $q_1= q_1^* = 200$. 
        \begin{enumerate}[-]
            \item La recette de 2 comme fonction de $q_1$ est:
            \begin{align}
                R_2(q_2) &=Pq_2= P(\underbrace{q_1^* + q_2}_{=Q})q_2 = 
                \left(50 - \frac{(q_1^* + q_2)}{10}\right)q_2 = \frac{(500-q_1)q_2}{10} -\frac{q_2^2}{10}
                \label{eq13}
            \end{align}
            \item Et son profit peut s'écrit:  
            \begin{align}
                \pi_2(q_2) &= R_2(q_2) - c_2(q_2) = \underbrace{\frac{(500-q_1)q_2}{10} -\frac{q_2^2}{10}}_{=R_2(q_2) \ \text{par \eqref{eq13}}}
                - \underbrace{(10q_2 + \frac{q_2^2}{40})}_{c_2(q_2),  \ \text{par \eqref{eq12}}}
                \label{eq14}
            \end{align}
            \item On note $q_2^*$ la quantité qui maximise \eqref{eq14} et vérifie donc(c.p.o.):  
            \begin{align}
                \frac{\partial \pi(q_2^*)}{\partial q_2} &=0 \Leftrightarrow \frac{(500-q_1)}{10} - 10 -  \frac{q_2^*}{4} = 0
                 \Rightarrow q_2^* = 160 - \frac{2q_1}{5}=: q_2^{mr}(q_1),
                 \label{eq15}
            \end{align}
            et pour $q_1 = q_1^*=200$, on obtient $q_2^* = 80$ d'où  $Q^* = q_1^* + q_2^* = 280$, $P^* := P(Q^*) = 22$.
        \end{enumerate}
    \end{itemize}
\end{frame}   

\begin{frame}[allowframebreaks]{(c) Stratégie de prix limite}
\begin{itemize}
\item 2 n'a que des coût variables et ce faisant $\pi_2 = 0 \Leftrightarrow q_2 = 0$.  
\item En utilisant  $q_2^{mr}(q_1)$ définie dans \eqref{eq15} on a: 
\begin{align*}
    q_2^{mr}(q_1) &= 0 \Leftrightarrow 160 - \frac{2q_1}{5} = 0 \Leftrightarrow q_{(1, q_2=0)} = 400=:q_1^L,
\end{align*}
où l'indice inf "$q_2=0$" est là pour indiquer que c'est le niveau de $q_1$ tel que $q_2=0$,  et 
qui définit ici le \textbf{\underline{la quantité associée au prix limite}}, c.à.d., pour lequel $\pi_2=0$, qu'on note $q_1^L$ ci-dessus. 
\item On note $Q^L = q_1^L + 0$ la quantité totale produite, et $P^L:=P(Q^L) = 10$, le prix limite.
\item On calcule aussi le profit obtenu par 1: 
\begin{align*}
    \pi_1(q_1^L)&=P^L q_1^L - c_1(q_1^L) = 0
\end{align*}
\end{itemize}
\end{frame}

\begin{frame}[allowframebreaks]{(d) Cournot}
    \begin{itemize}
        \item En concurrence à la Cournot le profit de 1 s'écrit: 
        \begin{align}
            \pi_1(q_1, q_2) &= P(\underbrace{Q}_{=q_1+ q_2})q_1 - c_1(q_1) = \left(50-\frac{(q_1+q_2)}{10}\right)q_1  - \frac{q_1^2}{40}   -
            =  50 q_1 -\frac{q_1q_2}{10} - \frac{q_1^2}{8},
            \label{eq16}
        \end{align}
        qu'on maximise pour obtenir $q_1^{*c}$ la quantité qui maximise le profit de 1 en Cournot. Elle vérifie(c.p.o.)
        \begin{align}
            \frac{\partial \pi_1(q_1^{*c}, q_2)}{\partial q_1} &= 0 \Leftrightarrow 50 - \frac{q_2}{10} -  
            \frac{q_1^{*c}}{4} = 0 \Rightarrow q_1^{*c} = 200-\frac{2 q_2}{5} =: q_1^{mr}(q_2).
            \label{eq17}
        \end{align}
        \item En utilise les meilleures réponses $q_1^{mr}(q_2)$ donnée ci-dessus et $q_2^{mr}(q_1)$ donnée en \eqref{eq15} pour obtenir: 
        \begin{align*}
            q_1^{*c} &= 200-\frac{2}{5}\left( \underbrace{160 - \frac{2q_1^{*c}}{5}}_{=q_2^{mr}} \right)   
            \Leftrightarrow \frac{21q_1^{*c}}{25} = 136     \Rightarrow    q_1^{*c} \approx 161.9
         \end{align*}
         d'où, 
         \begin{align*}
            q_2^{*c} &= q_2^{mr}(q_1^{*c}) \approx 95.2.
        \end{align*}
        \item On calcule aussi $Q^{*c} = q_1^{*c} + q_2^{*c}\approx 257.1$ et $P^{*c} = P(Q^{*c}) \approx 24.3$.
        \item Finalement comme: 
        \begin{align*}
          \pi_1(q_1^{*c}) &=  P^{*c}q_1^{*c} - c_1(q_1^{*c}) \approx 3278.9 > \pi_1(q_1^L) = 0
        \end{align*}
        la stratégie de prix limite par 1 n'apparaît pas crédible du point de vue de 2, car 1 a intérêt à une
         concurrence à la Cournot qui lui permet un profit non nul. 
    \end{itemize}
\end{frame}
\end{document}

























































































































































































































 
 
 
 

 
 
 
 
 


 
 






































