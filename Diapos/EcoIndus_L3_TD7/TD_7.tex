%\documentclass[ignorenonframetext, compress, 9pt, xcolor=svgnames]{beamer} 
\input{../Config_diapos}
\usepackage{color}
\usepackage{tikz}
\usetikzlibrary{shapes.geometric, arrows}
\usepackage{enumerate}   
\usepackage{multirow}
%\setbeamersize{text margin left=1.5em,text margin right=1.5em} 
%\setbeamersize{text margin left=1.2cm,text margin right=1.2cm} 
\setbeamersize{text margin left=1.5em,text margin right=1.5em} 
%\usepackage{xr}
%\externaldocument{Econometrie1_UGA_P2e}
  \usepackage{eso-pic}
%\newcommand\AtPagemyUpperLeft[1]{\AtPageLowerLeft{%
%\put(\LenToUnit{0.9\paperwidth},\LenToUnit{0.85\paperheight}){#1}}}
%\AddToShipoutPictureFG{
 % \AtPagemyUpperLeft{{\includegraphics[width=1.1cm,keepaspectratio]{logoUGA2020.pdf}}}
%}%

%\setbeamercolor{title}{fg=black}
%\setbeamercolor{frametitle}{fg=black}
%\setbeamercolor{section in head/foot}{fg=black}
%\setbeamercolor{author in head/foot}{bg=Brown}
%\setbeamercolor{date in head/foot}{fg=Brown}
\setbeamertemplate{section page}
{
    \begin{centering}
    \begin{beamercolorbox}[sep=11pt,center]{part title}
    \usebeamerfont{section title}\thesection.~\insertsection\par
    \end{beamercolorbox}
    \end{centering}
}
%\titlegraphic{\includegraphics[width=1cm]{logoUGA2020.pdf}}
\title[]{ \textbf{Économie Industrielle}\footnote{responsable du cours: Sylvain Rossiaud} \\ (UGA, L3 Éco, S2) \\ }
\subtitle{Travaux dirigés: No 7\\ FONDAMENTAUX DE L'ÉCONOMIE DES CONTRATS\\
(éléments de correction)}
\date{\today}
\author{Michal W. Urdanivia\inst{*}}
\institute{\inst{*}UGA, Facult\'e d'\'Economie, GAEL, \\
e-mail:
 \href{
     mailto:michal.wong-urdanivia@univ-grenoble-alpes.fr}{michal.wong-urdanivia@univ-grenoble-alpes.fr}}

%\titlegraphic{\includegraphics[width=1cm]{logoUGA2020.pdf}
%}

\begin{document}

%%% TIKZ STUFF
\usetikzlibrary{positioning}
\usetikzlibrary{snakes}
\usetikzlibrary{calc}
\usetikzlibrary{arrows}
\usetikzlibrary{decorations.markings}
\usetikzlibrary{shapes.misc}
\usetikzlibrary{matrix,shapes,arrows,fit,tikzmark}
\usetikzlibrary{shapes}
\usetikzlibrary{shapes.geometric, arrows}
\tikzset{   
        every picture/.style={remember picture,baseline},
        every node/.style={anchor=base,align=center,outer sep=1.5pt},
        every path/.style={thick},
        }
\newcommand\marktopleft[1]{
    \tikz[overlay,remember picture] 
        \node (marker-#1-a) at (-.3em,.3em) {};%
}
\newcommand\markbottomright[2]{%
    \tikz[overlay,remember picture] 
        \node (marker-#1-b) at (0em,0em) {};%
}
\tikzstyle{every picture}+=[remember picture] 
\tikzstyle{mybox} =[draw=black, very thick, rectangle, inner sep=10pt, inner ysep=20pt]
\tikzstyle{fancytitle} =[draw=black,fill=red, text=white]
\tikzstyle{observed}=[draw,circle,fill=gray!50]

\begin{frame}
\titlepage
\end{frame}
\begin{frame}
 \tableofcontents
    \end{frame}
%\begin{frame}
%\frametitle{Contenu}
%\tableofcontents[pausesections, pausesubsections]
%\end{frame}

%\section{Qu'est-ce que l’économétrie ? A quoi (à qui) ça sert ?}
%\frame{\sectionpage}
%\begin{frame}
%  \tableofcontents  
%\end{frame}


\section{Exercice d’application}
\frame{\sectionpage}
\begin{frame}[allowframebreaks]{Illustration du "problème d’Akerlof"}
\begin{itemize}
\item \textbf{\underline{Remarque}}: cette correction reprend "largement" celle d'Adrien Hervouet en 2021.
\end{itemize}
\begin{enumerate}
\item Prix auquel les consommateurs acceptent d’acheter un bien sur le marché quand ils ne peuvent pas
 faire la différence entre biens de bonne qualité(avec une probabilité de $0.2$) 
 et biens de mauvaise qualité(avec une probabilité de $0.8$): 
 \begin{align*}
    p^{e}&= \underbrace{0.2\times 100 + 0.8\times 20}_{\text{prix espéré}} = 36.
 \end{align*}
 (où l'indice "e" désigne "espéré")
 \item Effets de l’asymétrie d’information:
 \begin{enumerate}[$\star$]
    \item Disponibilité à payer ($=36$) est inférieure au coût de production d'un bien de bonne 
    qualité($=40$).
    \item Conséquence: le firmes produisant le bien de bonne qualité n'ont 
    pas intérêt à le produire(absence de bénéfice), et sortent du marché. 
    \item Conséquence: le pourcentage/probabilité d'avoir un bien de 
    bonne qualité est désormais 0 et celle d'avoir un bine de mauvaise qualité est 1,
     d'où il en résulte une disponibilité à payer de $0\times 100 + 1\times 20=20$.
    \item  En résumé: seul sont produits des biens de mauvaise qualité, ce qui illustre le type d'échec de 
    coordination marchande mis en évidence par \href{https://fr.wikipedia.org/wiki/George_Akerlof}{Akerlof}. 
    En l'occurrence celui de la \textbf{selection adverse}.
\end{enumerate}

\framebreak
\begin{enumerate}[$\star$]
\item On peut poursuivre en prenant le cas où dans l'exercice les probas sont 0.5 et 0.5. 
\item Alors la disponibilité à payer est  $20\times 0.5 + 100\times 0.5 = 60>40$
 \item Identification du problème de \textbf{l’équilibre mélangeant}:
 \begin{enumerate}[$\star$]
 \item Toutes les entreprises sont incitées à participer à l’échange. 
 \item Néanmoins, le prix ne permet pas de signaler aux consommateurs
  si le bien est de bonne ou de mauvaise qualité.
 \end{enumerate}
\end{enumerate}
\framebreak

\item Les firmes choisissent la qualité de leurs biens: 
\begin{enumerate}[$\star$]
    \item  \textbf{Problème d’action cachée et d’aléa moral}.
    \item Dans un jeu à un coup et avec informations parfaite, les firmes produisent de le bien de
     bonne qualité.
     \item Avec information imparfaite, 
     dans un jeu à un coup, toutes les firmes produisent de la 
     mauvaise qualité et imitent la tarification d’une firme qui produirait de la bonne qualité.
     \item L’asymétrie d’information(sur la qualité entre firmes qui la connaissent et consommateurs qui ne la connaissent pas) 
     induit un problème d’aléa moral et de sous-production des biens de bonne qualité.
\end{enumerate}

\framebreak

\item \textbf{(quelques) Outils de la science économique, traiter l’asymétrie d’information}.
\begin{enumerate}[$\star$]
    \item \href{https://fr.wikipedia.org/wiki/Signal\_(\%C3\%A9conomie)}{La théorie du signal}: 
    \begin{enumerate}[$\star$]
        \item Le problème de la crédibilité des stratégies de signalisation.
        \item Il ne doit pas être rentable pour la firme ayant/produisant de la mauvaise qualité d’imiter 
        le signal (tarifaire ou publicitaire) envoyé par la firme ayant/produisant B. 
        \item Il s’agit de retrouver un équilibre séparateur et non plus un équilibre mélangeant.
        \item Exemples de signal crédible:  dépenses publicitaires pour informer de la qualité, 
        jeu répété et phénomène de réputation, garantie et/ou label privé, norme publique.
    \end{enumerate}
    \item Remarque: une façon de traiter le problème d’Akerlof peut consister à sortir 
    de la coordination par le marché anonyme et décentralisée.
\end{enumerate}
    \end{enumerate}
\end{frame}

\section{Fondamentaux de la théorie des coûts de transaction}
\frame{\sectionpage}
\begin{frame}[allowframebreaks]{Économie des Organisations Économie Industrielle}
    \begin{itemize}
    \item L'E.O. est une branche de la nouvelle économie industrielle.
    \item Elle étudie la variété des arrangements qui permettent d’assurer la production et l’échange dans une économie de marché décentralisée,
    \item Cela depuis l’entreprise jusqu’au marché en passant par les modes hybrides (contrats inter-entreprises de moyen-long terme, \ldots).
    \item L’économie organisationnelle cherche à proposer une vision cohérente de la façon dont se structure l’activité de production et d’échange dans des économies de marché complexes.
    \end{itemize}
\end{frame}
\begin{frame}[allowframebreaks]{R. Coase: Apports}
    \begin{itemize}
        \item L’économie organisationnelle s’est construite en reprenant,
         parfois de manière critique, les quatre grandes hypothèses/intuitions 
         développées par Coase dans son article de 1937.
         \begin{enumerate}[$\star$]
      \item Comment expliquer théoriquement l’existence des firmes?
      \item Coase pose ainsi la problématique : “ Mais si la coordination est effectuée par le système de prix, pourquoi une telle organisation serait-elle nécessaire ? 
      Pourquoi ces îlots de pouvoir conscient existent-ils ? ”
         \end{enumerate}
         \item Plusieurs idées sont alors développées.

         \begin{enumerate}[$\star$]
            \item \textbf{Idée 1}: la firme et le marché constituent des modes différents de coordination.
            \item  p. 136:
            \begin{mdframed}
                “ Hors de la firme, les mouvements de prix dirigent la production, 
                laquelle se voit coordonnée à travers une série de transaction intervenant
                 sur le marché. A l’intérieur de la firme, ces transactions de marché sont éliminées 
                 et l’entrepreneur-coordinateur qui dirige la production se voit substitué à la 
                 structure compliquée du marché et de ses transactions d’échange.
                  Il est clair que ce sont des méthodes alternatives de coordination de la production ”
            \end{mdframed}
            \framebreak
            \item \textbf{Idée 2}: L’autorité de l’entrepreneur-coordinateur.
            \begin{enumerate}[$\star$]
                \item Ce qui caractérise la firme, c’est l’existence d’un pouvoir
                  d’autorité, la firme est une organisation hiérarchique.
                \item  Le contrat d’emploi institue la relation de subordination de 
                l’employé à l’employeur, c-a-d le droit conféré à l’entrepreneur de
                 diriger et de contrôler les actions de chacun.
            \end{enumerate}
            \item \textbf{Idée 3}: 
            \begin{enumerate}[$\star$]
            \item La firme en tant que mode alternatif de coordination existe au sein d’une économie 
            de marché en raison de l’existence de \textbf{coûts de transaction}.  
            \item c.à.d., de coûts d’utilisation du système de prix.   
            \item p. 139:
            \begin{mdframed}
                “La principale raison qui rend avantageuse la création d’une entreprise 
                paraît être qu’il existe un coût à l’utilisation du mécanisme de prix ”.
            \end{mdframed}
            \item Coût de découverte du prix adéquat
            \item Coût de négociation et de conclusion des contrats 
            \item Coût d’adaptation des contrats de long terme.
         \end{enumerate}
         \framebreak
         \item \textbf{Idée 4}:
         \begin{enumerate}[$\star$]
            \item L’économie doit avoir pour objet l’identification des variables qui 
            expliquent le choix efficace des modes d’organisation d’une transaction.
            \item S’il existe des coûts de transaction inhérents au marché, “ 
            pourquoi alors la production toute entière n’est-elle 
            pas le fait d’une seule grande entreprise ? ”.
           \item Développement d’une entreprise-> hausse des coûts administratifs,
            plus d’erreur de décision...
            \item un raisonnement marginaliste, p. 145:
            \begin{mdframed}
            “ une entreprise tendra à s’agrandir jusqu’à 
            ce que les coûts d’organisation de transaction 
            supplémentaires en son sein deviennent égaux au coût de 
            réalisation de cette même transaction par le biais d’un échange sur le marché ”
            \end{mdframed}
         \end{enumerate}
        \end{enumerate}
    \end{itemize}
    \end{frame}
    \begin{frame}[allowframebreaks]{R. Coase: Conclusion}
        \begin{itemize}
            \item Au final, ces idées sont au cœur du développement 
            de l’économie organisationnelle. Plus précisément :
            \begin{enumerate}[$\star$]
                \item Il existe une diversité de mode de coordination 
                des transactions au sein d’une économie de marché. 
                Il importe alors de les identifier et de les caractériser.
                \item Ce choix entre ces différents modes 
                d’organisation demeure un choix guidé par la rationalité des acteurs
                 : minimisation des coûts de transaction. Ceci ouvre une problématique 
                 de recherche relative au choix organisationnel efficace de la part des acteurs.
            \end{enumerate}
        \end{itemize}
    \end{frame}

    \begin{frame}[allowframebreaks]{Théorie des coûts de transactions (O. Willamson)}
        \begin{itemize}
            \item La théorie des coûts de transaction constitue aujourd’hui la grille d’analyse théorique principale 
            en économie organisationnelle.
            \begin{enumerate}[$\star$]
            \item O. Williamson s’attache à caractériser les trois grandes modes d’organisation d’une transaction.
            \item Opposition firme/marché.
            \item Et forme hybride. (ex : franchise, sous-traitance)
            \end{enumerate}
            \item  O. Williamson reprend et systématise l’intuition de Coase :
             le choix efficace entre ces différents modes d’organisation est tranché 
             par le critère de minimisation des coûts de transaction.

             \framebreak 

             \item Les coûts de transaction renvoient aux coûts de contractualisation des échanges, 
             les coûts comparatifs de planification, d’adaptation et de contrôle des tâches.
             \begin{enumerate}[$\star$]
              \item CT ex ante : coûts de recherche d’un partenaire, coûts d’élaboration du contrat, coûts des garanties accompagnant la transaction.
             \item CT ex post : coûts de suivi du contrat, coûts d’exécution, coûts d’adaptation et de renégociation.
             \end{enumerate}

             \framebreak

             \item Le niveau et la nature des coûts de transaction dépendent des 
             caractéristiques de la transaction, et en particulier 
             du degré de spécificité des actifs propre à la transaction considérée.
             \item Williamson: 
             \begin{mdframed}
                “ La spécificité des actifs se définit en référence au degré 
                avec lequel un actif peut être redéployé 
                pour un autre usage ou par d’autres utilisateurs sans perte de valeur productive. ”
             \end{mdframed}
             \item $\nearrow$ degré de spécificité des actifs, $\nearrow$ difficultés contractuelles et CT.
             \begin{enumerate}[$\star$]
                \item seuil 1 : un mode d’organisation hybride pour minimiser les CT.
                \item seuil 2 : Intégration pour minimiser les CT.
             \end{enumerate}
        \end{itemize}



    \end{frame}
%\begin{frame}[allowframebreaks]{References}
 %   \bibliographystyle{jpe}
  %  \bibliography{../Biblio}
   % \end{frame}
\end{document}